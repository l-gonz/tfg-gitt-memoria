\chapter{Estado del arte}
\label{chap:sota}

\section{Revisión de la literatura}
\label{sec:lit-rev}

\todoin{
Literature review: \\
  > Codecarbon paper \\
  > Research projects into machine learning consumption \\
  > Métodos que han utilizado y resultados \\
  > ChatGPT, LLM, computer vision, cloud, etc. how much energy?
}

\section{Aprendizaje automático}
\label{sec:intro-ml}

\todoin{Mini intro to machine learning:  \\
  > supervised vs unsupervised \\
  > classification vs regression \\
  > modelos vs redes neuronales \\
  > features vs labels \\
  > train vs test \\
  > types of features}

\section{Librerías empleadas}

\subsection{Entorno de desarrollo: Visual Studio Code y WSL}
\label{sec:dev-env}

\todo[inline]{Get better info from other tfg}

VSCode es un editor de código fuente abierto altamente extensible que se ha convertido en una herramienta predilecta entre desarrolladores debido a su interfaz amigable, integración con Git y soporte para una amplia gama de lenguajes de programación y extensiones. WSL (Windows Subsystem for Linux) es una característica de Windows que permite ejecutar un entorno de Linux completo directamente sobre Windows sin la necesidad de una máquina virtual o sistemas de arranque dual.

Es posible combinar VSCode con WSL mediante una extensión específica ofrecida por VSCode para WSL que permite a los usuarios abrir carpetas y archivos directamente en el sistema de archivos de Linux, beneficiándose de las capacidades avanzadas de depuración y análisis de código que ofrece el editor. Además, la extensión proporciona a los desarrolladores acceso a un terminal de Linux real, donde pueden ejecutar y probar su código en el mismo entorno en el que se desplegará. 


\subsection{Codecarbon}

CodeCarbon \cite{codecarbon}\cite{codecarbon-software} es un paquete creado con la intención de permitir a desarrolladores monitorizar las emisiones de dióxido de carbono ($CO_{2}$) producidas por aplicaciones en Inteligencia Artificial y modelos de Aprendizaje Automático, que surge de la motivación de contar con una forma de registrar las enormes cantidades de energía que el auge de la IA ha provocado en la industria. El incremento del rendimiento y la precisión de los modelos de Aprendizaje Automático que se ha producido en años recientes se ha logrado a cambio de la utilización de enormes cantidades de información para conseguir el aprendizaje de los patrones y características subyacentes. Así, los modelos más avanzados emplean cantidades significativas de poder computacional, entrenando en procesadores avanzados durante semanas o meses y consumiendo en el proceso una gran cantidad de energía. Dependiendo de la red eléctrica utilizada, este desarrollo puede comportar la emisión de grandes cantidades de gases de efecto invernadero como el $CO_{2}$.

CodeCarbon estima la huella de carbono de una aplicación medida como kilogramos de $CO_{2}$ equivalentes, o $CO_{2}eq$, una medida estandarizada utilizada para expresar la capacidad de calentamiento global de varios gases de efecto invernadero como la cantidad de $CO_{2}$ que causaría un impacto ambiental equivalente. Para tareas de computación, que emiten $CO_{2}$ por medio de la electricidad que están consumiendo y que es generada como parte de la red eléctrica (por ejemplo, mediante la quema de combustibles fósiles como el carbón) las emisiones de carbono se miden en kilogramos de $CO_{2}$ equivalentes por kilovatio-hora. De esta forma, las emisiones de dióxido de carbono totales se calculan como el producto de la intensidad de carbono de la electricidad utilizada para la computación y la energía consumida por la infraestructura.

La intensidad de carbono de la electricidad se calcula como la media ponderada de las emisiones de las distintas fuentes de energía usadas para generar electricidad, incluyendo combustibles fósiles y renovables. En la herramienta se asigna un valor conocido de dióxido de carbono emitido por kilovatio-hora generado para cada uno de los combustibles (carbón, petróleo y gas natural). Otras fuentes renovables o consideradas como de bajo carbono incluyen la energía solar, hidroeléctrica, biomasa o geotérmica. La intensidad de carbono de cada combustible individual se calcula en base a medidas de generación de carbono y electricidad en los Estados Unidos, y aplicadas de forma generalizada en el resto del mundo. Cada red eléctrica local incluye una mezcla distinta de fuentes de energía y tiene por lo tanto asignada una intensidad de carbono total particular.

% IMAGE: Global distribution of carbon intensity (carbonboard)
% TABLE: Carbon intensity by energy source (Codecarbon/Methodology)
% CITE: CodeCarbon documentation

% _opcional_ : explanation on zero value for low-carbon fuels
% _opcional_ : explanation on power consuption calculation by CPU

\subsection{Scikit-Learn}

Scikit-learn \cite{scikit-learn} es un módulo desarrollado para Python que integra un amplio rango de algoritmos de aprendizaje automático de última generación para problemas tanto supervisados como no supervisados. Este paquete pretende llevar el aprendizaje automático a desarrolladores no especialistas mediante el uso de un lenguaje generalista de alto nivel. Se hace hincapié en la facilidad de uso, el rendimiento, la documentación y la consistencia de la API \cite{scikit-learn-api}. Tiene las mínimas dependencias necesarias y está distribuido bajo la licencia BSD, con el objetivo de incentivar su uso tanto en ambientes educativos como comerciales.

Scikit-learn expone una gran variedad de algoritmos de aprendizaje utilizando una interfaz consistente y orientada a la resolución de tareas, lo que permite una comparación sencilla entre distintos métodos de aprendizaje para una misma aplicación. Al depender del ecosistema científico de Python, puede ser integrado con facilidad en aplicaciones que se salgan del rango tradicional del análisis estadístico de datos. Además, los algoritmos, que han sido implementados en un lenguaje de alto nivel, pueden ser utilizados como bloques de construcción para desarrollar estrategias más complejas que se adecuen a cada caso particular.

\subsection{Microsoft Azure}
\label{subsec:azure}

Microsoft Azure es una plataforma de servicios en la nube que ofrece una amplia gama de herramientas y recursos para el despliegue y gestión de aplicaciones y servicios. Entre sus múltiples servicios, Azure permite a los usuarios crear y gestionar máquinas virtuales (VMs) con diversas configuraciones de recursos, adaptándose a las necesidades específicas de cada proyecto. Esta capacidad es especialmente valiosa en el campo del aprendizaje automático, donde las tareas de entrenamiento y prueba de modelos requieren recursos computacionales significativos y variados. Con Azure, los desarrolladores e investigadores pueden seleccionar configuraciones específicas de CPU, GPU, memoria y almacenamiento para optimizar el rendimiento y costo de sus experimentos de aprendizaje automático. Microsoft Azure ofrece un programa especial para estudiantes llamado "Azure for Students", que proporciona acceso gratuito a una variedad de servicios en la nube. Este programa incluye un crédito inicial de \$100 USD para usar en cualquier servicio de Azure durante 12 meses, sin necesidad de una tarjeta de crédito para registrarse. 

\subsection{Matplotlib}

Matplotlib es una de las librerías más destacadas y utilizadas en el ecosistema de Python para la creación de gráficos y visualizaciones de datos. Esta herramienta permite a los desarrolladores y científicos de datos generar una amplia variedad de gráficos estáticos, animados e interactivos con relativa facilidad. Matplotlib se caracteriza por su flexibilidad y capacidad para crear visualizaciones de alta calidad y personalizables, que van desde simples gráficos de líneas y barras hasta complejas visualizaciones tridimensionales y de mapas de calor. Además, su integración con otras librerías populares de Python, como NumPy y pandas, facilita el proceso de análisis y visualización de datos, convirtiéndola en una opción preferida en ámbitos académicos y profesionales.

El diseño de Matplotlib sigue una filosofía similar a la de MATLAB, lo que hace que sea especialmente accesible para usuarios familiarizados con ese entorno. Sin embargo, a diferencia de MATLAB, Matplotlib es de código abierto y gratuito. Las capacidades avanzadas de esta librería incluyen la personalización detallada de todos los elementos del gráfico, la posibilidad de exportar gráficos en múltiples formatos (como PNG, PDF y SVG), y la creación de gráficos interactivos utilizando bibliotecas adicionales como mpld3 y Plotly. 


%%%%
%% ALGORITHMS, CURRENTLY
%%%%

\section{Modelos utilizados}
\label{sec:models}

\subsection{Modelos lineales: regresión logística}

Los modelos lineales son una clase fundamental de técnicas estadísticas y de aprendizaje automático que se utilizan para predecir una variable objetivo a partir de una o más variables independientes. La característica principal de estos modelos es la relación lineal entre las variables independientes y la variable objetivo. En su forma más simple, el modelo lineal se representa mediante la ecuación $y = \beta_0 + \beta_1 x_1 + \beta_2 x_2 + \dots + \beta_n x_n$, donde $y$ es la variable dependiente, $x_1, x_2, \dots, x_n$ son las variables independientes, y $\beta_1, \beta_2, \dots, \beta_n$ son los coeficientes que representan el impacto de cada variable independiente sobre la variable dependiente.

Dentro de los modelos lineales, la regresión logística es especialmente relevante para problemas de clasificación. A diferencia de la regresión lineal, que se utiliza para problemas de predicción continua, la regresión logística se emplea cuando la variable objetivo es categórica. En su forma binaria más simple, la regresión logística predice la probabilidad de que una observación pertenezca a una de dos posibles categorías. La función logística o sigmoide, 
\begin{equation*}
    P(y=1\mid x) = \frac{1}{1+e^{-\beta_0 + \beta_1 x_1 + \beta_2 x_2 + \dots + \beta_n x_n}},
\end{equation*}
transforma cualquier valor real de la combinación lineal de las variables independientes en un valor entre 0 y 1, lo que se interpreta como una probabilidad. Los parámetros desconocidos $\beta_i$ son estimados habitualmente a través del método de máxima verosimilitud.

El uso de la regresión logística en problemas de clasificación es ventajoso por varias razones. Primero, es un modelo interpretativo, ya que los coeficientes obtenidos pueden proporcionar una idea clara de cómo cada variable independiente afecta la probabilidad de pertenecer a una categoría específica. Además, la regresión logística es relativamente fácil de implementar y eficiente computacionalmente, lo que la hace adecuada para grandes conjuntos de datos. Por último, aunque su capacidad para capturar relaciones complejas es limitada en comparación con modelos no lineales más avanzados, su simplicidad y efectividad la convierten en una opción sólida para establecer una línea base en proyectos de clasificación, permitiendo comparaciones posteriores con modelos más complejos.

La librería scikit-learn proporciona una interfaz para construir y evaluar modelos de regresión logística a través de su clase \texttt{LogisticRegression} \cite{sk-logistic-regression}. Esta clase permite a los usuarios configurar parámetros como la regularización, la penalización y el algoritmo de optimización. Esta implementación admite dos tipos de penalización: \texttt{L1} (Lasso) y \texttt{L2} (Ridge) que ayudan a prevenir el sobreajuste. Además, permite ajustar la fuerza de la regularización a través de un parámetro \texttt{C}. Esta flexibilidad facilita la adaptación del modelo a diferentes problemas y conjuntos de datos. La personalización aumenta al elegir un algoritmo de optimización mediante el parámetro \texttt{solver}, que admite varias opciones que pueden adaptarse al tamaño o el tipo de atributos del conjunto de datos a utilizar. Adicionalmente, \texttt{LogisticRegression} es capaz de manejar problemas de clasificación binaria y multiclase, con diferentes estrategias de acuerdo al algoritmo de resolución escogido.


\subsection{Árboles de decisión: bosque aleatorio}

Otra técnica popular de aprendizaje automático utilizada para problemas tanto de clasificación como de regresión son los árboles de decisión \cite{sk-decision-trees}. Su estructura jerárquica consiste en nodos de decisión que dividen iterativamente el conjunto de datos en subconjuntos más homogéneos, facilitando así la predicción de la variable objetivo. Cada nodo del árbol representa una prueba sobre un atributo específico, y cada rama denota el resultado de la prueba. Los árboles de decisión son fáciles de interpretar y visualizar, lo que los hace muy útiles para entender las decisiones del modelo y comunicar los resultados a personas no técnicas.

Sin embargo, los árboles de decisión tienen algunas limitaciones, como su tendencia a sobreajustarse a los datos de entrenamiento, lo que puede llevar a un desempeño deficiente en datos nuevos. Para abordar estas limitaciones, se emplean técnicas de ensamblado como los bosques aleatorios (\emph{Random Forests}). Un bosque aleatorio es un conjunto de árboles de decisión entrenados sobre diferentes subconjuntos del conjunto de datos y con diferentes características seleccionadas aleatoriamente. La predicción final se obtiene mediante el promedio de las predicciones individuales de los árboles (en el caso de regresión) o mediante un voto mayoritario (en el caso de clasificación). Esta metodología mejora significativamente la precisión del modelo y reduce el riesgo de sobreajuste.

El uso de bosques aleatorios en problemas de clasificación ofrece varias ventajas. Primero, al combinar múltiples árboles, el modelo se vuelve más robusto y generalizable, mitigando la influencia de outliers y ruido en los datos. Además, los bosques aleatorios proporcionan una medida de importancia de las características, lo que permite identificar las variables más relevantes para la predicción. Esta capacidad es especialmente útil en contextos académicos y profesionales donde la interpretación del modelo y la identificación de factores clave son cruciales para la toma de decisiones informadas.

La implementación de un modelo de bosque aleatorio en la librería scikit-learn se ofrece a través de la clase \texttt{RandomForestClassifier} \cite{sk-random-forest}, que permite ajustar diversos parámetros como el número de árboles (\texttt{n\_estimators}), la profundidad máxima de los árboles (\texttt{max\_depth}) y el número mínimo de muestras por hoja (\texttt{min\_samples\_leaf}). 

\subsection{Máquinas de vector soporte (SVM)}

En las máquinas de vector soporte (\emph{Support Vector Machines}, SVM) \cite{sk-svm-theory} el principal objetivo es encontrar el hiperplano óptimo que maximiza el margen entre las diferentes clases en un espacio de características. Este margen es la distancia más amplia posible entre el hiperplano y los puntos de datos más cercanos de cualquier clase, conocidos como vectores soporte. La SVM es particularmente eficaz en espacios de alta dimensionalidad y es robusta frente al sobreajuste, lo que la hace adecuada para conjuntos de datos complejos.

Una de las ventajas clave de las SVM es su capacidad para manejar problemas no lineales mediante el uso de funciones kernel. Estos núcleos transforman los datos en un espacio de mayor dimensión donde un hiperplano lineal puede separarlos. Los tipos de kernel más comunes incluyen el lineal, polinómico, radial (RBF), y sigmoide. Esta flexibilidad permite a las SVM abordar una amplia gama de problemas de clasificación, incluso cuando las relaciones entre las características y las etiquetas son altamente no lineales. Sin embargo, para que funcione de forma adecuada es necesario seleccionar el kernel apropiado y ajustar sus parámetros, por ejemplo, mediante validación cruzada.

En el contexto de problemas de clasificación, las SVM son especialmente útiles debido a su alta precisión y su capacidad para manejar tanto clases linealmente separables como no separables. En escenarios de clasificación binaria, las SVM son capaces de encontrar la frontera de decisión que maximiza la separación entre las dos clases. Para problemas de clasificación multiclase, se pueden aplicar estrategias para descomponer el problema en múltiples problemas de clasificación binaria. Aunque las SVM pueden ser computacionalmente intensivas, especialmente con grandes conjuntos de datos, tienen una gran eficacia en términos de rendimiento y de capacidad para generalizar bien a datos no vistos.

La clase \texttt{SVC} \cite{sk-svm} de scikit-learn permite ajustar parámetros clave como el tipo de kernel, el parámetro de regularización (\texttt{C}) y coeficientes específicos del kernel elegido.

\subsection{Vecinos más cercanos (k-NN)}

El método de vecinos más cercanos (k-Nearest Neighbors, k-NN) es una técnica de aprendizaje supervisado utilizada para resolver problemas tanto de clasificación como de regresión. Este método se basa en la premisa de que objetos similares generalmente se encuentran cerca en el espacio de características. En problemas de clasificación, el algoritmo k-NN asigna una etiqueta a una nueva instancia basándose en las etiquetas de sus k vecinos más cercanos en el conjunto de datos de entrenamiento. La cercanía o similitud entre las instancias se mide generalmente utilizando distancias métricas, como la distancia euclidiana, Manhattan, o Minkowski.

Una de las principales ventajas del k-NN es su simplicidad y facilidad de implementación. A diferencia de otros algoritmos que requieren un proceso de entrenamiento explícito, k-NN es un algoritmo perezoso que almacena todas las instancias del conjunto de entrenamiento y realiza el cálculo de las distancias en el momento de la predicción. Esta característica lo convierte en un método intuitivo y fácil de entender, aunque también implica que puede ser computacionalmente costoso, especialmente con grandes conjuntos de datos. Además, la elección del número de vecinos (k) y la métrica de distancia son cruciales para el rendimiento del modelo. Valores de k demasiado bajos pueden llevar a un sobreajuste, mientras que valores demasiado altos pueden conducir a un subajuste.

En el contexto de problemas de clasificación, el k-NN es especialmente útil en situaciones donde las fronteras de decisión entre clases no son lineales y pueden ser complejas. Debido a su naturaleza basada en instancias, el k-NN puede capturar patrones locales en los datos y adaptarse a cambios en la distribución de las clases. Sin embargo, el rendimiento de k-NN puede verse afectado por la presencia de ruido y características irrelevantes, lo que subraya la importancia de la preprocesamiento de datos, como la normalización y la selección de características. A pesar de estas limitaciones, el k-NN sigue siendo una herramienta valiosa para la clasificación debido a su simplicidad y flexibilidad.

La clase \texttt{KNeighborsClassifier} \cite{sk-knn} de la librería scikit-learn proporciona la interfaz para configurar y utilizar el algoritmo k-NN. Esta clase ofrece la posibilidad de especificar el número de vecinos a considerar mediante el parámetro \texttt{n\_neighbors} y de seleccionar la métrica de distancia adecuada utilizando el parámetro \texttt{metric}, que incluye opciones como la distancia euclidiana y Manhattan. Además, scikit-learn permite ajustar parámetros adicionales como el peso de los vecinos, que puede ser uniforme o basado en la distancia.

\subsection{Naive Bayes (Naive Bayes gaussiano)}

Los métodos Naive Bayes \cite{sk-naive-bayes} son un conjunto de algoritmos de aprendizaje supervisado basados en la aplicación del teorema de Bayes con la suposición "ingenua" (en inglés, \emph{naive}) de independencia condicional entre cada par de características dado el valor de la variable de clase. El teorema de Bayes postula la siguiente relación, dada la variable de clase $y$ y los vectores característica dependientes $x_{1}$ a $x_{n}$:
\begin{equation*}
    P(y \mid x_{1},\dots,x_{n}) = \dfrac{P(y)P(x_{1},\dots,x_{n}\mid y)}{P(x_{1},\dots,x_{n})}
\end{equation*}

Usando la suposición ingenua de independencia condicional de forma que
\begin{equation*}
    P(x_{i} \mid y,x_{1},\dots,x_{i-1},x_{i+1},\dots,x_{n})=P(x_{i}\mid y),
\end{equation*}

para todo $i$, esta relación se simplifica a 
\begin{equation*}
    P(y\mid x_{1},\dots,x_{n})=\dfrac{P(y) \prod^{n}_{i=1} P(x_{i}\mid y)}{P(x_{1},\dots,x_{n})}
\end{equation*}

Como $P(x_{1},\dots,x_{n})$ es constante dada la entrada, se puede utilizar la siguiente regla de clasificación:
\begin{equation*}
    P(y\mid x_{1},\dots,x_{n}) \propto P(y) \prod^{n}_{i=1}P(x_{i}\mid y) \Rightarrow 
    \hat{y} = \arg \max_{y} P(y) \prod^{n}_{i=1}P(x_{i}\mid y),
\end{equation*}

y se puede usar la estimación de máximo a posteriori (MAP) para estimar $P(y)$ y $P(x_{i}\mid y)$; y la primera expresión es entonces la frecuencia relativa de la clase $y$ en el conjunto de entrenamiento.

Los diferentes clasificadores Naive-Bayes difieren sobre todo en la suposición que realicen sobre la distribución de $P(x_{i} \mid y)$. El algoritmo Naive-Bayes gaussiano es una variante específica de Naive-Bayes que se utiliza cuando las características son continuas y se asumen distribuidas según una distribución normal (gaussiana), lo que es común en muchas aplicaciones reales. En este caso, el clasificador estima los parámetros de la distribución (media y varianza) para cada característica en cada clase utilizando los datos de entrenamiento. Durante la clasificación de una nueva instancia, el algoritmo calcula la probabilidad de que esta instancia pertenezca a cada clase basándose en las distribuciones gaussianas estimadas y asigna la clase con la probabilidad a posteriori más alta.

A pesar de sus suposiciones aparentemente simplificadas, los clasificadores Naive-Bayes obtienen buenos resultados en muchos problemas del mundo real, especialmente en aplicaciones como la clasificación de documentos y filtros de spam; y requieren pequeñas cantidades de datos de entrenamiento para estimar los parámetros necesarios. Estos clasificadores pueden llegar a ser extremadamente rápidos comparados a otros métodos más sofisticados. La separación de las características condicionales de clase significa que cada distribución puede ser estimada independientemente como una distribución unidimensional. Esto a su vez ayuda a aliviar problemas provocados por la dimensionalidad de las distribuciones.

La implementación de este algoritmo en la librería scikit-learn se encuentra en la clase \texttt{GaussianNB} \cite{sk-nb-gaussian}.

\subsection{Métodos de ensamblaje (ensemble): máquinas de potenciación de gradiente}

El método de ensamblaje por máquinas de potenciación de gradiente (Gradient Boosting Machines, GBM) es una técnica de aprendizaje automático que combina múltiples modelos débiles para crear un modelo más fuerte y preciso. La idea central de GBM es construir el modelo de forma secuencial, donde cada modelo sucesivo intenta corregir los errores del modelo anterior. En el contexto de problemas de clasificación, cada nuevo árbol de decisión se ajusta a los residuos de los árboles anteriores, utilizando el gradiente de la función de pérdida como guía. Esta estrategia permite que GBM capture relaciones complejas y no lineales en los datos, mejorando considerablemente la precisión del modelo.

Uno de los mayores beneficios de usar máquinas de potenciación de gradiente para problemas de clasificación es su capacidad para manejar datos heterogéneos y características con diferentes escalas y distribuciones. GBM puede ajustarse a los datos de manera muy detallada, lo que lo hace adecuado para problemas donde las relaciones entre las variables no son lineales ni simples. Sin embargo, este alto nivel de ajuste también puede llevar a un sobreajuste si no se controla adecuadamente. Por ello, es crucial utilizar técnicas como la validación cruzada y ajustar parámetros como el número de árboles, la tasa de aprendizaje, y la profundidad máxima de los árboles para encontrar el equilibrio adecuado entre sesgo y varianza.

En problemas de clasificación, GBM es particularmente eficaz debido a su capacidad para manejar clases no balanceadas y proporcionar probabilidades de clasificación en lugar de simplemente etiquetas de clase. Esto es especialmente útil en aplicaciones como la detección de fraudes, diagnóstico médico y otros escenarios donde la probabilidad de pertenecer a una clase particular puede ser más informativa que la clasificación binaria. Además, las implementaciones modernas de GBM, como XGBoost, LightGBM y CatBoost, han optimizado significativamente la velocidad y eficiencia del algoritmo, permitiendo su aplicación en grandes conjuntos de datos y en tiempo real.

La implementación de máquinas de potenciación de gradiente en la librería scikit-learn es accesible a través de la clase GradientBoostingClassifier \cite{sk-gradient-boost} permite a los usuarios ajustar una variedad de hiperparámetros para optimizar el rendimiento del modelo. Los usuarios pueden especificar el número de árboles, la tasa de aprendizaje, y otros parámetros clave para controlar la complejidad y la capacidad de generalización del modelo.

\subsection{Redes neuronales (Deep Neural Networks, DNN) Multi-layer Perceptron}

Las redes neuronales están compuestas por capas de neuronas artificiales que imitan el comportamiento de las neuronas en el cerebro humano. Cada neurona realiza una operación matemática simple sobre la entrada y pasa la salida a la siguiente capa. A través de múltiples capas, la red es capaz de capturar patrones y características complejas en los datos. Estás propiedades pueden aplicarse a situaciones de aprendizaje supervisado mediante el uso de algoritmos como el perceptrón multicapa (Multilayer Perceptron, MLP).

El algoritmo del perceptrón multicapa es una clase de red neuronal supervisada que consiste en una capa de entrada, una o más capas ocultas y una capa de salida. Cada capa está totalmente conectada a la siguiente, y cada conexión tiene un peso ajustable que se optimiza durante el proceso de entrenamiento. En problemas de clasificación, el MLP es especialmente útil debido a su capacidad para aprender representaciones no lineales complejas de los datos. El algoritmo utiliza funciones de activación no lineales, como la función sigmoide o la ReLU (Rectified Linear Unit), que permiten que la red capture y modele relaciones no lineales en los datos. El entrenamiento de un MLP se realiza mediante un proceso llamado retropropagación del error (backpropagation), que ajusta iterativamente los pesos para minimizar una función de pérdida, típicamente la entropía cruzada en problemas de clasificación.

El uso del MLP ofrece varias ventajas. Primero, debido a su estructura de múltiples capas, el MLP puede aprender y generalizar bien a partir de datos complejos y de alta dimensionalidad. Esto es particularmente útil en dominios como el reconocimiento de imágenes, la detección de voz y la clasificación de texto, donde las relaciones entre las características no son triviales. Además, las redes neuronales pueden ser adaptadas a problemas multiclase con facilidad, utilizando técnicas como la salida softmax en la capa final para obtener probabilidades de clase. Sin embargo, el MLP también requiere una cantidad considerable de datos y poder computacional para entrenar eficazmente, y la selección de la arquitectura y los hiperparámetros adecuados puede ser un proceso desafiante que a menudo implica ensayo y error y validación cruzada.

La librería scikit-learn implementa la clase MLPClassifier \cite{sk-multilayer-perceptron} para el algoritmo del perceptrón multicapa. Los usuarios pueden definir parámetros como el número de capas ocultas y el número de neuronas por capa (hidden\_layer\_sizes), la función de activación (activation), y el algoritmo de optimización (solver). Scikit-learn también permite ajustar la tasa de aprendizaje (learning\_rate) y utilizar técnicas de regularización para prevenir el sobreajuste.
\section{Datos utilizados}
\label{sec:datasets}

\todoin{Update with new datasets:  \\
  > Iris, Ionosphere, Banknote, Phoneme, Eeg-eye-state, Electricity \\
  > Covertype, Poker-hand \\
  > For-each: \\
  ---> content explain \\
  ---> size \\
  ---> features \\
  ---> origin, cites \\
  ---> what has been used for before}

Todos los conjuntos de datos utilizados en los análisis realizados están disponibles de forma libre en la web y proceden de dos fuentes: el repositorio de aprendizaje automático de la Universidad de California Irvine y la plataforma OpenML.

El repositorio de aprendizaje automático de la UCI \cite{ml-uci} es una colección de bases de datos, teorías de dominios y generadores de datos que son utilizados por la comunidad de aprendizaje automático para el análisis empírico de algoritmos de aprendizaje automático. El archivo fue creado en 1987 como un servidor FTP por David Aha y otros compañeros estudiantes en la UCI. Desde entonces ha sido ampliamente utilizado por estudiantes, educadores e investigadores de todo el mundo como una fuente primaria de conjuntos de datos para aprendizaje automático.

OpenML \cite{openml} es una plataforma abierta para compartir conjuntos de datos, algoritmos y experimentos, que tiene como objetivo lograr que las investigaciones sobre aprendizaje automático sean más fácilmente accesibles y reutilizables. Esta plataforma contiene un repositorio con más de cinco mil conjuntos de datos y resultados de experimentos que otros usuarios han realizado con ellos. Además, ofrece librerías para integrar la recuperación de sus datos directamente con el código de desarrollo de modelos de aprendizaje. Otras librerías como Pandas para procesado de datos, ofrecen también opciones para descargar los datos fácilmente de la plataforma.

A continuación se describen las características más importantes de los conjuntos de datos empleados.
% ANNEX All attributes

\subsection{Iris}

% This is perhaps the best known database to be found in the pattern recognition literature. Fisher's paper is a classic in the field and is referenced frequently to this day. (See Duda & Hart, for example.) The data set contains 3 classes of 50 instances each, where each class refers to a type of iris plant. One class is linearly separable from the other 2; the latter are NOT linearly separable from each other.

% Predicted attribute: class of iris plant.
% This is an exceedingly simple domain.

% Attribute Information:
% 1. sepal length in cm
% 2. sepal width in cm
% 3. petal length in cm
% 4. petal width in cm
% 5. class: 
%    -- Iris Setosa
%    -- Iris Versicolour
%    -- Iris Virginica

Iris esta entre las primeras bases de datos recogidas, y es una de las más conocidas y utilizadas en la literatura sobre reconocimiento de patrones. Los datos originales fueron publicados por R.A. Fisher en 1936, y la versión utilizada en este proyecto procede de la UCI (1988) \cite{iris-dataset}.
El conjunto contiene tres clases distintas con 50 ejemplos cada una, donde cada clase se refiere a una especie del género de plantas Iris. Cada ejemplo contiene cuatro atributos que describen la longitud y la anchura del sépalo y el pétalo de la planta en centímetros. Las tres opciones de clasificación son Iris Setosa, Iris Versicolour e Iris Virginica.

Se trata de un campo de clasificación extremadamente simple, donde una de las clases es linealmente separable de las otras dos, y estas últimas son separable entre sí de forma no lineal.



\subsection{Ionosfera}

Se trata de un conjunto de datos con 351 muestras procedente del repositorio de la UCI\footnote{\url{https://archive.ics.uci.edu/ml/datasets/Ionosphere}}. Contiene datos de radar obtenidos por el grupo de física espacial de la Universidad John Hopkins y donados por Vince Sigillito en 1989. El sistema radar está ubicado en Goose Bay, Labrador y consiste en un array de 16 antenas de alta frecuencia. El objetivo es la medición de electrones libres en la ionosfera y su clasificación binaria entre "buenas" respuestas del radar que indican evidencia de algún tipo de estructura en la ionosfera y "malas" respuestas en las que las señales simplemente pasan a través de la ionosfera. Las señales recibidas se procesaron utilizando una función de autocorrelación con el tiempo de pulso y el número de pulso como argumentos y cada una de las muestras del conjunto de datos está descrita por dos atributos continuos para cada uno de los 17 números de pulso, correspondientes al valor complejo obtenido de la señal electromagnética compleja. Hay por lo tanto un total de 34 características continuas por muestra.

\subsection{Billetes}

Este conjunto de datos contiene información extraída de 1372 imágenes tomadas para evaluar un procedimiento de autenticación de billetes. Fue donado en Agosto de 2012 por Volker Lohweg de la Universidad de Ciencias Aplicadas de Ostwestfalen-Lippe, Alemania, al repositorio de la UCI\footnote{\url{https://archive.ics.uci.edu/ml/datasets/banknote+authentication}}. Para la digitalización de las imágenes tomadas se empleó una cámara industrial normalmente utilizada para la inspección de impresiones. Las imágenes finales tienen un tamaño de 400x400 píxeles con una resolución de alrededor de 660 dpi en escala de grises. Posteriormente, se empleó una herramienta de transformada ondícula para extraer cuatro características continuas de las imágenes.

\subsection{Fonemas}
\subsection{Electroencefalograma}
\subsection{Electricidad}
% \subsection{Tipo de cubierta}
% \subsection{Mano de póquer}


%\cleardoublepage