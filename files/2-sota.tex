\chapter{Estado del arte}
\label{chap:sota}

\section{Revisión de la literatura}
\label{sec:lit-rev}

% Herramientas
% Estudios comparativos
% Métodos de medición

El impacto energético del aprendizaje automático ha empezado ha desarrollarse recientemente como una preocupación significativa debido a la creciente complejidad y escala de los modelos utilizados en este campo. A pesar de la mayor concienciación existente entorno al consumo de los grandes modelos computacionales, muchas de las investigaciones actuales se ocupan únicamente de obtener las mejores predicciones posibles, independientemente del coste computacional que esto suponga. Sin embargo, algunas investigaciones recientes sí han intentado cuantificar y mitigar el consumo energético asociado con el entrenamiento y la inferencia de modelos de aprendizaje automático, desarrollando herramientas que permitan a los científicos de datos medir las emisiones de sus modelos. \cite{lottick2019} fue uno de los primeros artículos en proponer una declaración sistematizada de las emisiones de carbono como parte del proceso de elección de modelos de aprendizaje, proponiendo un modelo de informe energético fácilmente accesible para desarrolladores sin necesidad de análisis a nivel industrial.

Varios estudios han reportado el consumo energético de modelos de aprendizaje automático, destacando el alto costo energético de los grandes modelos de lenguaje natural y visión por computador. Por ejemplo, un estudio de 2019 \cite{strubell2019nlp} encontró que el entrenamiento de un modelo Transformer para búsqueda de arquitectura neuronal (NAS, una técnica para automatizar el diseño de redes neuronales artificiales) puede consumir tanta energía como el gasto de varios automóviles durante toda su vida útil. Otros estudios como \cite{dodge2022cloud} se centran en el gasto de la inteligencia artificial cuando los modelos son entrenados en instancias en la <<nube>> (centros de datos distribuidos) y como pueden reducirse las emisiones de $CO_2$ considerando la elección de región geográfica y hora del día en la que se lleva a cabo el entrenamiento. Estos hallazgos han impulsado la investigación en técnicas de optimización y eficiencia, tales como la poda de redes neuronales, la cuantización de pesos, y el uso de hardware especializado como TPU (Tensor Processing Units) y FPGA (Field-Programmable Gate Arrays), que pueden reducir significativamente el consumo energético.

A día de hoy existen un gran número de herramientas para medir el consumo de energía en los modelos de aprendizaje. Sin embargo, el uso de esta métrica no se ha generalizado todavía como parte del proceso. En \cite{eva2019review}, se hace una revisión de varias de estas herramientas y los métodos más destacados para estimar el consumo. En ella se observa que se debe encontrar un equilibrio entre las herramientas que proporcionan un resultado detallado de la energía con una gran carga adicional en el desarrollo y aquellas que proporcionan un resultado muy aproximado pero de fácil disponibilidad. Son estas últimas las que serán más interesantes para integrar de forma sencilla en un proceso de elección de modelos de aprendizaje, al proporcionar información en tiempo real sobre los algoritmos utilizados.

% Los métodos empleados para medir el consumo energético en el aprendizaje automático varían en precisión y granularidad. Codecarbon, por ejemplo, utiliza una combinación de información sobre la carga de trabajo del CPU/GPU, la ubicación geográfica de los centros de datos, y las emisiones promedio de CO2 por kWh de electricidad en diferentes regiones. Este enfoque permite estimaciones en tiempo real del consumo energético y las emisiones de carbono asociadas a las tareas de aprendizaje automático. Otros métodos más directos incluyen el uso de medidores de energía que se conectan directamente a los dispositivos de hardware utilizados para el entrenamiento de modelos, proporcionando datos más precisos pero con un alcance limitado a configuraciones específicas.

%%%% HERRAMIENTAS

Entre los proyectos destacados se encuentra Codecarbon\cite{codecarbon}, una iniciativa que se ha centrado en medir la huella de carbono de las actividades computacionales en aprendizaje automático mediante un paquete para Python fácilmente integrable con librerías existentes. Otras herramientas (\cite{getzner2023accuracy}) proponen un método de estimación sin necesidad de llegar a entrenar los modelos, mediante la recopilación previa de medidas de calidad de consumo energético en modelos de base. El \emph{Machine Learning Emissions Calculator} (MLCO2)\cite{lacoste2022mlco2} presenta otro modelo de medida basado en la localización, la red eléctrica y el tiempo utilizado por el servidor responsable del entrenamiento. Otras herramientas disponibles para Python incluyen \emph{experiment-impact-tracker} \cite{henderson2020tracker}, Carbontracker \cite{anthony2020carbontracker} y Eco2AI \cite{budennyy2022eco2ai}. Estos paquetes utilizan métodos similares de estimación de la energía utilizada por CPUs y GPUs y su correlación con las emisiones de carbono de acuerdo a la red eléctrica utilizada.

Los resultados obtenidos de estas investigaciones resaltan la necesidad de considerar la eficiencia energética como un criterio fundamental en el diseño y entrenamiento de modelos de aprendizaje automático. Numerosas iniciativas han mostrado que, mediante la optimización de código y la selección de infraestructuras más sostenibles, se pueden lograr reducciones significativas en el consumo de energía y emisiones de carbono. Otros estudios han demostrado que la implementación de modelos más eficientes no necesariamente compromete la precisión, sugiriendo que es posible alcanzar un balance entre rendimiento y sostenibilidad.

En conclusión, la literatura destaca la importancia de desarrollar métodos y tecnologías que reduzcan el consumo energético en el aprendizaje automático. La medición del consumo eléctrico o las emisiones de carbono son cruciales para aumentar la concienciación sobre este problema y proporcionar herramientas prácticas para medir y reducir la huella de carbono. A medida que la demanda de modelos grandes y complejos continúa creciendo, es imperativo que la comunidad de investigación se enfoque en soluciones sostenibles que equilibren la eficiencia energética con el rendimiento del modelo.

\section{Introducción al aprendizaje automático}
\label{sec:intro-ml}

El aprendizaje automático se ha consolidado como una disciplina fundamental dentro del campo de la ingeniería y la informática, debido a su capacidad para desarrollar sistemas que pueden aprender y mejorar a partir de la experiencia sin ser explícitamente programados. En el contexto de este trabajo de fin de grado, se desarrollará una aplicación para medir el consumo energético de proyectos de aprendizaje automático, un aspecto crítico dado el creciente uso de estos modelos en diversas aplicaciones y su impacto ambiental. Para explicar su funcionamiento, es necesario describir primero los conceptos básicos de esta materia.

El aprendizaje automático puede dividirse principalmente en dos categorías: aprendizaje supervisado y no supervisado. En el aprendizaje supervisado, el modelo se entrena utilizando un conjunto de datos etiquetados, es decir, cada muestra del conjunto de datos está asociada con una etiqueta que representa el resultado esperado. Las tareas más comunes en este tipo de aprendizaje son la clasificación, donde el objetivo es asignar muestras a una de varias categorías predefinidas, y la regresión, donde se predice un valor continuo. 
En contraste, el aprendizaje no supervisado no utiliza etiquetas. En su lugar, el modelo intenta identificar patrones y estructuras inherentes en los datos. Las tareas típicas en este enfoque incluyen la agrupación (clustering), donde el objetivo es agrupar muestras similares, y la reducción de dimensionalidad, que busca simplificar los datos manteniendo su estructura esencial. Este proyecto se centrará en tareas de clasificación mediante aprendizaje supervisado.

Dentro de las tareas de clasificación se pueden distinguir dos casos: la clasificación binaria y la multiclase. La clasificación multiclase es una técnica de aprendizaje automático supervisado en la que el objetivo es categorizar cada muestra de datos en una de tres o más clases posibles. A diferencia de la clasificación binaria, que se limita a dos clases, la clasificación multiclase es más compleja debido a varios factores. En primer lugar, el aumento en el número de clases incrementa la dificultad para separar las categorías correctamente, ya que el modelo debe aprender a distinguir entre múltiples fronteras de decisión. Además, el desequilibrio entre clases puede ser más pronunciado, lo que complica el entrenamiento del modelo. Por último, las métricas de evaluación se vuelven más complejas, ya que es necesario considerar el rendimiento del modelo en todas las clases para obtener una visión completa de su precisión y robustez. 

Entre los modelos más habituales en el aprendizaje automático se encuentran los árboles de decisión, los modelos de Naive Bayes, las máquinas de soporte vectorial y las redes neuronales. Las redes neuronales, inspiradas en la estructura del cerebro humano, son particularmente relevantes en problemas complejos de reconocimiento de patrones y se destacan por su capacidad de aproximar funciones no lineales mediante la composición de varias capas de neuronas artificiales. Otros modelos, como los bosques aleatorios (Random Forests) y los modelos de potenciación de gradiente (Gradient Boosting Machines), también son ampliamente utilizados debido a su capacidad para mejorar la precisión a través de la combinación de múltiples predictores.

La calidad y estructura de los conjuntos de datos juegan un papel crucial en el éxito de los modelos de aprendizaje automático. Un conjunto de datos típico está compuesto por muestras (o instancias), cada una descrita por una serie de atributos (o características) y, en el caso de aprendizaje supervisado, asociada con una etiqueta, la variable objetivo. Los atributos pueden ser de diversos tipos: continuos, categóricos, binarios u ordinales.

Para evaluar el rendimiento de un modelo, los datos se dividen generalmente en un conjunto de entrenamiento y un conjunto de prueba. El conjunto de entrenamiento se utiliza para ajustar los parámetros del modelo, mientras que el conjunto de prueba se emplea para evaluar su capacidad de generalización a datos no vistos. Este proceso ocurre en varias etapas clave. Primero, los datos se recopilan y preprocesan para asegurar su calidad y pertinencia. Luego, el modelo se entrena utilizando el conjunto de entrenamiento, ajustando sus parámetros internos para minimizar el error en la predicción. Este ajuste se realiza mediante algoritmos de optimización, como el descenso del gradiente, que iterativamente modifica los parámetros para encontrar la configuración que produce el menor error. Finalmente, el modelo se utiliza para predecir la variable objetivo en los datos del conjunto de prueba y se evalúa comparando las predicciones con las etiquetas reales de los datos.

La precisión de un modelo de aprendizaje automático está influenciada por diversos factores. La calidad y cantidad de los datos de entrenamiento son cruciales; datos ruidosos o insuficientes pueden llevar a modelos inexactos. Además, la elección del modelo y sus hiperparámetros, como la profundidad de los árboles en un bosque aleatorio o la tasa de aprendizaje en un modelo de potenciación de gradiente, también juegan un papel determinante. La complejidad del modelo debe equilibrarse cuidadosamente para evitar el sobreajuste, donde el modelo se ajusta demasiado bien a los datos de entrenamiento y falla en generalizar a nuevos datos.
Otro problema común en el aprendizaje automático es el desequilibrio de clases, donde una clase puede estar significativamente subrepresentada en el conjunto de datos, afectando la precisión del modelo, que tiende a ignorar las clases minoritarias. Algunas estrategias para subsanar este problema son la penalización de las clases mayoritarias mediante la aplicación de pesos distintos a cada clase o la utilización de métodos de ensamblaje que no son sensibles al desequilibrio de clases.

A medida que aumentan la escala y complejidad de los modelos, el consumo energético de los modelos de aprendizaje automático se volverá un aspecto más crítico. Modelos más complejos, como las redes neuronales profundas, requieren un mayor poder computacional y, por ende, un mayor consumo de energía durante el entrenamiento y la inferencia. Otros factores que afectan el consumo energético incluyen la arquitectura del modelo, la cantidad de datos, el hardware utilizado (como CPUs, GPUs o TPUs) y la eficiencia de los algoritmos de optimización. Medir y optimizar el consumo energético es esencial para desarrollar soluciones sostenibles y escalables, especialmente en aplicaciones a gran escala como el procesamiento de grandes volúmenes de datos o la inteligencia artificial en tiempo real.

En conclusión, este trabajo de fin de grado se centrará en la implementación de una aplicación para medir el consumo energético de proyectos de aprendizaje automático. Dicha aplicación considerará varios aspectos técnicos y metodológicos del aprendizaje automático, incluyendo la selección y evaluación de modelos, la gestión de conjuntos de datos, y la mitigación de problemas comunes, con el objetivo de contribuir a la eficiencia energética y sostenibilidad en el desarrollo de estas tecnologías.

\section{Librerías empleadas}

\subsection{Entorno de desarrollo: Visual Studio Code y WSL}
\label{sec:dev-env}

VSCode es un editor de código fuente abierto altamente extensible que se ha convertido en una herramienta predilecta entre desarrolladores debido a su interfaz amigable, integración con Git y soporte para una amplia gama de lenguajes de programación y extensiones. WSL (Windows Subsystem for Linux) es una característica de Windows que permite ejecutar un entorno de Linux completo directamente sobre Windows sin la necesidad de una máquina virtual o sistemas de arranque dual.

Es posible combinar VSCode con WSL mediante una extensión específica ofrecida por VSCode para WSL que permite a los usuarios abrir carpetas y archivos directamente en el sistema de archivos de Linux, beneficiándose de las capacidades avanzadas de depuración y análisis de código que ofrece el editor. Además, la extensión proporciona a los desarrolladores acceso a un terminal de Linux real, donde pueden ejecutar y probar su código en el mismo entorno en el que se desplegará. 


\subsection{Codecarbon}

CodeCarbon \cite{codecarbon}\cite{codecarbon-software} es un paquete creado con la intención de permitir a desarrolladores monitorizar las emisiones de dióxido de carbono ($CO_{2}$) producidas por aplicaciones en Inteligencia Artificial y modelos de Aprendizaje Automático, que surge de la motivación de contar con una forma de registrar las enormes cantidades de energía que el auge de la IA ha provocado en la industria. El incremento del rendimiento y la precisión de los modelos de Aprendizaje Automático que se ha producido en años recientes se ha logrado a cambio de la utilización de enormes cantidades de información para conseguir el aprendizaje de los patrones y características subyacentes. Así, los modelos más avanzados emplean cantidades significativas de poder computacional, entrenando en procesadores avanzados durante semanas o meses y consumiendo en el proceso una gran cantidad de energía. Dependiendo de la red eléctrica utilizada, este desarrollo puede comportar la emisión de grandes cantidades de gases de efecto invernadero como el $CO_{2}$.

CodeCarbon estima la huella de carbono de una aplicación medida como kilogramos de $CO_{2}$ equivalentes, o $CO_{2}eq$, una medida estandarizada utilizada para expresar la capacidad de calentamiento global de varios gases de efecto invernadero como la cantidad de $CO_{2}$ que causaría un impacto ambiental equivalente. Para tareas de computación, que emiten $CO_{2}$ por medio de la electricidad que están consumiendo y que es generada como parte de la red eléctrica (por ejemplo, mediante la quema de combustibles fósiles como el carbón) las emisiones de carbono se miden en kilogramos de $CO_{2}$ equivalentes por kilovatio-hora. De esta forma, las emisiones de dióxido de carbono totales se calculan como el producto de la intensidad de carbono de la electricidad utilizada para la computación y la energía consumida por la infraestructura.

La intensidad de carbono de la electricidad se calcula como la media ponderada de las emisiones de las distintas fuentes de energía usadas para generar electricidad, incluyendo combustibles fósiles y renovables. En la herramienta se asigna un valor conocido de dióxido de carbono emitido por kilovatio-hora generado para cada uno de los combustibles (carbón, petróleo y gas natural). Otras fuentes renovables o consideradas como de bajo carbono incluyen la energía solar, hidroeléctrica, biomasa o geotérmica. La intensidad de carbono de cada combustible individual se calcula en base a medidas de generación de carbono y electricidad en los Estados Unidos, y aplicadas de forma generalizada en el resto del mundo. Cada red eléctrica local incluye una mezcla distinta de fuentes de energía y tiene por lo tanto asignada una intensidad de carbono total particular.

% IMAGE: Global distribution of carbon intensity (carbonboard)
% TABLE: Carbon intensity by energy source (Codecarbon/Methodology)
% CITE: CodeCarbon documentation

% _opcional_ : explanation on zero value for low-carbon fuels
% _opcional_ : explanation on power consuption calculation by CPU

\subsection{Scikit-Learn}

Scikit-learn \cite{scikit-learn} es un módulo desarrollado para Python que integra un amplio rango de algoritmos de aprendizaje automático de última generación para problemas tanto supervisados como no supervisados. Este paquete pretende llevar el aprendizaje automático a desarrolladores no especialistas mediante el uso de un lenguaje generalista de alto nivel. Se hace hincapié en la facilidad de uso, el rendimiento, la documentación y la consistencia de la API \cite{scikit-learn-api}. Tiene las mínimas dependencias necesarias y está distribuido bajo la licencia BSD, con el objetivo de incentivar su uso tanto en ambientes educativos como comerciales.

Scikit-learn expone una gran variedad de algoritmos de aprendizaje utilizando una interfaz consistente y orientada a la resolución de tareas, lo que permite una comparación sencilla entre distintos métodos de aprendizaje para una misma aplicación. Al depender del ecosistema científico de Python, puede ser integrado con facilidad en aplicaciones que se salgan del rango tradicional del análisis estadístico de datos. Además, los algoritmos, que han sido implementados en un lenguaje de alto nivel, pueden ser utilizados como bloques de construcción para desarrollar estrategias más complejas que se adecuen a cada caso particular.

\subsection{Microsoft Azure}
\label{subsec:azure}

Microsoft Azure es una plataforma de servicios en la nube que ofrece una amplia gama de herramientas y recursos para el despliegue y gestión de aplicaciones y servicios. Entre sus múltiples servicios, Azure permite a los usuarios crear y gestionar máquinas virtuales (VMs) con diversas configuraciones de recursos, adaptándose a las necesidades específicas de cada proyecto. Esta capacidad es especialmente valiosa en el campo del aprendizaje automático, donde las tareas de entrenamiento y prueba de modelos requieren recursos computacionales significativos y variados. Con Azure, los desarrolladores e investigadores pueden seleccionar configuraciones específicas de CPU, GPU, memoria y almacenamiento para optimizar el rendimiento y costo de sus experimentos de aprendizaje automático. Microsoft Azure ofrece un programa especial para estudiantes llamado "Azure for Students", que proporciona acceso gratuito a una variedad de servicios en la nube. Este programa incluye un crédito inicial de \$100 USD para usar en cualquier servicio de Azure durante 12 meses, sin necesidad de una tarjeta de crédito para registrarse. 

\subsection{Matplotlib}
\label{subsec:matplotlib}

Matplotlib es una de las librerías más destacadas y utilizadas en el ecosistema de Python para la creación de gráficos y visualizaciones de datos. Esta herramienta permite a los desarrolladores y científicos de datos generar una amplia variedad de gráficos estáticos, animados e interactivos con relativa facilidad. Matplotlib se caracteriza por su flexibilidad y capacidad para crear visualizaciones de alta calidad y personalizables, que van desde simples gráficos de líneas y barras hasta complejas visualizaciones tridimensionales y de mapas de calor. Además, su integración con otras librerías populares de Python, como NumPy y pandas, facilita el proceso de análisis y visualización de datos, convirtiéndola en una opción preferida en ámbitos académicos y profesionales.

El diseño de Matplotlib sigue una filosofía similar a la de MATLAB, lo que hace que sea especialmente accesible para usuarios familiarizados con ese entorno. Sin embargo, a diferencia de MATLAB, Matplotlib es de código abierto y gratuito. Las capacidades avanzadas de esta librería incluyen la personalización detallada de todos los elementos del gráfico, la posibilidad de exportar gráficos en múltiples formatos (como PNG, PDF y SVG), y la creación de gráficos interactivos utilizando bibliotecas adicionales como mpld3 y Plotly. 


%%%%
%% ALGORITHMS, CURRENTLY
%%%%

\section{Modelos utilizados}
\label{sec:models}

\subsection{Modelos lineales: regresión logística}
\label{subsec:model-linear}

Los modelos lineales son una clase fundamental de técnicas estadísticas y de aprendizaje automático que se utilizan para predecir una variable objetivo a partir de una o más variables independientes. La característica principal de estos modelos es la relación lineal entre las variables independientes y la variable objetivo. En su forma más simple, el modelo lineal se representa mediante la ecuación $y = \beta_0 + \beta_1 x_1 + \beta_2 x_2 + \dots + \beta_n x_n$, donde $y$ es la variable dependiente, $x_1, x_2, \dots, x_n$ son las variables independientes, y $\beta_1, \beta_2, \dots, \beta_n$ son los coeficientes que representan el impacto de cada variable independiente sobre la variable dependiente.

Dentro de los modelos lineales, la regresión logística es especialmente relevante para problemas de clasificación. A diferencia de la regresión lineal, que se utiliza para problemas de predicción continua, la regresión logística se emplea cuando la variable objetivo es categórica. En su forma binaria más simple, la regresión logística predice la probabilidad de que una observación pertenezca a una de dos posibles categorías. La función logística o sigmoide, 
\begin{equation*}
    P(y=1\mid x) = \frac{1}{1+e^{-\beta_0 + \beta_1 x_1 + \beta_2 x_2 + \dots + \beta_n x_n}},
\end{equation*}
transforma cualquier valor real de la combinación lineal de las variables independientes en un valor entre 0 y 1, lo que se interpreta como una probabilidad. Los parámetros desconocidos $\beta_i$ son estimados habitualmente a través del método de máxima verosimilitud.

El uso de la regresión logística en problemas de clasificación es ventajoso por varias razones. Primero, es un modelo interpretativo, ya que los coeficientes obtenidos pueden proporcionar una idea clara de cómo cada variable independiente afecta la probabilidad de pertenecer a una categoría específica. Además, la regresión logística es relativamente fácil de implementar y eficiente computacionalmente, lo que la hace adecuada para grandes conjuntos de datos. Por último, aunque su capacidad para capturar relaciones complejas es limitada en comparación con modelos no lineales más avanzados, su simplicidad y efectividad la convierten en una opción sólida para establecer una línea base en proyectos de clasificación, permitiendo comparaciones posteriores con modelos más complejos.

La librería scikit-learn proporciona una interfaz para construir y evaluar modelos de regresión logística a través de su clase \texttt{LogisticRegression} \cite{sk-logistic-regression}. Esta clase permite a los usuarios configurar parámetros como la regularización, la penalización y el algoritmo de optimización. Esta implementación admite dos tipos de penalización: \texttt{L1} (Lasso) y \texttt{L2} (Ridge) que ayudan a prevenir el sobreajuste. Además, permite ajustar la fuerza de la regularización a través de un parámetro \texttt{C}. Esta flexibilidad facilita la adaptación del modelo a diferentes problemas y conjuntos de datos. La personalización aumenta al elegir un algoritmo de optimización mediante el parámetro \texttt{solver}, que admite varias opciones que pueden adaptarse al tamaño o el tipo de atributos del conjunto de datos a utilizar. Adicionalmente, \texttt{LogisticRegression} es capaz de manejar problemas de clasificación binaria y multiclase, con diferentes estrategias de acuerdo al algoritmo de resolución escogido.


\subsection{Árboles de decisión: bosque aleatorio}
\label{subsec:model-random-forest}

Otra técnica popular de aprendizaje automático utilizada para problemas tanto de clasificación como de regresión son los árboles de decisión \cite{sk-decision-trees}. Su estructura jerárquica consiste en nodos de decisión que dividen iterativamente el conjunto de datos en subconjuntos más homogéneos, facilitando así la predicción de la variable objetivo. Cada nodo del árbol representa una prueba sobre un atributo específico, y cada rama denota el resultado de la prueba. Los árboles de decisión son fáciles de interpretar y visualizar, lo que los hace muy útiles para entender las decisiones del modelo y comunicar los resultados a personas no técnicas.

Sin embargo, los árboles de decisión tienen algunas limitaciones, como su tendencia a sobreajustarse a los datos de entrenamiento, lo que puede llevar a un desempeño deficiente en datos nuevos. Para abordar estas limitaciones, se emplean técnicas de ensamblado como los bosques aleatorios (\emph{Random Forests}). Un bosque aleatorio es un conjunto de árboles de decisión entrenados sobre diferentes subconjuntos del conjunto de datos y con diferentes características seleccionadas aleatoriamente. La predicción final se obtiene mediante el promedio de las predicciones individuales de los árboles (en el caso de regresión) o mediante un voto mayoritario (en el caso de clasificación). Esta metodología mejora significativamente la precisión del modelo y reduce el riesgo de sobreajuste.

El uso de bosques aleatorios en problemas de clasificación ofrece varias ventajas. Primero, al combinar múltiples árboles, el modelo se vuelve más robusto y generalizable, mitigando la influencia de outliers y ruido en los datos. Además, los bosques aleatorios proporcionan una medida de importancia de las características, lo que permite identificar las variables más relevantes para la predicción. Esta capacidad es especialmente útil en contextos académicos y profesionales donde la interpretación del modelo y la identificación de factores clave son cruciales para la toma de decisiones informadas.

La implementación de un modelo de bosque aleatorio en la librería scikit-learn se ofrece a través de la clase \texttt{RandomForestClassifier} \cite{sk-random-forest}, que permite ajustar diversos parámetros como el número de árboles (\texttt{n\_estimators}), la profundidad máxima de los árboles (\texttt{max\_depth}) y el número mínimo de muestras por hoja (\texttt{min\_samples\_leaf}). 

\subsection{Máquinas de vector soporte (SVM)}
\label{subsec:model-svm}

En las máquinas de vector soporte (\emph{Support Vector Machines}, SVM) \cite{sk-svm-theory} el principal objetivo es encontrar el hiperplano óptimo que maximiza el margen entre las diferentes clases en un espacio de características. Este margen es la distancia más amplia posible entre el hiperplano y los puntos de datos más cercanos de cualquier clase, conocidos como vectores soporte. La SVM es particularmente eficaz en espacios de alta dimensionalidad y es robusta frente al sobreajuste, lo que la hace adecuada para conjuntos de datos complejos.

Una de las ventajas clave de las SVM es su capacidad para manejar problemas no lineales mediante el uso de funciones kernel. Estos núcleos transforman los datos en un espacio de mayor dimensión donde un hiperplano lineal puede separarlos. Los tipos de kernel más comunes incluyen el lineal, polinómico, radial (RBF), y sigmoide. Esta flexibilidad permite a las SVM abordar una amplia gama de problemas de clasificación, incluso cuando las relaciones entre las características y las etiquetas son altamente no lineales. Sin embargo, para que funcione de forma adecuada es necesario seleccionar el kernel apropiado y ajustar sus parámetros, por ejemplo, mediante validación cruzada.

En el contexto de problemas de clasificación, las SVM son especialmente útiles debido a su alta precisión y su capacidad para manejar tanto clases linealmente separables como no separables. En escenarios de clasificación binaria, las SVM son capaces de encontrar la frontera de decisión que maximiza la separación entre las dos clases. Para problemas de clasificación multiclase, se pueden aplicar estrategias para descomponer el problema en múltiples problemas de clasificación binaria. Aunque las SVM pueden ser computacionalmente intensivas, especialmente con grandes conjuntos de datos, tienen una gran eficacia en términos de rendimiento y de capacidad para generalizar bien a datos no vistos.

La clase \texttt{SVC} \cite{sk-svm} de scikit-learn permite ajustar parámetros clave como el tipo de kernel, el parámetro de regularización (\texttt{C}) y coeficientes específicos del kernel elegido.

\subsection{Vecinos más cercanos (k-NN)}
\label{subsec:model-neighbors}

El método de vecinos más cercanos (k-Nearest Neighbors, k-NN) es una técnica de aprendizaje supervisado utilizada para resolver problemas tanto de clasificación como de regresión. Este método se basa en la premisa de que objetos similares generalmente se encuentran cerca en el espacio de características. En problemas de clasificación, el algoritmo k-NN asigna una etiqueta a una nueva instancia basándose en las etiquetas de sus k vecinos más cercanos en el conjunto de datos de entrenamiento. La cercanía o similitud entre las instancias se mide generalmente utilizando distancias métricas, como la distancia euclidiana, Manhattan, o Minkowski.

Una de las principales ventajas del k-NN es su simplicidad y facilidad de implementación. A diferencia de otros algoritmos que requieren un proceso de entrenamiento explícito, k-NN es un algoritmo perezoso que almacena todas las instancias del conjunto de entrenamiento y realiza el cálculo de las distancias en el momento de la predicción. Esta característica lo convierte en un método intuitivo y fácil de entender, aunque también implica que puede ser computacionalmente costoso, especialmente con grandes conjuntos de datos. Además, la elección del número de vecinos (k) y la métrica de distancia son cruciales para el rendimiento del modelo. Valores de k demasiado bajos pueden llevar a un sobreajuste, mientras que valores demasiado altos pueden conducir a un subajuste.

En el contexto de problemas de clasificación, el k-NN es especialmente útil en situaciones donde las fronteras de decisión entre clases no son lineales y pueden ser complejas. Debido a su naturaleza basada en instancias, el k-NN puede capturar patrones locales en los datos y adaptarse a cambios en la distribución de las clases. Sin embargo, el rendimiento de k-NN puede verse afectado por la presencia de ruido y características irrelevantes, lo que subraya la importancia de la preprocesamiento de datos, como la normalización y la selección de características. A pesar de estas limitaciones, el k-NN sigue siendo una herramienta valiosa para la clasificación debido a su simplicidad y flexibilidad.

La clase \texttt{KNeighborsClassifier} \cite{sk-knn} de la librería scikit-learn proporciona la interfaz para configurar y utilizar el algoritmo k-NN. Esta clase ofrece la posibilidad de especificar el número de vecinos a considerar mediante el parámetro \texttt{n\_neighbors} y de seleccionar la métrica de distancia adecuada utilizando el parámetro \texttt{metric}, que incluye opciones como la distancia euclidiana y Manhattan. Además, scikit-learn permite ajustar parámetros adicionales como el peso de los vecinos, que puede ser uniforme o basado en la distancia.

\subsection{Naive Bayes (Naive Bayes gaussiano)}
\label{subsec:model-naive-bayes}

Los métodos Naive Bayes \cite{sk-naive-bayes} son un conjunto de algoritmos de aprendizaje supervisado basados en la aplicación del teorema de Bayes con la suposición "ingenua" (en inglés, \emph{naive}) de independencia condicional entre cada par de características dado el valor de la variable de clase. El teorema de Bayes postula la siguiente relación, dada la variable de clase $y$ y los vectores característica dependientes $x_{1}$ a $x_{n}$:
\begin{equation*}
    P(y \mid x_{1},\dots,x_{n}) = \dfrac{P(y)P(x_{1},\dots,x_{n}\mid y)}{P(x_{1},\dots,x_{n})}
\end{equation*}

Usando la suposición ingenua de independencia condicional de forma que
\begin{equation*}
    P(x_{i} \mid y,x_{1},\dots,x_{i-1},x_{i+1},\dots,x_{n})=P(x_{i}\mid y),
\end{equation*}

para todo $i$, esta relación se simplifica a 
\begin{equation*}
    P(y\mid x_{1},\dots,x_{n})=\dfrac{P(y) \prod^{n}_{i=1} P(x_{i}\mid y)}{P(x_{1},\dots,x_{n})}
\end{equation*}

Como $P(x_{1},\dots,x_{n})$ es constante dada la entrada, se puede utilizar la siguiente regla de clasificación:
\begin{equation*}
    P(y\mid x_{1},\dots,x_{n}) \propto P(y) \prod^{n}_{i=1}P(x_{i}\mid y) \Rightarrow 
    \hat{y} = \arg \max_{y} P(y) \prod^{n}_{i=1}P(x_{i}\mid y),
\end{equation*}

y se puede usar la estimación de máximo a posteriori (MAP) para estimar $P(y)$ y $P(x_{i}\mid y)$; y la primera expresión es entonces la frecuencia relativa de la clase $y$ en el conjunto de entrenamiento.

Los diferentes clasificadores Naive-Bayes difieren sobre todo en la suposición que realicen sobre la distribución de $P(x_{i} \mid y)$. El algoritmo Naive-Bayes gaussiano es una variante específica de Naive-Bayes que se utiliza cuando las características son continuas y se asumen distribuidas según una distribución normal (gaussiana), lo que es común en muchas aplicaciones reales. En este caso, el clasificador estima los parámetros de la distribución (media y varianza) para cada característica en cada clase utilizando los datos de entrenamiento. Durante la clasificación de una nueva instancia, el algoritmo calcula la probabilidad de que esta instancia pertenezca a cada clase basándose en las distribuciones gaussianas estimadas y asigna la clase con la probabilidad a posteriori más alta.

A pesar de sus suposiciones aparentemente simplificadas, los clasificadores Naive-Bayes obtienen buenos resultados en muchos problemas del mundo real, especialmente en aplicaciones como la clasificación de documentos y filtros de spam; y requieren pequeñas cantidades de datos de entrenamiento para estimar los parámetros necesarios. Estos clasificadores pueden llegar a ser extremadamente rápidos comparados a otros métodos más sofisticados. La separación de las características condicionales de clase significa que cada distribución puede ser estimada independientemente como una distribución unidimensional. Esto a su vez ayuda a aliviar problemas provocados por la dimensionalidad de las distribuciones.

La implementación de este algoritmo en la librería scikit-learn se encuentra en la clase \texttt{GaussianNB} \cite{sk-nb-gaussian}.

\subsection{Métodos de ensamblaje (ensemble): máquinas de potenciación de gradiente}
\label{subsec:model-gradient}

El método de ensamblaje por máquinas de potenciación de gradiente (Gradient Boosting Machines, GBM) es una técnica de aprendizaje automático que combina múltiples modelos débiles para crear un modelo más fuerte y preciso. La idea central de GBM es construir el modelo de forma secuencial, donde cada modelo sucesivo intenta corregir los errores del modelo anterior. En el contexto de problemas de clasificación, cada nuevo árbol de decisión se ajusta a los residuos de los árboles anteriores, utilizando el gradiente de la función de pérdida como guía. Esta estrategia permite que GBM capture relaciones complejas y no lineales en los datos, mejorando considerablemente la precisión del modelo.

Uno de los mayores beneficios de usar máquinas de potenciación de gradiente para problemas de clasificación es su capacidad para manejar datos heterogéneos y características con diferentes escalas y distribuciones. GBM puede ajustarse a los datos de manera muy detallada, lo que lo hace adecuado para problemas donde las relaciones entre las variables no son lineales ni simples. Sin embargo, este alto nivel de ajuste también puede llevar a un sobreajuste si no se controla adecuadamente. Por ello, es crucial utilizar técnicas como la validación cruzada y ajustar parámetros como el número de árboles, la tasa de aprendizaje, y la profundidad máxima de los árboles para encontrar el equilibrio adecuado entre sesgo y varianza.

En problemas de clasificación, GBM es particularmente eficaz debido a su capacidad para manejar clases no balanceadas y proporcionar probabilidades de clasificación en lugar de simplemente etiquetas de clase. Esto es especialmente útil en aplicaciones como la detección de fraudes, diagnóstico médico y otros escenarios donde la probabilidad de pertenecer a una clase particular puede ser más informativa que la clasificación binaria. Además, las implementaciones modernas de GBM, como XGBoost, LightGBM y CatBoost, han optimizado significativamente la velocidad y eficiencia del algoritmo, permitiendo su aplicación en grandes conjuntos de datos y en tiempo real.

La implementación de máquinas de potenciación de gradiente en la librería scikit-learn es accesible a través de la clase GradientBoostingClassifier \cite{sk-gradient-boost} permite a los usuarios ajustar una variedad de hiperparámetros para optimizar el rendimiento del modelo. Los usuarios pueden especificar el número de árboles, la tasa de aprendizaje, y otros parámetros clave para controlar la complejidad y la capacidad de generalización del modelo.

\subsection{Redes neuronales (Deep Neural Networks, DNN) Multi-layer Perceptron}
\label{subsec:model-neural}

Las redes neuronales están compuestas por capas de neuronas artificiales que imitan el comportamiento de las neuronas en el cerebro humano. Cada neurona realiza una operación matemática simple sobre la entrada y pasa la salida a la siguiente capa. A través de múltiples capas, la red es capaz de capturar patrones y características complejas en los datos. Estás propiedades pueden aplicarse a situaciones de aprendizaje supervisado mediante el uso de algoritmos como el perceptrón multicapa (Multilayer Perceptron, MLP).

El algoritmo del perceptrón multicapa es una clase de red neuronal supervisada que consiste en una capa de entrada, una o más capas ocultas y una capa de salida. Cada capa está totalmente conectada a la siguiente, y cada conexión tiene un peso ajustable que se optimiza durante el proceso de entrenamiento. En problemas de clasificación, el MLP es especialmente útil debido a su capacidad para aprender representaciones no lineales complejas de los datos. El algoritmo utiliza funciones de activación no lineales, como la función sigmoide o la ReLU (Rectified Linear Unit), que permiten que la red capture y modele relaciones no lineales en los datos. El entrenamiento de un MLP se realiza mediante un proceso llamado retropropagación del error (backpropagation), que ajusta iterativamente los pesos para minimizar una función de pérdida, típicamente la entropía cruzada en problemas de clasificación.

El uso del MLP ofrece varias ventajas. Primero, debido a su estructura de múltiples capas, el MLP puede aprender y generalizar bien a partir de datos complejos y de alta dimensionalidad. Esto es particularmente útil en dominios como el reconocimiento de imágenes, la detección de voz y la clasificación de texto, donde las relaciones entre las características no son triviales. Además, las redes neuronales pueden ser adaptadas a problemas multiclase con facilidad, utilizando técnicas como la salida softmax en la capa final para obtener probabilidades de clase. Sin embargo, el MLP también requiere una cantidad considerable de datos y poder computacional para entrenar eficazmente, y la selección de la arquitectura y los hiperparámetros adecuados puede ser un proceso desafiante que a menudo implica ensayo y error y validación cruzada.

La librería scikit-learn implementa la clase MLPClassifier \cite{sk-multilayer-perceptron} para el algoritmo del perceptrón multicapa. Los usuarios pueden definir parámetros como el número de capas ocultas y el número de neuronas por capa (hidden\_layer\_sizes), la función de activación (activation), y el algoritmo de optimización (solver). Scikit-learn también permite ajustar la tasa de aprendizaje (learning\_rate) y utilizar técnicas de regularización para prevenir el sobreajuste.
\section{Datos utilizados}
\label{sec:datasets}

\todo[inline]{Add info on previous results ??}
\todo[inline]{Show class spread ??}

Todos los conjuntos de datos utilizados en los análisis realizados están disponibles de forma libre en la web y proceden de dos fuentes: el repositorio de aprendizaje automático de la Universidad de California Irvine y la plataforma OpenML.

El repositorio de aprendizaje automático de la UCI \cite{ml-uci} es una colección de bases de datos, teorías de dominios y generadores de datos que son utilizados por la comunidad de aprendizaje automático para el análisis empírico de algoritmos de aprendizaje automático. El archivo fue creado en 1987 como un servidor FTP por David Aha y otros compañeros estudiantes en la UCI. Desde entonces ha sido ampliamente utilizado por estudiantes, educadores e investigadores de todo el mundo como una fuente primaria de conjuntos de datos para aprendizaje automático.

OpenML \cite{openml} es una plataforma abierta para compartir conjuntos de datos, algoritmos y experimentos, que tiene como objetivo lograr que las investigaciones sobre aprendizaje automático sean más fácilmente accesibles y reutilizables. Esta plataforma contiene un repositorio con más de cinco mil conjuntos de datos y resultados de experimentos que otros usuarios han realizado con ellos. Además, ofrece librerías propias para integrar la recuperación de sus datos directamente con el código de desarrollo de modelos de aprendizaje. Scikit-Learn ofrece también una función para descargar los datos fácilmente de esta plataforma, \texttt{fetch\_openml} \cite{sk-fetch-openml}, que toma el nombre de un conjunto de datos y devuelve un dataframe de Pandas con toda la información disponible.

A continuación se describen las características más importantes de los conjuntos de datos empleados.
% ANNEX All attributes

\subsection{Iris}

Iris esta entre las primeras bases de datos recogidas, y es una de las más conocidas y utilizadas en la literatura sobre reconocimiento de patrones. Los datos originales fueron publicados por R.A. Fisher en 1936, y la versión utilizada en este proyecto procede de la UCI (1988) \cite{iris-dataset}.
El conjunto contiene tres clases distintas con 50 ejemplos cada una, donde cada clase se refiere a una especie del género de plantas Iris. Cada ejemplo contiene cuatro atributos que describen la longitud y la anchura del sépalo y el pétalo de la planta en centímetros. Las tres opciones de clasificación son Iris Setosa, Iris Versicolour e Iris Virginica, donde una de las clases es linealmente separable de las otras dos, y estas últimas no son separable entre sí de forma lineal.

La versión utilizada corresponde a la implementación \texttt{load\_iris} de Scikit-Learn.

\subsection{Ionosfera}

Se trata de un conjunto de datos con 351 muestras procedente del repositorio de la UCI \cite{ionosphere-dataset}. Contiene datos de radar obtenidos por el grupo de física espacial de la Universidad John Hopkins y donados por Vince Sigillito en 1989. El sistema radar está ubicado en Goose Bay, Labrador y consiste en un array de 16 antenas de alta frecuencia. El objetivo es la medición de electrones libres en la ionosfera y su clasificación binaria entre "buenas" respuestas del radar que indican evidencia de algún tipo de estructura en la ionosfera y "malas" respuestas en las que las señales simplemente pasan a través de la ionosfera. Las señales recibidas se procesaron utilizando una función de autocorrelación con el tiempo de pulso y el número de pulso como argumentos y cada una de las muestras del conjunto de datos está descrita por dos atributos continuos para cada uno de los 17 números de pulso, correspondientes al valor complejo obtenido de la señal electromagnética compleja. Hay por lo tanto un total de 34 características continuas por muestra.

\subsection{Autenticación de billetes}

Este conjunto de datos contiene información extraída de 1372 imágenes tomadas para evaluar un procedimiento de autenticación de billetes. Fue donado en Agosto de 2012 por Volker Lohweg de la Universidad de Ciencias Aplicadas de Ostwestfalen-Lippe, Alemania, al repositorio de la UCI \cite{banknote-dataset}. Para la digitalización de las imágenes tomadas se empleó una cámara industrial normalmente utilizada para la inspección de impresiones. Las imágenes finales tienen un tamaño de 400x400 píxeles con una resolución de alrededor de 660 dpi en escala de grises. Posteriormente, se empleó una herramienta de transformada ondícula para extraer cuatro características continuas de las imágenes. A partir de estas cuatro características, el objetivo es clasificar los billetes en dos clases, representadas en la tabla de datos con un cero para billetes falsos y un uno para billetes auténticos.

\subsection{Fonemas}

Este conjunto de datos tiene como objetivo diferenciar entre sonidos nasales (clase 0) y sonidos orales (clase 1) mediante el análisis de cinco atributos que representan las amplitudes de los cinco primeros armónicos normalizados con la energía total. El formato actual de los datos corresponde al repositorio KEEL \cite{keel-repo}, aunque originalmente los datos fueron alojados por el proyecto ELENA y obtenidos para el proyecto europeo ESPRIT 5516, ROARS, cuyo objetivo era desarrollar sistemas de reconocimiento del habla para español y francés.

La base de datos consiste en vocales de 1809 sílabas aisladas, con tres puntos de observación por vocal, lo que resulta en 5427 instancias iniciales, que posteriormente se reducen a 5404 tras eliminar aquellas muestras con una amplitud armónica nula. Las observaciones incluyen el momento de máxima energía total y otro dos momentos 8 ms antes y después de este pico. Esta estrategia permite distinguir vocales nasales y orales examinando sus características armónicas en el tiempo. Cada observación contiene cinco atributos correspondientes a los armónicos de cinco fonemas: "sh" (como en inglés en \emph{she}), "dcl" (como en \emph{dark}), "iy" (como la vocal en \emph{she}), "aa" (como la vocal en \emph{dark}), y "ao" (como la primera vocal en \emph{water}).

\subsection{Electroencefalograma}

Los datos proceden de una medición continua de un electroencefalograma tomada con el dispositivo neuronal \emph{Emotiv EEG} \cite{eeg-eye-dataset}. La medición tuvo una duración total de 117 segundos,en los que se obtuvieron 14980 muestras individuales. Adicionalmente, el estado del ojo fue detectado mediante una cámara durante la medición del EEG y añadido posteriormente a los datos mediante el análisis de los fotogramas del vídeo. Para cada muestra, se identifico de forma manual el estado de ojo cerrado, al que se asignó un valor de "1", y de ojo abierto, con un valor "0". La base de datos está ordenada de forma cronológica con el primer valor medido al principio de los datos. Los atributos de cada instante de tiempo corresponden a las 14 medidas EEG del dispositivo, originalmente etiquetadas AF3, F7, F3, FC5, T7, P, O1, O2, P8, T8, FC6, F4, F8, AF4, en ese orden.

\subsection{Electricidad}
\label{subsec:dataset-electricity}

Este conjunto de datos fue descrito por M. Harries \cite{electricity-dataset} en 1999 y ha sido ampliamente utilizado para aprendizaje automático desde entonces. Los datos fueron obtenidos del mercado eléctrico australiano de Nuevo Gales del Sur. En este mercado, los precios no son fijos, sino que cambian según la demanda y el abastecimiento de electricidad cada cinco minutos. Además, se tienen en cuenta las transferencias de electricidad para aliviar fluctuaciones procedentes del estado contiguo de Virginia.

El conjunto de datos (denominado originalmente ELEC2) contiene 45.312 muestras fechadas entre el 7 de mayo de 1996 y el 5 de diciembre de 1998. Cada ejemplo del conjunto hace referencia a un periodo de 30 minutos, lo que resulta en 48 muestras individuales para un periodo de un día. Hay ocho atributos para cada muestra: la fecha, el día de la semana, la hora, la demanda eléctrica y el precio en Nuevo Gales del Sur, la demanda eléctrica y el precio en Victoria y la cantidad planeada de electricidad a transferir entre los dos estados. Estos atributos están normalizados entre 0 y 1. La clasificación es binaria entre dos categorías: el cambio a la alza (UP) o a la baja (DOWN) del precio de la electricidad en Nuevo Gales del Sur relativo a una media móvil de las últimas 24 horas (para eliminar el impacto de las tendencias de precio a más largo plazo). 

% \subsection{Tipo de cubierta}
% \subsection{Mano de póquer}


%\cleardoublepage