\chapter{Conclusiones y trabajos futuros}
\label{chap:conclusiones}

Este capítulo resume las conclusiones principales obtenidas durante el proyecto y los obstáculos más importantes que surgieron durante el desarrollo. A continuación, se examinarán los objetivos iniciales para comprobar si han sido conseguidos, se analizarán los conceptos adquiridos durante el proyecto y se discutirán posibles trabajos posteriores para profundizar en este tema.

El logro más importante de este proyecto ha sido el desarrollo de una herramienta que permite medir tanto el rendimiento de un modelo de aprendizaje como sus emisiones de carbono y consumo eléctrico, de una forma altamente personalizable para distintos conjuntos de datos y modelos representativos.

Los resultados obtenidos al utilizar esta herramienta para medir el impacto ambiental de varios modelos representativos muestran la importancia de considerar las emisiones producidas como una métrica más en el proceso de elección de modelo de aprendizaje para un problema concreto. Esto se debe a que muchos modelos ofrecen resultados de precisión muy similares, pero al analizar las emisiones se observan claras diferencias en las tendencias de consumo energético. Por lo tanto, es factible reducir las emisiones sin comprometer el rendimiento de la tarea de aprendizaje.

Durante el desarrollo de la aplicación, los principales obstáculos encontrados han estado relacionados con el preprocesamiento de los datos. Para conseguir que la herramienta pudiera comparar las emisiones producidas en una gran variedad de conjuntos de datos, se debieron encontrar los métodos más generales para limpiar y homogeneizar los datos, que consiguieran que todos los modelos pudieran obtener resultados de forma comparable. Al aplicar el mismo proceso de preparación para todos los conjuntos de datos, es imposible obtener el mismo rendimiento que se obtendría con un preprocesamiento específico. Sin embargo, el objetivo buscado ha sido sólo que fuera suficientemente cercano para todos los modelos por igual.


\section{Consecución de objetivos}
\label{sec:consecucion-objetivos}

El objetivo principal de este proyecto era obtener una comparación del consumo y el impacto medioambiental de diferentes técnicas de clasificación. Este objetivo ha sido conseguido mediante la creación de una herramienta que usa técnicas de medición modernas para calcular las emisiones producidas durante el aprendizaje. La aplicación de esta herramienta a diferentes modelos representativos permite obtener una comparación de su rendimiento en múltiples situaciones.

Respecto a los objetivos específicos, la sección~\ref{sec:lit-rev} estudió distintas herramientas ya existentes usadas en el campo de la medición de emisiones del aprendizaje automático. De estas, se eligió scikit-learn como motor de aprendizaje y CodeCarbon como base para las mediciones del impacto ambiental, debido a su facilidad de desarrollo e integración con otras librerías. Durante el desarrollo de la aplicación, ambas herramientas han sido ampliamente estudiadas para asegurar su buen funcionamiento en las pruebas llevadas a cabo.

Estas pruebas tenían como objetivo comparar el consumo energético de los algoritmos más importantes y analizar la relación entre la precisión y el consumo. El experimento de la sección~\ref{sec:test-1-models} recoge varias de estas pruebas y extrae conclusiones sobre que modelos pueden ser más interesantes en el contexto de la limitación de las emisiones de carbono producidas.

\section{Aplicación de lo aprendido}
\label{sec:aplicacion}

En el desarrollo de este proyecto, fue necesario emplear gran parte del conocimiento adquirido durante el Grado en Ingeniería en Tecnologías de la Telecomunicación completado. Las siguientes asignaturas han sido especialmente relevantes, al proporcionar la experiencia necesaria para completar este proyecto:
\begin{itemize}
    \item \textbf{Fundamentos de la programación y de la informática.} Esta asignatura ofrece una primera introducción a la informática y la programación en el Grado. Además, provee el conocimiento fundacional de como escribir buen código, necesario para desarrollar un proyecto de estas características.
    \item \textbf{Procesamiento digital de la información.} Esta asignatura supone una introducción a los procesos de clasificación y estimación en el aprendizaje automático y estudia, incluyendo fundamentos teóricos de teoría de la decisión y diseño de clasificadores como vecinos más cercanos y regresión logística. Estos fundamentos teóricos construyen la base de los modelos de aprendizaje comparados en este proyecto.
    \item \textbf{Introduction to Artificial Intelligence} (Erasmus - NTNU). Esta asignatura proporciona una introducción a los conceptos y técnicas de la inteligencia artificial, utilizando Python como lenguaje de programación. Abarca desde algoritmos básicos hasta aplicaciones prácticas en aprendizaje automático y redes neuronales, sentando las bases necesarias para el desarrollo y la implementación de sistemas de inteligencia artificial en proyectos complejos.
    \item \textbf{Ingeniería de sistemas de la información.} Esta asignatura ofrece herramientas importantes para gestionar el control de versiones de cualquier proyecto de software y, especificamente, el uso de la plataforma GitHub para albergar el código del proyecto y mantener un registro de las modificaciones efectuadas.
    \item \textbf{Servicios y aplicaciones telemáticas.} La principal aportación de esta asignatura para este proyecto es la experiencia adicional en programación en Python, especialmente su uso en proyectos complejos de mayor tamaño. Además, introduce la utilización de la herramienta \texttt{pip} para la administración de paquetes, utilizada en el desarrollo de este proyecto.
\end{itemize}

\section{Lecciones aprendidas}
\label{sec:lecciones_aprendidas}

Durante el desarrollo de este proyecto, varios desafíos han requerido expandir el conocimiento mencionado en la sección anterior y adquirir nuevas competencias para encontrar soluciones a los problemas que surgían. Estas son algunas de las principales habilidades desarrolladas:

\begin{itemize}
    \item Programación con Python: como organizar proyectos de gran tamaño con múltiples submódulos y crear paquetes personalizados. Adicionalmente, experiencia en varias librarías muy extendidas:
    \begin{itemize}
        \item Gestión de vectores y matrices con la librería Pandas
        \item Gestión de argumentos de línea de comandos con la librería Click.
        \item Creación de gráficas personalizables para diferentes motores gráficos con la librería Matplotlib.
    \end{itemize}
    \item Conocimiento de modelos de aprendizaje, ampliando el conocimiento teórico a la aplicación de los modelos a diferentes conjuntos de datos.
    \item Uso de scikit-learn como motor de aprendizaje automático, abstrayendo la implementación teórica de los algoritmos a analizar.
    \item Métodos de cálculo de emisiones, incluyendo el método utilizado por CodeCarbon en función de la energía consumida y la mezcla energética de la red eléctrica de cada país.
\end{itemize}


\section{Trabajos futuros}
\label{sec:trabajos_futuros}

El trabajo de este proyecto está enfocado en presentar pruebas básicas del funcionamiento de la aplicación comparativa desarrollada. Por lo tanto, existen muchas avenidas adicionales de estudio para ampliar este análisis. 

\subsubsection{Ampliación a modelos personalizados}
Una posible línea de investigación futura es la ampliación del número y la diversidad de los modelos utilizados, incluyendo la posibilidad de trabajar con algoritmos adicionales además de los incluidos en la librería scikit-learn. De este modo, se podría comparar el consumo energético en contextos más específicos. Además, probar una gama más amplia de modelos de aprendizaje automático, incluidos aquellos basados en arquitecturas modernas como transformadores y redes neuronales profundas, podría ofrecer una visión más completa del consumo energético y el rendimiento.

\subsubsection{Optimización Energética}
Otro enfoque relevante sería investigar técnicas de optimización energética específicamente diseñadas para el entrenamiento de modelos de aprendizaje automático. Esto podría incluir el desarrollo de algoritmos de ajuste de hiperparámetros que consideren no solo el rendimiento del modelo, sino también su consumo energético. Asimismo, explorar técnicas de compresión de modelos y aprendizaje federado (aprendizaje colaborativo a través de una arquitectura descentralizada en múltiples dispositivos) podría reducir significativamente las emisiones asociadas al entrenamiento.

\subsubsection{Paralelismo y Distribución de Tareas}
El estudio de métodos más avanzados de paralelismo y distribución de tareas puede ser beneficioso. Investigaciones adicionales podrían centrarse en la eficiencia energética de diferentes enfoques de paralelización, como el uso de GPU y TPU, y en cómo la asignación dinámica de recursos en entornos distribuidos afecta el consumo energético. Además, la evaluación del impacto de técnicas de balanceo de carga y gestión de colas de tareas en sistemas de computación en la nube puede ofrecer perspectivas valiosas para optimizar el uso de recursos.

\subsubsection{Medición en Entornos Reales}
Realizar mediciones en entornos de producción reales podría proporcionar datos más precisos y aplicables sobre el consumo energético de los modelos. Esto incluiría la integración de herramientas de monitoreo de energía en sistemas de despliegue y la evaluación del impacto de diversas cargas de trabajo y patrones de uso en el consumo energético.
