\chapter{Estado del arte}               %% a.k.a "Tecnologías utilizadas"
\label{chap:tecnologias}

\begin{comment}
Descripción de las tecnologías que utilizas en tu trabajo. 
Con dos o tres párrafos por cada tecnología, vale. 
Se supone que aquí viene todo lo que no has hecho tú.

Puedes citar libros, como el de Bonabeau et al., sobre procesos estigmérgicos~\cite{bonabeau:_swarm}. 
Me encantan los procesos estigmérgicos.
Deberías leer más sobre ellos.
Pero quizás no ahora, que tenemos que terminar la memoria para sacarnos por fin el título.
Nota que el \~ \ añade un espacio en blanco, pero no deja que exista un salto de línea. 
Imprescindible ponerlo para las citas.

Citar es importantísimo en textos científico-técnicos. 
Porque no partimos de cero.
Es más, partir de cero es de tontos; lo suyo es aprovecharse de lo ya existente para construir encima y hacer cosas más sofisticadas.
¿Dónde puedo encontrar textos científicos que referenciar?
Un buen sitio es Google Scholar\footnote{\url{http://scholar.google.com}}.
Por ejemplo, si buscas por ``stigmergy libre software'' para encontrar trabajo sobre software libre y el concepto de \emph{estigmergia} (¿te he comentado que me gusta el concepto de estigmergia ya?), encontrarás un artículo que escribí hace tiempo cuyo título es ``Self-organized development in libre software: a model based on the stigmergy concept''.
Si pulsas sobre las comillas dobles (entre la estrella y el ``citado por ...'', justo debajo del extracto del resumen del artículo, te saldrá una ventana emergente con cómo citar.
Abajo a la derecha, aparece un enlace BibTeX.
Púlsalo y encontrarás la referencia en formato BibTeX, tal que así:

\clearpage
{\footnotesize
\begin{minted}{bibtex}
@inproceedings{robles2005self,
  title={Self-organized development in libre software:
         a model based on the stigmergy concept},
  author={Robles, Gregorio and Merelo, Juan Juli\'an 
          and Gonz\'alez-Barahona, Jes\'us M.},
  booktitle={ProSim'05},
  year={2005}
}
\end{minted}
}

Copia el texto en BibTeX y pégalo en el fichero \texttt{memoria.bib}, que es donde están las referencias bibliográficas.
Para incluir la referencia en el texto de la memoria, deberás citarlo, como hemos hecho antes con~\cite{bonabeau:_swarm}, lo que pasa es que en vez de el identificador de la cita anterior (bonabeau:\_swarm), tendrás que poner el nuevo (robles2005self).
Compila el fichero \texttt{memoria.tex} (\texttt{pdflatex memoria.tex}), añade la bibliografía (\texttt{bibtex memoria.aux}) y vuelve a compilar \texttt{memoria.tex} (\texttt{pdflatex memoria.tex})\ldots y \emph{voilà} ¡tenemos una nueva cita~\cite{robles2005self}!

También existe la posibilidad de poner notas al pie de página, por ejemplo, una para indicarte que visite la página del GSyC\footnote{\url{http://gsyc.es}}.

\section{Sección 1} 
\label{sec:seccion1}

Hemos hablado de cómo incluir figuras, pero no se ha descrito cómo incluir tablas.
A continuación se presenta un ejemplo de tabla, la Tabla \ref{tabla:ejemplo} (fíjate 
en cómo se introduce una referencia a la tabla).

\begin{table}
 \begin{center}
  \begin{tabular}{ | l | c | r |} % tenemos tres colummnas, la primera alineada a la izquierda (l), la segunda al centro (c) y la tercera a la derecha (r). El | indica que entre las columnas habrá una línea separadora.
    \hline
    Uno & 2 & 3 \\ \hline % el hline nos da una línea vertical
    Cuatro & 5 & 6 \\ \hline
    Siete & 8 & 9 \\
    \hline
  \end{tabular}
  \caption{Ejemplo de tabla. Aquí viene una pequeña descripción (el \emph{caption}) del contenido de la tabla. Si la tabla no es autoexplicativa, siempre viene bien aclararla aquí.}
  \label{tabla:ejemplo}
 \end{center}
\end{table}
\end{comment}

%%%%%%%%%%%%%%%%%%%%%%%%%%%%%%%%%%%%%%%%%%%%%%%%%%%%%%%%%%%%%%%%%%%%%%%%%%%%%%%
%%%%%%%%%%%%%%%%%%%%%%%%%%%%%%%%%%%%%%%%%%%%%%%%%%%%%%%%%%%%%%%%%%%%%%%%%%%%%%%
%%%%%%%%%%%%%%%%%%%         START HERE         %%%%%%%%%%%%%%%%%%%%%%%%%%%%%%%%
%%%%%%%%%%%%%%%%%%%%%%%%%%%%%%%%%%%%%%%%%%%%%%%%%%%%%%%%%%%%%%%%%%%%%%%%%%%%%%%
%%%%%%%%%%%%%%%%%%%%%%%%%%%%%%%%%%%%%%%%%%%%%%%%%%%%%%%%%%%%%%%%%%%%%%%%%%%%%%%

\section{Entorno de desarrollo: }
\label{sec:entorno_de_desarrollo}

\todo[inline]{VSCode + WSL}

%%-- El comando \gls{} permite incluir términos en el glosario, para luego reunirlos todos
%%-- en una tabla al comienzo o al final del documento, junto con sus definiciones.

% PyCharm es un \gls{ide} dedicado concretamente a la programación en Python y desarrollado por la compañía checa JetBrains.

% Proporciona análisis de código, un depurador gráfico, una consola de Python integrada, control de versiones y, además, soporta desarrollo web con Django. Todas estas características lo convierten en un entorno completo e intuitivo, idóneo para el desarrollo de proyectos académicos como el que nos ocupa.


\section{Redacción de la memoria: LaTeX/Overleaf}
\label{sec:redaccion_de_la_memoria}

% LaTeX es un sistema de composición tipográfica de alta calidad que incluye características especialmente diseñadas para la producción de documentación técnica y científica. Estas características, entre las que se encuentran la posibilidad de incluir expresiones matemáticas, fragmentos de código, tablas y referencias, junto con el hecho de que se distribuya como software libre, han hecho que LaTeX se convierta en el estándar de facto para la redacción y publicación de artículos académicos, tesis y todo tipo de documentos científico-técnicos. 

% Por su parte, Overleaf es un editor LaTeX colaborativo basado en la nube. Lanzado originalmente en 2012, fue creado por dos matemáticos que se inspiraron en su propia experiencia en el ámbito académico para crear una solución satisfactoria para la escritura científica colaborativa.

% Además de por su perfil colaborativo, Overleaf destaca porque, pese a que en LaTeX el escritor utiliza texto plano en lugar de texto formateado (como ocurre en otros procesadores de texto como Microsoft Word, LibreOffice Writer y Apple Pages), éste puede visualizar en todo momento y paralelamente el texto formateado que resulta de la escritura del código fuente.

\section{Codecarbon}

CodeCarbon es un paquete creado con la intención de permitir a desarrolladores monitorizar las emisiones de dióxido de carbono ($CO_{2}$) producidas por aplicaciones en Inteligencia Artificial y modelos de Aprendizaje Automático, que surge de la motivación de contar con una forma de registrar las enormes cantidades de energía que el auge de la IA ha provocado en la industria. El incremento del rendimiento y la precisión de los modelos de Aprendizaje Automático que se ha producido en años recientes se ha logrado a cambio de la utilización de enormes cantidades de información para conseguir el aprendizaje de los patrones y características subyacentes. Así, los modelos más avanzados emplean cantidades significativas de poder computacional, entrenando en procesadores avanzados durante semanas o meses y consumiendo en el proceso una gran cantidad de energía. Dependiendo de la red eléctrica utilizada, este desarrollo puede comportar la emisión de grandes cantidades de gases de efecto invernadero como el $CO_{2}$.

CodeCarbon estima la huella de carbono de una aplicación medida como kilogramos de $CO_{2}$ equivalentes, o $CO_{2}eq$, una medida estandarizada utilizada para expresar la capacidad de calentamiento global de varios gases de efecto invernadero como la cantidad de $CO_{2}$ que causaría un impacto ambiental equivalente. Para tareas de computación, que emiten $CO_{2}$ por medio de la electricidad que están consumiendo y que es generada como parte de la red eléctrica (por ejemplo, mediante la quema de combustibles fósiles como el carbón) las emisiones de carbono se miden en kilogramos de $CO_{2}$ equivalentes por kilovatio-hora. De esta forma, las emisiones de dióxido de carbono totales se calculan como el producto de la intensidad de carbono de la electricidad utilizada para la computación y la energía consumida por la infraestructura.

La intensidad de carbono de la electricidad se calcula como la media ponderada de las emisiones de las distintas fuentes de energía usadas para generar electricidad, incluyendo combustibles fósiles y renovables. En la herramienta se asigna un valor conocido de dióxido de carbono emitido por kilovatio-hora generado para cada uno de los combustibles (carbón, petróleo y gas natural). Otras fuentes renovables o consideradas como de bajo carbono incluyen la energía solar, hidroeléctrica, biomasa o geotérmica. La intensidad de carbono de cada combustible individual se calcula en base a medidas de generación de carbono y electricidad en los Estados Unidos, y aplicadas de forma generalizada en el resto del mundo. Cada red eléctrica local incluye una mezcla distinta de fuentes de energía y tiene asignada entonces una intensidad de carbono total particular.

% IMAGE: Global distribution of carbon intensity (carbonboard)
% TABLE: Carbon intensity by energy source (Codecarbon/Methodology)
% CITE: CodeCarbon documentation

% _opcional_ : explanation on zero value for low-carbon fuels
% _opcional_ : explanation on power consuption calculation by CPU

\section{Scikit-Learn}

Scikit-learn es un módulo desarrollado para Python que integra un amplio rango de algoritmos de aprendizaje automático de última generación para problemas tanto supervisados como no supervisados. Este paquete pretende llevar el aprendizaje automático a desarrolladores no especialistas mediante el uso de un lenguaje generalista de alto nivel. Se hace hincapié en la facilidad de uso, el rendimiento, la documentación y la consistencia de la API. Tiene las mínimas dependencias necesarias y está distribuido bajo la licencia BSD, con el objetivo de incentivar su uso tanto en ambientes educativos como comerciales.

Scikit-learn expone una gran variedad de algoritmos de aprendizaje utilizando una interfaz consistente y orientada a la resolución de tareas, lo que permite una comparación sencilla entre distintos métodos de aprendizaje para una misma aplicación. Al depender del ecosistema científico de Python, puede ser integrado con facilidad en aplicaciones que se salgan del rango tradicional del análisis estadístico de datos. Además, los algoritmos, que han sido implementados en un lenguaje de alto nivel, pueden ser utilizados como bloques de construcción para desarrollar estrategias más complejas que se adecuen a cada caso particular.

% CITE: Scickit-learn/About


\section{Stress}

\todo[inline]{Stress}

\section{Matplotlib / Altair}

\todo[inline]{Graphics libraries}

\section{Datos utilizados}

Todos los conjuntos de datos utilizados en los análisis realizados están disponibles de forma libre en la web y proceden de dos fuentes: el repositorio de aprendizaje automático de la Universidad de California Irvine y el proyecto Galaxy Zoo de clasificación de galaxias.

El repositorio de aprendizaje automático de la UCI es una colección de bases de datos, teorías de dominios y generadores de datos que son utilizados por la comunidad de aprendizaje automático para el análisis empírico de algoritmos de aprendizaje automático. El archivo fue creado en 1987 como un servidor FTP por David Aha y otros compañeros estudiantes en la UCI. Desde entonces ha sido ampliamente utilizado por estudiantes, educadores e investigadores de todo el mundo como una fuente primaria de conjuntos de datos para aprendizaje automático \cite{ml-uci}.

A continuación se describen las características más importantes de los conjuntos empleados en orden ascendente de tamaño.
% ANNEX All attributes

\subsection{Hepatitis}

Se trata de un conjunto de datos de pequeño tamaño (155 instancias) que se puede encontrar en el repositorio de la UCI\footnote{\url{https://archive.ics.uci.edu/ml/datasets/Hepatitis}}. Los datos proceden de pacientes diagnosticados de hepatitis y se clasifican de acuerdo a su supervivencia. Incluye 19 atributos sobre la historia del paciente, en los que la mayor parte (13) son atributos categóricos con dos posible valores y el resto son valores discretizados. Los datos fueron donados por Gail Gong en Noviembre de 1988. Trabajos pasados con este conjunto de datos obtuvieron medidas de precisión del 80\% \cite{hepatitis-gong}.

\subsection{Ionosfera}

Se trata de un conjunto de datos con 351 instancias procedente del repositorio de la UCI\footnote{\url{https://archive.ics.uci.edu/ml/datasets/Ionosphere}}. Contiene datos de radar obtenidos por el grupo de física espacial de la Universidad John Hopkins y donados por Vince Sigillito en 1989. El sistema radar está ubicado en Goose Bay, Labrador y consiste en un array de 16 antenas de alta frecuencia. El objetivo es la medición de electrones libres en la ionosfera y su clasificación binaria entre "buenas" respuestas del radar que indican evidencia de algún tipo de estructura en la ionosfera y "malas" respuestas en las que las señales simplemente pasan a través de la ionosfera. Las señales recibidas se procesaron utilizando una función de autocorrelación con el tiempo de pulso y el número de pulso como argumentos y cada una de las instancias del conjunto de datos está descrita por dos atributos continuos para cada uno de los 17 números de pulso, correspondientes al valor complejo obtenido de la señal electromagnética compleja. Hay por lo tanto un total de 34 características continuas por instancia.

\subsection{Billetes}

Este conjunto de datos contiene información extraída de 1372 imágenes tomadas para evaluar un procedimiento de autenticación de billetes. Fue donado en Agosto de 2012 por Volker Lohweg de la Universidad de Ciencias Aplicadas de Ostwestfalen-Lippe, Alemania, al repositorio de la UCI\footnote{\url{https://archive.ics.uci.edu/ml/datasets/banknote+authentication}}. Para la digitalización de las imágenes tomadas se empleó una cámara industrial normalmente utilizada para la inspección de impresiones. Las imágenes finales tienen un tamaño de 400x400 píxeles con una resolución de alrededor de 660 dpi en escala de grises. Posteriormente, se empleó una herramienta de transformada ondícula para extraer cuatro características continuas de las imágenes.

\subsection{Adultos}

Se trata de un conjunto de datos de mediano tamaño con casi 50.000 instancias. Contiene datos del censo poblacional de 1994 de los Estados Unidos, recogidos en Mayo de 1996 y donados al repositorio UCI\footnote{\url{https://archive.ics.uci.edu/ml/datasets/Adult}} por Ronny Kohavi y Barry Becker, de la sección de análisis de datos y visualización de Silicon Graphics. Cuenta con hasta 14 características categóricas y continuas extraídas del censo como edad, raza, nivel de educación o estado ocupacional. El objetivo de clasificación es predecir si los ingresos de la persona sobrepasan los 50 mil dolares al año.

\subsection{Galaxy Zoo}

\cite{galaxy-zoo}

%\cleardoublepage