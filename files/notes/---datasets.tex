\section{Datos utilizados}
\label{sec:datasets}

\todoin{Update with new datasets:  \\
  > Iris, Ionosphere, Banknote, Phoneme, Eeg-eye-state, Electricity \\
  > Covertype, Poker-hand \\
  > For-each: \\
  ---> content explain \\
  ---> size \\
  ---> features \\
  ---> origin, cites \\
  ---> what has been used for before}

Todos los conjuntos de datos utilizados en los análisis realizados están disponibles de forma libre en la web y proceden de dos fuentes: el repositorio de aprendizaje automático de la Universidad de California Irvine y la plataforma OpenML.

El repositorio de aprendizaje automático de la UCI \cite{ml-uci} es una colección de bases de datos, teorías de dominios y generadores de datos que son utilizados por la comunidad de aprendizaje automático para el análisis empírico de algoritmos de aprendizaje automático. El archivo fue creado en 1987 como un servidor FTP por David Aha y otros compañeros estudiantes en la UCI. Desde entonces ha sido ampliamente utilizado por estudiantes, educadores e investigadores de todo el mundo como una fuente primaria de conjuntos de datos para aprendizaje automático.

OpenML \cite{openml} es una plataforma abierta para compartir conjuntos de datos, algoritmos y experimentos, que tiene como objetivo lograr que las investigaciones sobre aprendizaje automático sean más fácilmente accesibles y reutilizables. Esta plataforma contiene un repositorio con más de cinco mil conjuntos de datos y resultados de experimentos que otros usuarios han realizado con ellos. Además, ofrece librerías para integrar la recuperación de sus datos directamente con el código de desarrollo de modelos de aprendizaje. Otras librerías como Pandas para procesado de datos, ofrecen también opciones para descargar los datos fácilmente de la plataforma.

A continuación se describen las características más importantes de los conjuntos de datos empleados.
% ANNEX All attributes

\subsection{Iris}

% This is perhaps the best known database to be found in the pattern recognition literature. Fisher's paper is a classic in the field and is referenced frequently to this day. (See Duda & Hart, for example.) The data set contains 3 classes of 50 instances each, where each class refers to a type of iris plant. One class is linearly separable from the other 2; the latter are NOT linearly separable from each other.

% Predicted attribute: class of iris plant.
% This is an exceedingly simple domain.

% Attribute Information:
% 1. sepal length in cm
% 2. sepal width in cm
% 3. petal length in cm
% 4. petal width in cm
% 5. class: 
%    -- Iris Setosa
%    -- Iris Versicolour
%    -- Iris Virginica

Iris esta entre las primeras bases de datos recogidas, y es una de las más conocidas y utilizadas en la literatura sobre reconocimiento de patrones. Los datos originales fueron publicados por R.A. Fisher en 1936, y la versión utilizada en este proyecto procede de la UCI (1988) \cite{iris-dataset}.
El conjunto contiene tres clases distintas con 50 ejemplos cada una, donde cada clase se refiere a una especie del género de plantas Iris. Cada ejemplo contiene cuatro atributos que describen la longitud y la anchura del sépalo y el pétalo de la planta en centímetros. Las tres opciones de clasificación son Iris Setosa, Iris Versicolour e Iris Virginica.

Se trata de un campo de clasificación extremadamente simple, donde una de las clases es linealmente separable de las otras dos, y estas últimas son separable entre sí de forma no lineal.



\subsection{Ionosfera}

Se trata de un conjunto de datos con 351 muestras procedente del repositorio de la UCI\footnote{\url{https://archive.ics.uci.edu/ml/datasets/Ionosphere}}. Contiene datos de radar obtenidos por el grupo de física espacial de la Universidad John Hopkins y donados por Vince Sigillito en 1989. El sistema radar está ubicado en Goose Bay, Labrador y consiste en un array de 16 antenas de alta frecuencia. El objetivo es la medición de electrones libres en la ionosfera y su clasificación binaria entre "buenas" respuestas del radar que indican evidencia de algún tipo de estructura en la ionosfera y "malas" respuestas en las que las señales simplemente pasan a través de la ionosfera. Las señales recibidas se procesaron utilizando una función de autocorrelación con el tiempo de pulso y el número de pulso como argumentos y cada una de las muestras del conjunto de datos está descrita por dos atributos continuos para cada uno de los 17 números de pulso, correspondientes al valor complejo obtenido de la señal electromagnética compleja. Hay por lo tanto un total de 34 características continuas por muestra.

\subsection{Billetes}

Este conjunto de datos contiene información extraída de 1372 imágenes tomadas para evaluar un procedimiento de autenticación de billetes. Fue donado en Agosto de 2012 por Volker Lohweg de la Universidad de Ciencias Aplicadas de Ostwestfalen-Lippe, Alemania, al repositorio de la UCI\footnote{\url{https://archive.ics.uci.edu/ml/datasets/banknote+authentication}}. Para la digitalización de las imágenes tomadas se empleó una cámara industrial normalmente utilizada para la inspección de impresiones. Las imágenes finales tienen un tamaño de 400x400 píxeles con una resolución de alrededor de 660 dpi en escala de grises. Posteriormente, se empleó una herramienta de transformada ondícula para extraer cuatro características continuas de las imágenes.

\subsection{Fonemas}
\subsection{Electroencefalograma}
\subsection{Electricidad}
% \subsection{Tipo de cubierta}
% \subsection{Mano de póquer}
