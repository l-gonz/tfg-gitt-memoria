\section{Datos utilizados}
\label{sec:datasets}

\todoin{Update with new datasets:  \\
  > Iris, Ionosphere, Banknote, Phoneme, Eeg-eye-state, Electricity \\
  > Covertype, Poker-hand \\
  > For-each: \\
  ---> content explain \\
  ---> size \\
  ---> features \\
  ---> origin, cites \\
  ---> what has been used for before}

Todos los conjuntos de datos utilizados en los análisis realizados están disponibles de forma libre en la web y proceden de dos fuentes: el repositorio de aprendizaje automático de la Universidad de California Irvine y la plataforma OpenML.

El repositorio de aprendizaje automático de la UCI es una colección de bases de datos, teorías de dominios y generadores de datos que son utilizados por la comunidad de aprendizaje automático para el análisis empírico de algoritmos de aprendizaje automático. El archivo fue creado en 1987 como un servidor FTP por David Aha y otros compañeros estudiantes en la UCI. Desde entonces ha sido ampliamente utilizado por estudiantes, educadores e investigadores de todo el mundo como una fuente primaria de conjuntos de datos para aprendizaje automático \cite{ml-uci}.

La plataforma OpenML ... \todo[inline]{Completar}

A continuación se describen las características más importantes de los conjuntos empleados en orden ascendente de tamaño.
% ANNEX All attributes

\subsection{Iris}

\subsection{Ionosfera}

Se trata de un conjunto de datos con 351 muestras procedente del repositorio de la UCI\footnote{\url{https://archive.ics.uci.edu/ml/datasets/Ionosphere}}. Contiene datos de radar obtenidos por el grupo de física espacial de la Universidad John Hopkins y donados por Vince Sigillito en 1989. El sistema radar está ubicado en Goose Bay, Labrador y consiste en un array de 16 antenas de alta frecuencia. El objetivo es la medición de electrones libres en la ionosfera y su clasificación binaria entre "buenas" respuestas del radar que indican evidencia de algún tipo de estructura en la ionosfera y "malas" respuestas en las que las señales simplemente pasan a través de la ionosfera. Las señales recibidas se procesaron utilizando una función de autocorrelación con el tiempo de pulso y el número de pulso como argumentos y cada una de las muestras del conjunto de datos está descrita por dos atributos continuos para cada uno de los 17 números de pulso, correspondientes al valor complejo obtenido de la señal electromagnética compleja. Hay por lo tanto un total de 34 características continuas por muestra.

\subsection{Billetes}

Este conjunto de datos contiene información extraída de 1372 imágenes tomadas para evaluar un procedimiento de autenticación de billetes. Fue donado en Agosto de 2012 por Volker Lohweg de la Universidad de Ciencias Aplicadas de Ostwestfalen-Lippe, Alemania, al repositorio de la UCI\footnote{\url{https://archive.ics.uci.edu/ml/datasets/banknote+authentication}}. Para la digitalización de las imágenes tomadas se empleó una cámara industrial normalmente utilizada para la inspección de impresiones. Las imágenes finales tienen un tamaño de 400x400 píxeles con una resolución de alrededor de 660 dpi en escala de grises. Posteriormente, se empleó una herramienta de transformada ondícula para extraer cuatro características continuas de las imágenes.

\subsection{Fonemas}
\subsection{Electroencefalograma}
\subsection{Electricidad}
\subsection{Tipo de cubierta}
\subsection{Mano de póquer}
