\section{Codecarbon}

CodeCarbon es un paquete creado con la intención de permitir a desarrolladores monitorizar las emisiones de dióxido de carbono ($CO_{2}$) producidas por aplicaciones en Inteligencia Artificial y modelos de Aprendizaje Automático, que surge de la motivación de contar con una forma de registrar las enormes cantidades de energía que el auge de la IA ha provocado en la industria. El incremento del rendimiento y la precisión de los modelos de Aprendizaje Automático que se ha producido en años recientes se ha logrado a cambio de la utilización de enormes cantidades de información para conseguir el aprendizaje de los patrones y características subyacentes. Así, los modelos más avanzados emplean cantidades significativas de poder computacional, entrenando en procesadores avanzados durante semanas o meses y consumiendo en el proceso una gran cantidad de energía. Dependiendo de la red eléctrica utilizada, este desarrollo puede comportar la emisión de grandes cantidades de gases de efecto invernadero como el $CO_{2}$.

CodeCarbon estima la huella de carbono de una aplicación medida como kilogramos de $CO_{2}$ equivalentes, o $CO_{2}eq$, una medida estandarizada utilizada para expresar la capacidad de calentamiento global de varios gases de efecto invernadero como la cantidad de $CO_{2}$ que causaría un impacto ambiental equivalente. Para tareas de computación, que emiten $CO_{2}$ por medio de la electricidad que están consumiendo y que es generada como parte de la red eléctrica (por ejemplo, mediante la quema de combustibles fósiles como el carbón) las emisiones de carbono se miden en kilogramos de $CO_{2}$ equivalentes por kilovatio-hora. De esta forma, las emisiones de dióxido de carbono totales se calculan como el producto de la intensidad de carbono de la electricidad utilizada para la computación y la energía consumida por la infraestructura.

La intensidad de carbono de la electricidad se calcula como la media ponderada de las emisiones de las distintas fuentes de energía usadas para generar electricidad, incluyendo combustibles fósiles y renovables. En la herramienta se asigna un valor conocido de dióxido de carbono emitido por kilovatio-hora generado para cada uno de los combustibles (carbón, petróleo y gas natural). Otras fuentes renovables o consideradas como de bajo carbono incluyen la energía solar, hidroeléctrica, biomasa o geotérmica. La intensidad de carbono de cada combustible individual se calcula en base a medidas de generación de carbono y electricidad en los Estados Unidos, y aplicadas de forma generalizada en el resto del mundo. Cada red eléctrica local incluye una mezcla distinta de fuentes de energía y tiene asignada entonces una intensidad de carbono total particular.

% IMAGE: Global distribution of carbon intensity (carbonboard)
% TABLE: Carbon intensity by energy source (Codecarbon/Methodology)
% CITE: CodeCarbon documentation

% _opcional_ : explanation on zero value for low-carbon fuels
% _opcional_ : explanation on power consuption calculation by CPU

\section{Scikit-Learn}

Scikit-learn es un módulo desarrollado para Python que integra un amplio rango de algoritmos de aprendizaje automático de última generación para problemas tanto supervisados como no supervisados. Este paquete pretende llevar el aprendizaje automático a desarrolladores no especialistas mediante el uso de un lenguaje generalista de alto nivel. Se hace hincapié en la facilidad de uso, el rendimiento, la documentación y la consistencia de la API. Tiene las mínimas dependencias necesarias y está distribuido bajo la licencia BSD, con el objetivo de incentivar su uso tanto en ambientes educativos como comerciales.

Scikit-learn expone una gran variedad de algoritmos de aprendizaje utilizando una interfaz consistente y orientada a la resolución de tareas, lo que permite una comparación sencilla entre distintos métodos de aprendizaje para una misma aplicación. Al depender del ecosistema científico de Python, puede ser integrado con facilidad en aplicaciones que se salgan del rango tradicional del análisis estadístico de datos. Además, los algoritmos, que han sido implementados en un lenguaje de alto nivel, pueden ser utilizados como bloques de construcción para desarrollar estrategias más complejas que se adecuen a cada caso particular.

% CITE: Scickit-learn/About