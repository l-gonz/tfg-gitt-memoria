\chapter{Diseño e implementación}
\label{chap:diseño}


Aquí viene todo lo que has hecho tú (tecnológicamente). 
Puedes entrar hasta el detalle. 
Es la parte más importante de la memoria, porque describe lo que has hecho tú.
Eso sí, normalmente aconsejo no poner código, sino diagramas.

\section{Arquitectura general} 
\label{sec:arquitectura}

Si tu proyecto es un software, siempre es bueno poner la arquitectura (que es cómo se estructura tu programa a ``vista de pájaro'').

Por ejemplo, puedes verlo en la Figura~\ref{fig:arquitectura}.
\LaTeX \ pone las figuras donde mejor cuadran. 
Y eso quiere decir que quizás no lo haga donde lo hemos puesto\ldots
Eso no es malo.
A veces queda un poco raro, pero es la filosofía de \LaTeX: tú al contenido, que yo me encargo de la maquetación.

\begin{figure}
  \centering
  \includegraphics[width=9cm, keepaspectratio]{img/arquitectura.png}
  \caption{Estructura del parser básico.}\label{fig:arquitectura}
\end{figure}

\begin{figure}
    \centering
    \includegraphics[bb=0 0 800 600, width=12cm, keepaspectratio]{img/foro1}
    \caption{Página con enlaces a hilos}\label{fig:_arquitectura}
\end{figure}

 
Recuerda que toda figura que añadas a tu memoria debe ser explicada.
Sí, aunque te parezca evidente lo que se ve en la Figura~\ref{fig:arquitectura}, la figura en sí solamente es un apoyo a tu texto.
Así que explica lo que se ve en la Figura, haciendo referencia a la misma tal y como ves aquí.
Por ejemplo: En la Figura~\ref{fig:arquitectura} se puede ver que la estructura del \emph{parser} básico, que consta de seis componentes diferentes: los datos se obtienen de la red, y según el tipo de dato, se pasará a un \emph{parser} específico y bla, bla, bla\ldots

Si utilizas una base de datos, no te olvides de incluir también un diagrama de entidad-relación.

%\cleardoublepage