%\cleardoublepage
\chapter{Introducción}
\label{sec:intro}
\pagenumbering{arabic} % para empezar la numeración de página con números

%En este capítulo se introduce el proyecto.
%Debería tener información general sobre el mismo, dando la información sobre el contexto en el que se ha desarrollado.

%No te olvides de echarle un ojo a la página con los cinco errores de escritura más frecuentes\footnote{\url{http://www.tallerdeescritores.com/errores-de-escritura-frecuentes}}.

%Aconsejo a todo el mundo que mire y se inspire en memorias pasadas.
%Las memorias de los proyectos que he llevado yo están (casi) todas almacenadas en mi web del GSyC\footnote{\url{https://gsyc.urjc.es/~grex/pfcs/}}.

%%%%%%%%%%%%%%%%%%%

El aprendizaje automático (\emph{Machine Learning} en inglés) es una rama de la inteligencia artificial y la ciencia computacional que se centra en el uso de datos y algoritmos para imitar la forma en la que los humanos aprenden con el objetivo de aumentar gradualmente su precisión.
Es un componente fundamental del campo de la ciencia de datos, cuya importancia ha experimentado un gran crecimiento recientemente. 
El aprendizaje automático hace uso de métodos estadísticos para entrenar algoritmos que hacen clasificaciones o predicciones y que permiten descubrir piezas clave de información dentro de proyectos de procesamiento de datos. 
Esta información afecta posteriormente en la toma de decisiones dentro de distintas aplicaciones y negocios, con una gran capacidad de impactar en el crecimiento de los mismos.

\section{Aprendizaje automático y consumo energético}

Gracias al desarrollo de nuevas tecnologías computacionales, el aprendizaje automático que se utiliza hoy en día es muy diferente de como era en el pasado.
Este surgió del reconocimiento de patrones y de la teoría de que los ordenadores pueden aprender sin necesidad de ser específicamente programados para resolver tareas específicas, cuando los investigadores interesados en la inteligencia artificial se empezaron a plantear si los ordenadores serían capaces de aprender a partir de datos.
De aquí surge la importancia del aspecto iterativo de este aprendizaje, que permite que el sistema se adapte de forma independiente cada vez que nuevos datos son incorporados y sea capaz de aprender de cada computación previa para producir decisiones y resultados que sean confiables y repetibles.

Es así que, aunque una gran parte de los algoritmos utilizados en el aprendizaje automático son conocidos desde hace relativamente bastante tiempo, la habilidad de automáticamente aplicar complejos cálculos matemáticos a grandes cantidades de datos una y otra vez, cada vez más rápidamente, es un desarrollo muy reciente conseguido gracias a los avances en componentes informáticos y la disminución de costes de grandes sistemas computacionales con enormes capacidades de memoria y procesamiento.
Este es especialmente el caso para el campo del aprendizaje profundo (\emph{deep learning}), donde los modelos han crecido en cálculos para llegar a alcanzar típicamente el orden de los GigaFlops y en requisitos de memoria que se encuentran típicamente en el orden de los millones de parámetros.

Sin embargo, este gran poder de procesamiento trae consigo un gran gasto energético.
El consumo de energía en la arquitectura de computadores ha sido foco de atención de investigadores interesados en obtener procesadores eficientes energéticamente de última generación durante décadas. 
Por otro lado, los investigadores interesados en el aprendizaje automático se han centrado principalmente en la producción de modelos cada vez más profundos y precisos, sin poner ningún límite en términos computacionales más allá de la disponibilidad de procesadores capaces.

% Some awareness in energy consumption is starting to arise, originating from a few machine learning research groups [12], [14], [47], [61] and challenges such as The Low Power Image Recognition Challenge (LPIRC) [26]. Thus, we believe that efforts towards estimating energy consumption and developing tools for researchers to advance their research in energy consumption are necessary for a more scalable and sustainable future. %

% MORE Por qué clasificación, menciones a otros proyectos de análisis energético
% CITE https://www.sciencedirect.com/science/article/pii/S0743731518308773#b18

%%-- Objetivos del  proyecto
%%-- Si la sección anterior ha quedado muy extensa, se puede considerar convertir
%%-- Las siguientes tres secciones en un capítulo independiente de la memoria

\section{Objetivos del proyecto}
\label{sec:objetivos}

\subsection{Objetivo general} % título de subsección (se muestra)
\label{sec:objetivo-general} % identificador de subsección (no se muestra, es para poder referenciarla)


Este Trabajo de Fin de Grado tiene como objetivo crear una herramienta que permita la comparación sistemática del consumo energético y el impacto de la huella de carbono en los modelos más representativos de técnicas de clasificación de aprendizaje automático supervisado.


\subsection{Objetivos específicos}
\label{sec:objetivos-especificos}

Para lograr esta meta se han tenido en cuenta los siguientes objetivos específicos:

    \begin{itemize}
        \item Estudiar los algoritmos de clasificación más importantes.
        \item Analizar la relación entre la precisión de los modelos y su consumo energético en conjuntos de datos de distintos tamaños y características.
        \item \todo[inline]{Completar objetivos específicos}
    \end{itemize}

\section{Planificación temporal}
\label{sec:planificacion-temporal}

\todo[inline]{Completar con información temporal de Clockify}
% Es conveniente que incluyas una descripción de lo que te ha llevado realizar el trabajo.
% Hay gente que añade un diagrama de GANTT.
% Lo importante es que quede claro cuánto tiempo has consumido en realizar el TFG/TFM 
% (tiempo natural, p.ej., 6 meses) y a qué nivel de esfuerzo (p.ej., principalmente los 
% fines de semana).

\section{Estructura de la memoria}
\label{sec:estructura}

%% al final %%
\todo[inline]{TODO: estructura}

Por último, en esta sección se introduce a alto nivel la organización del resto del documento
y qué contenidos se van a encontrar en cada capítulo.

    \begin{itemize}
      \item En el primer capítulo se hace una breve introducción al proyecto, se describen los objetivos del mismo y se refleja la planificación temporal.
      \item En el siguiente capítulo se describen las tecnologías utilizadas en el desarrollo de este TFM/TFG (Capítulo~\ref{chap:tecnologias}).
      \item En el capítulo~\ref{chap:diseño} Se describe el proceso de desarrollo
      de la herramienta \ldots
      \item En el capítulo~\ref{chap:experimentos} Se presentan las principales pruebas realizadas
      para validación de la plataforma/herramienta\ldots (o resultados de los experimentos
      efectuados).
      \item Por último, se presentan las conclusiones del proyecto así como los trabajos futuros que podrían derivarse de éste (Capítulo~\ref{chap:conclusiones}).
    \end{itemize}

%%%%%%%%%%%%%%%%%%%%%%%%%%%%%%%%%%%%%%%%%%%%%%%%%%%%%%%%%%%%%%%%%%%%%%%%%%%%%%%
%%%%%%%%%%%%%%%%%%%%%%%%%%%%%%%%%%%%%%%%%%%%%%%%%%%%%%%%%%%%%%%%%%%%%%%%%%%%%%%
%%%%%%%%%%%%%%%%%%%          NOTES             %%%%%%%%%%%%%%%%%%%%%%%%%%%%%%%%
%%%%%%%%%%%%%%%%%%%%%%%%%%%%%%%%%%%%%%%%%%%%%%%%%%%%%%%%%%%%%%%%%%%%%%%%%%%%%%%
%%%%%%%%%%%%%%%%%%%%%%%%%%%%%%%%%%%%%%%%%%%%%%%%%%%%%%%%%%%%%%%%%%%%%%%%%%%%%%%
\begin{comment}
\section{Sección}
\label{sec:seccion}

Esto es una sección, que es una estructura menor que un capítulo. 

Por cierto, a veces me comentáis que no os compila por las tildes.
Eso es un problema de codificación.
Al guardar el archivo, guardad la codificación de ``ISO-Latin-1'' a ``UTF-8'' (o viceversa) y funcionará.

\subsection{Estilo}
\label{subsec:estilo}

Recomiendo leer los consejos prácticos sobre escribir documentos científicos en \LaTeX \ de Diomidis Spinellis\footnote{\url{https://github.com/dspinellis/latex-advice}}.

Lee sobre el uso de las comas\footnote{\url{http://narrativabreve.com/2015/02/opiniones-de-un-corrector-de-estilo-11-recetas-para-escribir-correctamente-la-coma.html}}. 
Las comas en español no se ponen al tuntún.
Y nunca, nunca entre el sujeto y el predicado (p.ej. en ``Yo, hago el TFG'' sobre la coma).
La coma no debe separar el sujeto del predicado en una oración, pues se cortaría la secuencia natural del discurso.
No se considera apropiado el uso de la llamada coma respiratoria o \emph{coma criminal}.
Solamente se suele escribir una coma para marcar el lugar que queda cuando omitimos el verbo de una oración, pero es un caso que se da de manera muy infrecuente al escribir un texto científico (p.ej. ``El Real Madrid, campeón de Europa'').

A continuación, viene una figura, la Figura~\ref{figura:foro_hilos}. 
Observarás que el texto dentro de la referencia es el identificador de la figura (que se corresponden con el ``label'' dentro de la misma). 
También habrás tomado nota de cómo se ponen las ``comillas dobles'' para que se muestren correctamente. 
Nota que hay unas comillas de inicio (``) y otras de cierre (''), y que son diferentes.
Volviendo a las referencias, nota que al compilar, la primera vez se crea un diccionario con las referencias, y en la segunda compilación se ``rellenan'' estas referencias. 
Por eso hay que compilar dos veces tu memoria.
Si no, no se crearán las referencias.

 \begin{figure}
    \centering
    \includegraphics[bb=0 0 800 600, width=12cm, keepaspectratio]{img/foro1}
    \caption{Página con enlaces a hilos}
    \label{figura:foro_hilos}
 \end{figure}

A continuación un bloque ``verbatim'', que se utiliza para mostrar texto tal cual.
Se puede utilizar para ofrecer el contenido de correos electrónicos, código, entre otras cosas.

{\footnotesize
\begin{verbatim}
    From gaurav at gold-solutions.co.uk  Fri Jan 14 14:51:11 2005
    From: gaurav at gold-solutions.co.uk (gaurav_gold)
    Date: Fri Jan 14 19:25:51 2005
    Subject: [Mailman-Users] mailman issues
    Message-ID: <003c01c4fa40$1d99b4c0$94592252@gaurav7klgnyif>
    Dear Sir/Madam,
    How can people reply to the mailing list?  How do i turn off
    this feature? How can i also enable a feature where if someone
    replies the newsletter the email gets deleted?
    Thanks
    From msapiro at value.net  Fri Jan 14 19:48:51 2005
    From: msapiro at value.net (Mark Sapiro)
    Date: Fri Jan 14 19:49:04 2005
    Subject: [Mailman-Users] mailman issues
    In-Reply-To: <003c01c4fa40$1d99b4c0$94592252@gaurav7klgnyif>
    Message-ID: <PC173020050114104851057801b04d55@msapiro>
    gaurav_gold wrote:
    >How can people reply to the mailing list?  How do i turn off
    this feature? How can i also enable a feature where if someone
    replies the newsletter the email gets deleted?
    See the FAQ
    >Mailman FAQ: http://www.python.org/cgi-bin/faqw-mm.py
    article 3.11
\end{verbatim}
}

\end{comment}

%\cleardoublepage