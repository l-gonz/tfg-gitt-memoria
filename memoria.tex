%%%%%%%%%%%%%%%%%%%%%%%%%%%%%%%%%%%%%%%%%%%%%%%%%%%%%%%%%%%%%%%%%%%%%%%%%%%%%%%%
%% Plantilla de memoria en LaTeX para TFG/TFM - Universidad Rey Juan Carlos
%%
%% Por Gregorio Robles <grex arroba gsyc.urjc.es>
%%     Felipe Ortega   <felipe.ortega@urjc.es>
%%     Grupo de Sistemas y Comunicaciones (GSyC)
%%     Escuela Técnica Superior de Ingenieros de Telecomunicación
%%     Universidad Rey Juan Carlos
%%
%% (Muchas ideas tomadas de Internet, colegas del GSyC, antiguos alumnos...
%%  etc. Muchas gracias a todos)
%%
%% La última versión de esta plantilla está siempre disponible en:
%%     https://github.com/glimmerphoenix/plantilla-memoria
%%
%% - Ejecución en sistema local:
%% Para obtener el documento en PDF, ejecuta en la shell:
%%   make
%%
%% A diferencia de la anterior versión, que usaba la herramienta pdfLaTeX 
%% para compilar el documento, esta nueva versión de la plantilla usa
%% XeLaTeX. Es un compilador más moderno que, entre otras mejoras, incluye
%% soporte nativo para caracteres con codificación UTF-8, traducción políglota
%% de referencias (usando Biblatex) y soporte para fuentes OTF. Esta última
%% característic permite, por ejemplo, insertar iconos de la colección 
%% Fontawesome en el texto.
%%
%% XeLaTeX viene ya incluido en todas las distribuciones modernas de LaTeX.
%%
%% - Edición y ejecución en línea: 
%% Puedes descargar y subir la plantilla a
%% Overleaf, un editor de LaTeX colaborativo en línea. Overleaf ya tiene
%% instalados todos los paquetes LaTeX y otras dependencias software para
%% que esta plantilla compile correctamente.
%%
%% IMPORTANTE: Si compilas este documento en Overleaf, recuerda cambiar
%% la configuración (botón "Menu" en la esquina superior izquierda de la interfaz)
%% y elegir la opción Compiler --> XeLaTeX. En caso contrario no funcionará.
%%
%% - Nota: las imágenes deben ir en PNG, JPG, EPS o PDF. También se pueden usar
%% imágenes en otros formatos con algunos cambios en el preámbulo del documento.

%%%%%%%%%%%%%%%%%%%%%%%%%%%%%%%%%%%%%%%%%%%%%%%%%%%%%%%%%%%%%%%%%%%%%%%%%%%%%%%%

\documentclass[a4paper, 12pt]{book}

%%-- Geometría principal (dejar activada la siguiente línea en la versión final)
\usepackage[a4paper, left=2.5cm, right=2.5cm, top=3cm, bottom=3cm]{geometry}
%%-- Activar esta línea y comentar la anterior en modo borrador, para comentarios al margen
%\usepackage[a4paper, left=2.5cm, right=2.5cm, top=3cm, bottom=3cm, marginparwidth=60pt]{geometry}

%%-- Hay que cargarlo antes que las traducciones
\usepackage{listing}                    % Listados de código

% Traducciones en XeLaTeX
\usepackage{polyglossia}
\setmainlanguage{spanish}    % Comenta esta línea si tu memoria es en inglés

% Traducciones particulares para español
% Caption tablas
\gappto\captionsspanish{
	\def\tablename{Tabla}
	\def\listingscaption{Código}
	\def\refname{Bibliografía}
	\def\appendixname{Apéndice}
	\def\listtablename{Índice de tablas}
	\def\listingname{Código}
	\def\listlistingname{Índice de fragmentos de código}
}

%% Tipografía y estilos
\usepackage[OT1]{fontenc}               % Keeps eulervm happy about accents encoding

% Símbolos y fuentes matemáticas elegantes: Euler virtual math fonts
% ¡Importante! Carga siempre las fuentes math AMS Euler ANTES QUE fontspec
\usepackage{amsmath}
\usepackage{amssymb}
\usepackage[OT1,euler-digits,euler-hat-accent,small]{eulervm}

% En XeLaTeX las fuentes se especifican con fontspec
\usepackage{fontspec}
\defaultfontfeatures{Scale=MatchLowercase, Ligatures=TeX}     % Default option in font config

% Fix para fuentes usadas con operadores y \mathrm
\DeclareSymbolFont{operators}{\encodingdefault}{\familydefault}{m}{n}

% Configura la fuente principal (serif): MinionPro
\setmainfont[Scale=0.96]{TeX Gyre Pagella}
% Configura la fuente sans-serif (\sffamily)
\setsansfont[Scale=MatchLowercase]{Lato}
% Configura la fuente para letra monoespaciada: Source Code Pro, escala 0.85
\setmonofont[Scale=0.85]{Source Code Pro}

%%-- Familias de fuentes específicas
%%-- Se pueden definir etiquetas para familias de fuentes personalizadas
%%-- que luego puedes emplear para cambiar el formato de una parte de texto
%%-- Ejemplo:
% \newfontfamily{\myriadprocond}{Myriad Pro Semibold Condensed.otf}

%%-- Opciones de interlineado y espacios
\linespread{1.07}                   % Aumentar interlineado para fuentes tipo Palatino
\setlength{\parskip}{\baselineskip} % Separar párrafos con línea en blanco

%%-- Hipervínculos
\usepackage{url}

%%-- Gráficos y tablas
\PassOptionsToPackage{
    dvipdfmx,usenames,dvipsnames,
    x11names,table}{xcolor}             % Definiciones de colores
\PassOptionsToPackage{xetex}{graphicx}

\usepackage{subfig}                     % Subfiguras
\usepackage{pgf}
\usepackage{svg}                        % Integración de imágenes en formato SVG
\usepackage{float}                      % H para posicionar figuras
\usepackage{booktabs}                   % Already loads package xcolor
\usepackage{multicol}                   % multiple column layout facilities
\usepackage{colortbl}                   % For coloured tables

%%-- Bibliografía con Biblatex y Biber
% Más info:
% https://www.overleaf.com/learn/latex/Biblatex_bibliography_styles
% https://www.overleaf.com/learn/latex/biblatex_citation_styles
\usepackage[
    backend=biber,
    style=numeric,
    sorting=nty
    ]{biblatex}
\addbibresource{memoria.bib}
\DeclareFieldFormat{url}{\mkbibacro{URL}\addcolon\nobreakspace\url{#1}}
%\usepackage[nottoc, notlot, notlof, notindex]{tocbibind} %% Opciones de índice

%%-- Matemáticas e ingeniería
% El paquete units permite mostrar unidades correctamente
% Permite escribir unidades con espaciado y estilo de fuente correctos
\usepackage[ugly]{units}         
% Ejemplo de uso: $\unit[100]{m}$ or $\unitfrac[100]{m}{s}$
% Entornos matemáticos
\newtheorem{theorem}{Theorem}

% Paquetes adicionales
\usepackage{url}                        %% Gestión correcta de enlaces
\usepackage{float}                      %% H para posicionar figuras
\usepackage[nottoc, notlot, notlof, notindex]{tocbibind}    %% Opciones de índice
\usepackage{metalogo}                   %% Múltiples logos para XeLaTeX

% Fuentes especiales y glifos
\usepackage{ccicons}                % Creative Commons icons
\usepackage{metalogo}               % XeTeX logo
\usepackage{fontawesome5}           % Fontawesome 5 icons
\usepackage{adforn} 

% Blindtext
% Opciones pangram, bible, random (defecto)
\usepackage[pangram]{blindtext}
% Lorem ipsum
\usepackage{lipsum}
% Kant lipsum
\usepackage{kantlipsum}

\usepackage{fancyvrb}               % Entornos verbatim extendidos
	\fvset{fontsize=\normalsize}    % Tamaño de fuente por defecto en fancy-verbatim
	
% Configura listas (itemize, enumerate) con iconos personalizados
% Fácil reinicio de numeración con enumerate
% Info: http://ctan.org/pkg/enumitem
\usepackage[shortlabels]{enumitem}
% Usar \usageitem para configurar iconos personalizados en listas
\newcommand{\usageitem}[1]{%
	\item[%
	{\makebox[2em]{\strut\color{GSyCblue} #1}}%
	]
}

%%-- Definición de colores personalizados
% \definecolor{LightGrey}{HTML}{EEEEEE}
% \definecolor{darkred}{rgb}{0.5,0,0}     %% Refs. cruzadas
% \definecolor{darkgreen}{rgb}{0,0.5,0}   %% Citas bibliográficas
% \definecolor{darkblue}{rgb}{0,0,0.5}    %% Hiperenlaces ordinarios (también ToC)

%%-- Configuración fragmentos de código
%%-- Minted necesita Python Pygments instalado en el sistema para funcionar
%%-- En Overleaf ya está instalada esta dependencia
% \usepackage[center, labelfont=bf]{caption}
\usepackage{minted}

%%-- Se debe cargar aquí para evitar warnings
\usepackage{csquotes}                   % Para traducciones con biblatex

%%-- Glosario de términos
\usepackage[acronym]{glossaries}
\makeglossaries
\loadglsentries{glossary}

% % Definición de cabeceras del documento, usando fancyhdr
% \usepackage{fancyhdr}
% %% Configuración de cabeceras para el cuerpo principal del documento
% \pagestyle{fancy}
% \fancyhead{}
% \fancyhead[RO,LE]{\myriadprocond{\thepage}}
% \renewcommand{\chaptermark}[1]{\markboth{\chaptername\ \thechapter.\ #1}{}}
% \renewcommand{\sectionmark}[1]{\markright{\thesection.\ #1}}
% \fancyhead[RE]{\myriadprocond{\leftmark}}
% \fancyhead[LO]{\myriadprocond{\rightmark}}
% \renewcommand{\headrulewidth}{0pt}
% \setlength{\headheight}{15pt} %% Al menos 15pt para evitar warning al compilar
% \fancyfoot{}
% %% Configuración para páginas con cabecera en blanco
% \fancypagestyle{plain}{%
% \fancyhf{}% clear all header and footer fields
% \fancyhead[RO,LE]{\myriadprocond{\thepage}}
% \renewcommand{\headrulewidth}{0pt}%
% \renewcommand{\footrulewidth}{0pt}%
% }

%%-- Metadatos del doc
\title{Memoria del Proyecto}
\author{Nombre del autor}

%%-- Hiperenlaces, siempre se carga al final del preámbulo
\usepackage[colorlinks]{hyperref}
\hypersetup{
    pdftoolbar=true,	% Muestra barra de herramientas en Adobe Acrobat
	pdfmenubar=true,	% Muestra menú en Adobe Acrobat
	pdftitle={Título doc en ventana del visor o navegador},
	pdfauthor={Nombre del alumno/a},
	pdfcreator={ETSII/ETSIT, URJC},
	pdfproducer={XeLaTeX},
	pdfsubject={Topic1, Topic2, Topic3},
	pdfnewwindow=true,              %links open in new window
    colorlinks=true,                % false: boxed links; true: coloured links
    linkcolor=Firebrick4,           % enlaces internos 
    citecolor=Aquamarine4,          % enlaces a citas bibliográficas
    urlcolor=RoyalBlue3,            % hiperenlances ordinarios
    linktocpage=true                % Enlaces en núm. pág. en ToC
}

%%%---------------------------------------------------------------------------
% Comentarios en línea de revisión
% Este bloque se puede borrar cuando finalizamos el borrador

\usepackage[colorinlistoftodos]{todonotes}
\usepackage{verbatim}
%%%---------------------------------------------------------------------------

\begin{document}

%%-- Configuración común para todos los entornos listing
%%-- Descomentar para usar y personalizar valores
%\lstset{%
%breakatwhitespace=true,
% breaklines=true, 
% basicstyle=\footnotesize\ttfamily,
% keywordstyle=\color{blue},
% commentstyle=\color{green!40!black}, 
% language=Python} 
 

%%%%%%%%%%%%%%%%%%%%%%%%%%%%%%%%%%%%%%%%%%%%%%%%%%%%%%%%%%%%%%%%%%%%%%%%%%%%%%%%
% PORTADA

\begin{titlepage}
\begin{center}
\begin{tabular}[c]{c c}
%\includegraphics[bb=0 0 194 352, scale=0.25]{logo} &
\includegraphics[scale=1.5]{img/LogoURJC.png}
%&
%\begin{tabular}[b]{l}
%\Huge
%\textsf{UNIVERSIDAD} \\
%\Huge
%\textsf{REY JUAN CARLOS} \\
%\end{tabular}
\\
\end{tabular}

\vspace{3cm}

\Large 
GRADO EN INGENIERÍA EN TECNOLOGÍAS DE LA TELECOMUNICACIÓN

\vspace{0.4cm}

\large
Curso Académico 2023/2024

\vspace{0.8cm}

Trabajo Fin de Grado

\vspace{2cm}

\LARGE Comparativa de modelos de aprendizaje automático con respecto a su consumo energético
\vspace{3cm}

\large
Autor/a : Laura González Fernández \\
Tutor/a : Dr. Nombre del Profesor/a
\end{center}
\end{titlepage}

\newpage
\mbox{}
\thispagestyle{empty} % para que no se numere esta pagina


%%%%%%%%%%%%%%%%%%%%%%%%%%%%%%%%%%%%%%%%%%%%%%%%%%%%%%%%%%%%%%%%%%%%%%%%%%%%%%%%
%%%% Para firmar
\clearpage
\pagenumbering{gobble}
\chapter*{}

\vspace{-4cm}
\begin{center}
\LARGE
\textbf{Trabajo Fin de Grado}

\vspace{1cm}
\large
Comparativa del Consumo Energético de Algoritmos de Aprendizaje Automático

\vspace{1cm}
\large
\textbf{Autora :} Laura González Fernández  \\
\textbf{Tutor/a :} Dr. Nombre del profesor/a

\end{center}

\vspace{1cm}
La defensa del presente Proyecto Fin de Grado se realizó el día _\qquad$\;\,$ de
\qquad\qquad\qquad\qquad \newline de 2024, siendo calificada por el siguiente tribunal:


\vspace{0.5cm}
\textbf{Presidente:}

\vspace{0.8cm}
\textbf{Secretario:}

\vspace{0.8cm}
\textbf{Vocal:}


\vspace{0.8cm}
y habiendo obtenido la siguiente calificación:

\vspace{0.8cm}
\textbf{Calificación:}


\vspace{0.8cm}
\begin{flushright}
Móstoles/Fuenlabrada, a \qquad$\;\,$ de \qquad\qquad\qquad\qquad de 20XX
\end{flushright}

%%%%%%%%%%%%%%%%%%%%%%%%%%%%%%%%%%%%%%%%%%%%%%%%%%%%%%%%%%%%%%%%%%%%%%%%%%%%%%%%
%%%% Dedicatoria

% \chapter*{}
% %\pagenumbering{Roman} % para comenzar la numeración de paginas en numeros romanos
% \begin{flushright}
% \textit{Aquí normalmente \\
% se inserta una dedicatoria corta \\}
% \end{flushright}

%%%%%%%%%%%%%%%%%%%%%%%%%%%%%%%%%%%%%%%%%%%%%%%%%%%%%%%%%%%%%%%%%%%%%%%%%%%%%%%%
%%%% Agradecimientos

\chapter*{Agradecimientos}
%\addcontentsline{toc}{chapter}{Agradecimientos} % si queremos que aparezca en el índice
\markboth{AGRADECIMIENTOS}{AGRADECIMIENTOS} % encabezado 

Aquí vienen los agradecimientos\ldots

Hay más espacio para explayarse y explicar a quién agradeces su apoyo o ayuda para
haber acabado el proyecto: familia, pareja, amigos, compañeros de clase\ldots

También hay quien, en algunos casos, hasta agradecer a su tutor o tutores del proyecto
la ayuda prestada\ldots

%%%%%%%%%%%%%%%%%%%%%%%%%%%%%%%%%%%%%%%%%%%%%%%%%%%%%%%%%%%%%%%%%%%%%%%%%%%%%%%%
%%%% Resumen

\chapter*{Resumen}
%\addcontentsline{toc}{chapter}{Resumen} % si queremos que aparezca en el índice
\markboth{RESUMEN}{RESUMEN} % encabezado
%(máximo una página, se recomienda la estructura: antecedentes,
%objetivos, métodos, resultados y conclusiones)

Aquí viene un resumen del proyecto.
Ha de constar de tres o cuatro párrafos, donde se presente de manera clara y concisa de qué va el proyecto. 
Han de quedar respondidas las siguientes preguntas:

\begin{itemize}
  \item ¿De qué va este proyecto? ¿Cuál es su objetivo principal?
  \item ¿Cómo se ha realizado? ¿Qué tecnologías están involucradas?
  \item ¿En qué contexto se ha realizado el proyecto? ¿Es un proyecto dentro de un marco general?
\end{itemize}

Lo mejor es escribir el resumen al final.
\\\noindent\rule{\textwidth}{0.4pt}
Resumen provisional:

El aprendizaje automático se ha desarrollado a pasos de gigante durante los últimos años y sus avances han permitido que los sistemas informáticos sean capaces de abstraer relaciones entre objetos y hacer predicciones sobre el futuro. Pero los grandes proyectos de aprendizaje requieren de la utilización de grandes cantidades de datos y los correspondientes potentes procesadores que sean capaces de trabajar con ellos. En el panorama actual cada vez más se vuelve una responsabilidad ser consciente de la huella causada en el medio ambiente por el consumo energético de los ordenadores.

Con este proyecto se pretende proporcionar una herramienta capaz de comparar entre los modelos más comunes empleados en aprendizaje automático, que pueda dar respuesta a la preocupación de equilibrar la precisión y eficacia de los modelos de aprendizaje con su impacto desde el punto de vista energético. Para ello se utilizarán las herramientas proporcionadas por Scikit-learn, Python y Codecarbon y conjuntos de datos representativos disponibles públicamente en la web con licencia abierta.


%%%%%%%%%%%%%%%%%%%%%%%%%%%%%%%%%%%%%%%%%%%%%%%%%%%%%%%%%%%%%%%%%%%%%%%%%%%%%%%%
%%%% Resumen en inglés

\chapter*{Summary}
%\addcontentsline{toc}{chapter}{Summary} % si queremos que aparezca en el índice
\markboth{SUMMARY}{SUMMARY} % encabezado

Here comes a translation of the ``Resumen'' into English. 
Please, double check it for correct grammar and spelling.
As it is the translation of the ``Resumen'', which is supposed to be written at the end, this as well should be filled out just before submitting.

%%%%--------------------------------------------------------------------
% Lista de comentarios de revisión
% Se puede borrar este bloque al acabar el borrador

\listoftodos
\markboth{TODO LIST}{TODO LIST} % encabezado
%%%%--------------------------------------------------------------------

%%%%%%%%%%%%%%%%%%%%%%%%%%%%%%%%%%%%%%%%%%%%%%%%%%%%%%%%%%%%%%%%%%%%%%%%%%%%%%%%
%%%%%%%%%%%%%%%%%%%%%%%%%%%%%%%%%%%%%%%%%%%%%%%%%%%%%%%%%%%%%%%%%%%%%%%%%%%%%%%%
% ÍNDICES %
%%%%%%%%%%%%%%%%%%%%%%%%%%%%%%%%%%%%%%%%%%%%%%%%%%%%%%%%%%%%%%%%%%%%%%%%%%%%%%%%

% Las buenas noticias es que los índices se generan automáticamente.
% Lo único que tienes que hacer es elegir cuáles quieren que se generen,
% y comentar/descomentar esa instrucción de LaTeX.

%%-- Índice de contenidos
\tableofcontents 
\cleardoublepage
%%-- Índice de figuras
%\addcontentsline{toc}{chapter}{Lista de figuras} % para que aparezca en el indice de contenidos
\listoffigures % indice de figuras
%\cleardoublepage
%%-- Índice de tablas
%\addcontentsline{toc}{chapter}{Lista de tablas} % para que aparezca en el indice de contenidos
%\listoftables % indice de tablas
\cleardoublepage
%%-- Índice de fragmentos de código
\listoflistings

%%%%%%%%%%%%%%%%%%%%%%%%%%%%%%%%%%%%%%%%%%%%%%%%%%%%%%%%%%%%%%%%%%%%%%%%%%%%%%%%
%%%%%%%%%%%%%%%%%%%%%%%%%%%%%%%%%%%%%%%%%%%%%%%%%%%%%%%%%%%%%%%%%%%%%%%%%%%%%%%%
% INTRODUCCIÓN %
%%%%%%%%%%%%%%%%%%%%%%%%%%%%%%%%%%%%%%%%%%%%%%%%%%%%%%%%%%%%%%%%%%%%%%%%%%%%%%%%

\include{files/1-introduccion.tex}

%%%%%%%%%%%%%%%%%%%%%%%%%%%%%%%%%%%%%%%%%%%%%%%%%%%%%%%%%%%%%%%%%%%%%%%%%%%%%%%%
%%%%%%%%%%%%%%%%%%%%%%%%%%%%%%%%%%%%%%%%%%%%%%%%%%%%%%%%%%%%%%%%%%%%%%%%%%%%%%%%
% ESTADO DEL ARTE %
%%%%%%%%%%%%%%%%%%%%%%%%%%%%%%%%%%%%%%%%%%%%%%%%%%%%%%%%%%%%%%%%%%%%%%%%%%%%%%%%

\chapter{Estado del arte}               %% a.k.a "Tecnologías utilizadas"
\label{chap:tecnologias}

\begin{comment}
Descripción de las tecnologías que utilizas en tu trabajo. 
Con dos o tres párrafos por cada tecnología, vale. 
Se supone que aquí viene todo lo que no has hecho tú.

Puedes citar libros, como el de Bonabeau et al., sobre procesos estigmérgicos~\cite{bonabeau:_swarm}. 
Me encantan los procesos estigmérgicos.
Deberías leer más sobre ellos.
Pero quizás no ahora, que tenemos que terminar la memoria para sacarnos por fin el título.
Nota que el \~ \ añade un espacio en blanco, pero no deja que exista un salto de línea. 
Imprescindible ponerlo para las citas.

Citar es importantísimo en textos científico-técnicos. 
Porque no partimos de cero.
Es más, partir de cero es de tontos; lo suyo es aprovecharse de lo ya existente para construir encima y hacer cosas más sofisticadas.
¿Dónde puedo encontrar textos científicos que referenciar?
Un buen sitio es Google Scholar\footnote{\url{http://scholar.google.com}}.
Por ejemplo, si buscas por ``stigmergy libre software'' para encontrar trabajo sobre software libre y el concepto de \emph{estigmergia} (¿te he comentado que me gusta el concepto de estigmergia ya?), encontrarás un artículo que escribí hace tiempo cuyo título es ``Self-organized development in libre software: a model based on the stigmergy concept''.
Si pulsas sobre las comillas dobles (entre la estrella y el ``citado por ...'', justo debajo del extracto del resumen del artículo, te saldrá una ventana emergente con cómo citar.
Abajo a la derecha, aparece un enlace BibTeX.
Púlsalo y encontrarás la referencia en formato BibTeX, tal que así:

\clearpage
{\footnotesize
\begin{minted}{bibtex}
@inproceedings{robles2005self,
  title={Self-organized development in libre software:
         a model based on the stigmergy concept},
  author={Robles, Gregorio and Merelo, Juan Juli\'an 
          and Gonz\'alez-Barahona, Jes\'us M.},
  booktitle={ProSim'05},
  year={2005}
}
\end{minted}
}

Copia el texto en BibTeX y pégalo en el fichero \texttt{memoria.bib}, que es donde están las referencias bibliográficas.
Para incluir la referencia en el texto de la memoria, deberás citarlo, como hemos hecho antes con~\cite{bonabeau:_swarm}, lo que pasa es que en vez de el identificador de la cita anterior (bonabeau:\_swarm), tendrás que poner el nuevo (robles2005self).
Compila el fichero \texttt{memoria.tex} (\texttt{pdflatex memoria.tex}), añade la bibliografía (\texttt{bibtex memoria.aux}) y vuelve a compilar \texttt{memoria.tex} (\texttt{pdflatex memoria.tex})\ldots y \emph{voilà} ¡tenemos una nueva cita~\cite{robles2005self}!

También existe la posibilidad de poner notas al pie de página, por ejemplo, una para indicarte que visite la página del GSyC\footnote{\url{http://gsyc.es}}.

\section{Sección 1} 
\label{sec:seccion1}

Hemos hablado de cómo incluir figuras, pero no se ha descrito cómo incluir tablas.
A continuación se presenta un ejemplo de tabla, la Tabla \ref{tabla:ejemplo} (fíjate 
en cómo se introduce una referencia a la tabla).

\begin{table}
 \begin{center}
  \begin{tabular}{ | l | c | r |} % tenemos tres colummnas, la primera alineada a la izquierda (l), la segunda al centro (c) y la tercera a la derecha (r). El | indica que entre las columnas habrá una línea separadora.
    \hline
    Uno & 2 & 3 \\ \hline % el hline nos da una línea vertical
    Cuatro & 5 & 6 \\ \hline
    Siete & 8 & 9 \\
    \hline
  \end{tabular}
  \caption{Ejemplo de tabla. Aquí viene una pequeña descripción (el \emph{caption}) del contenido de la tabla. Si la tabla no es autoexplicativa, siempre viene bien aclararla aquí.}
  \label{tabla:ejemplo}
 \end{center}
\end{table}
\end{comment}

%%%%%%%%%%%%%%%%%%%%%%%%%%%%%%%%%%%%%%%%%%%%%%%%%%%%%%%%%%%%%%%%%%%%%%%%%%%%%%%
%%%%%%%%%%%%%%%%%%%%%%%%%%%%%%%%%%%%%%%%%%%%%%%%%%%%%%%%%%%%%%%%%%%%%%%%%%%%%%%
%%%%%%%%%%%%%%%%%%%         START HERE         %%%%%%%%%%%%%%%%%%%%%%%%%%%%%%%%
%%%%%%%%%%%%%%%%%%%%%%%%%%%%%%%%%%%%%%%%%%%%%%%%%%%%%%%%%%%%%%%%%%%%%%%%%%%%%%%
%%%%%%%%%%%%%%%%%%%%%%%%%%%%%%%%%%%%%%%%%%%%%%%%%%%%%%%%%%%%%%%%%%%%%%%%%%%%%%%

\section{Entorno de desarrollo: PyCharm}
\label{sec:entorno_de_desarrollo}

%%-- El comando \gls{} permite incluir términos en el glosario, para luego reunirlos todos
%%-- en una tabla al comienzo o al final del documento, junto con sus definiciones.

PyCharm es un \gls{ide} dedicado concretamente a la programación en Python y desarrollado por la compañía checa JetBrains.

Proporciona análisis de código, un depurador gráfico, una consola de Python integrada, control de versiones y, además, soporta desarrollo web con Django. Todas estas características lo convierten en un entorno completo e intuitivo, idóneo para el desarrollo de proyectos académicos como el que nos ocupa.


\section{Redacción de la memoria: LaTeX/Overleaf}
\label{sec:redaccion_de_la_memoria}

LaTeX es un sistema de composición tipográfica de alta calidad que incluye características especialmente diseñadas para la producción de documentación técnica y científica. Estas características, entre las que se encuentran la posibilidad de incluir expresiones matemáticas, fragmentos de código, tablas y referencias, junto con el hecho de que se distribuya como software libre, han hecho que LaTeX se convierta en el estándar de facto para la redacción y publicación de artículos académicos, tesis y todo tipo de documentos científico-técnicos. 

Por su parte, Overleaf es un editor LaTeX colaborativo basado en la nube. Lanzado originalmente en 2012, fue creado por dos matemáticos que se inspiraron en su propia experiencia en el ámbito académico para crear una solución satisfactoria para la escritura científica colaborativa.

Además de por su perfil colaborativo, Overleaf destaca porque, pese a que en LaTeX el escritor utiliza texto plano en lugar de texto formateado (como ocurre en otros procesadores de texto como Microsoft Word, LibreOffice Writer y Apple Pages), éste puede visualizar en todo momento y paralelamente el texto formateado que resulta de la escritura del código fuente.

\section{Codecarbon}

CodeCarbon es un paquete creado con la intención de permitir a desarrolladores monitorizar las emisiones de dióxido de carbono ($CO_{2}$) producidas por aplicaciones en Inteligencia Artificial y modelos de Aprendizaje Automático, que surge de la motivación de contar con una forma de registrar las enormes cantidades de energía que el auge de la IA ha provocado en la industria. El incremento del rendimiento y la precisión de los modelos de Aprendizaje Automático que se ha producido en años recientes se ha logrado a cambio de la utilización de enormes cantidades de información para conseguir el aprendizaje de los patrones y características subyacentes. Así, los modelos más avanzados emplean cantidades significativas de poder computacional, entrenando en procesadores avanzados durante semanas o meses y consumiendo en el proceso una gran cantidad de energía. Dependiendo de la red eléctrica utilizada, este desarrollo puede comportar la emisión de grandes cantidades de gases de efecto invernadero como el $CO_{2}$.

CodeCarbon estima la huella de carbono de una aplicación medida como kilogramos de $CO_{2}$ equivalentes, o $CO_{2}eq$, una medida estandarizada utilizada para expresar la capacidad de calentamiento global de varios gases de efecto invernadero como la cantidad de $CO_{2}$ que causaría un impacto ambiental equivalente. Para tareas de computación, que emiten $CO_{2}$ por medio de la electricidad que están consumiendo y que es generada como parte de la red eléctrica (por ejemplo, mediante la quema de combustibles fósiles como el carbón) las emisiones de carbono se miden en kilogramos de $CO_{2}$ equivalentes por kilovatio-hora. De esta forma, las emisiones de dióxido de carbono totales se calculan como el producto de la intensidad de carbono de la electricidad utilizada para la computación y la energía consumida por la infraestructura.

La intensidad de carbono de la electricidad se calcula como la media ponderada de las emisiones de las distintas fuentes de energía usadas para generar electricidad, incluyendo combustibles fósiles y renovables. En la herramienta se asigna un valor conocido de dióxido de carbono emitido por kilovatio-hora generado para cada uno de los combustibles (carbón, petróleo y gas natural). Otras fuentes renovables o consideradas como de bajo carbono incluyen la energía solar, hidroeléctrica, biomasa o geotérmica. La intensidad de carbono de cada combustible individual se calcula en base a medidas de generación de carbono y electricidad en los Estados Unidos, y aplicadas de forma generalizada en el resto del mundo. Cada red eléctrica local incluye una mezcla distinta de fuentes de energía y tiene asignada entonces una intensidad de carbono total particular.

% IMAGE: Global distribution of carbon intensity (carbonboard)
% TABLE: Carbon intensity by energy source (Codecarbon/Methodology)
% CITE: CodeCarbon documentation

% _opcional_ : explanation on zero value for low-carbon fuels
% _opcional_ : explanation on power consuption calculation by CPU

\section{Scikit-Learn}

Scikit-learn es un módulo desarrollado para Python que integra un amplio rango de algoritmos de aprendizaje automático de última generación para problemas tanto supervisados como no supervisados. Este paquete pretende llevar el aprendizaje automático a desarrolladores no especialistas mediante el uso de un lenguaje generalista de alto nivel. Se hace hincapié en la facilidad de uso, el rendimiento, la documentación y la consistencia de la API. Tiene las mínimas dependencias necesarias y está distribuido bajo la licencia BSD, con el objetivo de incentivar su uso tanto en ambientes educativos como comerciales.

Scikit-learn expone una gran variedad de algoritmos de aprendizaje utilizando una interfaz consistente y orientada a la resolución de tareas, lo que permite una comparación sencilla entre distintos métodos de aprendizaje para una misma aplicación. Al depender del ecosistema científico de Python, puede ser integrado con facilidad en aplicaciones que se salgan del rango tradicional del análisis estadístico de datos. Además, los algoritmos, que han sido implementados en un lenguaje de alto nivel, pueden ser utilizados como bloques de construcción para desarrollar estrategias más complejas que se adecuen a cada caso particular.

% CITE: Scickit-learn/About

\section{Datos utilizados}

Todos los conjuntos de datos utilizados en los análisis realizados están disponibles de forma libre en la web y proceden de dos fuentes: el repositorio de aprendizaje automático de la Universidad de California Irvine y el proyecto Galaxy Zoo de clasificación de galaxias.

El repositorio de aprendizaje automático de la UCI es una colección de bases de datos, teorías de dominios y generadores de datos que son utilizados por la comunidad de aprendizaje automático para el análisis empírico de algoritmos de aprendizaje automático. El archivo fue creado en 1987 como un servidor FTP por David Aha y otros compañeros estudiantes en la UCI. Desde entonces ha sido ampliamente utilizado por estudiantes, educadores e investigadores de todo el mundo como una fuente primaria de conjuntos de datos para aprendizaje automático \cite{ml-uci}.

A continuación se describen las características más importantes de los conjuntos empleados en orden ascendente de tamaño.
% ANNEX All attributes

\subsection{Hepatitis}

Se trata de un conjunto de datos de pequeño tamaño (155 instancias) que se puede encontrar en el repositorio de la UCI\footnote{\url{https://archive.ics.uci.edu/ml/datasets/Hepatitis}}. Los datos proceden de pacientes diagnosticados de hepatitis y se clasifican de acuerdo a su supervivencia. Incluye 19 atributos sobre la historia del paciente, en los que la mayor parte (13) son atributos categóricos con dos posible valores y el resto son valores discretizados. Los datos fueron donados por Gail Gong en Noviembre de 1988. Trabajos pasados con este conjunto de datos obtuvieron medidas de precisión del 80\% \cite{hepatitis-gong}.

\subsection{Ionosfera}

Se trata de un conjunto de datos con 351 instancias procedente del repositorio de la UCI\footnote{\url{https://archive.ics.uci.edu/ml/datasets/Ionosphere}}. Contiene datos de radar obtenidos por el grupo de física espacial de la Universidad John Hopkins y donados por Vince Sigillito en 1989. El sistema radar está ubicado en Goose Bay, Labrador y consiste en un array de 16 antenas de alta frecuencia. El objetivo es la medición de electrones libres en la ionosfera y su clasificación binaria entre "buenas" respuestas del radar que indican evidencia de algún tipo de estructura en la ionosfera y "malas" respuestas en las que las señales simplemente pasan a través de la ionosfera. Las señales recibidas se procesaron utilizando una función de autocorrelación con el tiempo de pulso y el número de pulso como argumentos y cada una de las instancias del conjunto de datos está descrita por dos atributos continuos para cada uno de los 17 números de pulso, correspondientes al valor complejo obtenido de la señal electromagnética compleja. Hay por lo tanto un total de 34 características continuas por instancia.

\subsection{Billetes}

Este conjunto de datos contiene información extraída de 1372 imágenes tomadas para evaluar un procedimiento de autenticación de billetes. Fue donado en Agosto de 2012 por Volker Lohweg de la Universidad de Ciencias Aplicadas de Ostwestfalen-Lippe, Alemania, al repositorio de la UCI\footnote{\url{https://archive.ics.uci.edu/ml/datasets/banknote+authentication}}. Para la digitalización de las imágenes tomadas se empleó una cámara industrial normalmente utilizada para la inspección de impresiones. Las imágenes finales tienen un tamaño de 400x400 píxeles con una resolución de alrededor de 660 dpi en escala de grises. Posteriormente, se empleó una herramienta de transformada ondícula para extraer cuatro características continuas de las imágenes.

\subsection{Adultos}

Se trata de un conjunto de datos de mediano tamaño con casi 50.000 instancias. Contiene datos del censo poblacional de 1994 de los Estados Unidos, recogidos en Mayo de 1996 y donados al repositorio UCI\footnote{\url{https://archive.ics.uci.edu/ml/datasets/Adult}} por Ronny Kohavi y Barry Becker, de la sección de análisis de datos y visualización de Silicon Graphics. Cuenta con hasta 14 características categóricas y continuas extraídas del censo como edad, raza, nivel de educación o estado ocupacional. El objetivo de clasificación es predecir si los ingresos de la persona sobrepasan los 50 mil dolares al año.

\subsection{Galaxy Zoo}

\cite{galaxy-zoo}

%\cleardoublepage

%%%%%%%%%%%%%%%%%%%%%%%%%%%%%%%%%%%%%%%%%%%%%%%%%%%%%%%%%%%%%%%%%%%%%%%%%%%%%%%%
%%%%%%%%%%%%%%%%%%%%%%%%%%%%%%%%%%%%%%%%%%%%%%%%%%%%%%%%%%%%%%%%%%%%%%%%%%%%%%%%
% DISEÑO E IMPLEMENTACIÓN %
%%%%%%%%%%%%%%%%%%%%%%%%%%%%%%%%%%%%%%%%%%%%%%%%%%%%%%%%%%%%%%%%%%%%%%%%%%%%%%%%

\chapter{Diseño e implementación}
\label{chap:diseño}

Este capítulo proporciona una visión detallada sobre la metodología y las herramientas utilizadas para medir el consumo energético y evaluar el rendimiento de modelos de aprendizaje automático en el contexto de la aplicación desarrollada: \texttt{MLCost}. Se comenzará describiendo la arquitectura de la aplicación y los módulos que la componen, para posteriormente profundizar en la metodología del proceso de aprendizaje. Este proceso empezará con la preparación de los datos, donde destacan las técnicas de preprocesamiento y la selección de modelos representativos de diversas familias algorítmicas. 

Durante la fase de entrenamiento, se empleará la biblioteca CodeCarbon para medir las emisiones de carbono y el consumo de energía, utilizando técnicas como la validación cruzada para evaluar la precisión y el rendimiento de los modelos. La evaluación de las predicciones se realiza a través de métricas estándar como precisión, exhaustividad y F-score, para obtener una estimación robusta del desempeño del modelo. Para finalizar, se presentarán los resultados con distintas herramientas de visualización y se discutirá la gestión eficaz de recursos, enfocándose en la configuración del procesador y el uso de múltiples núcleos para optimizar el rendimiento y minimizar el sesgo en las mediciones de consumo energético.


\section{Arquitectura}

\begin{figure}[H]
  \centerline{
     \includegraphics[width=\textwidth, keepaspectratio]{img/general-arch.jpg}
  }
  \caption{Arquitectura de la aplicación desarrollada}
  \label{fig:app-arch}
\end{figure}

La arquitectura de la aplicación se organiza en torno a varios componentes clave que se integran en el paquete MLCost para ofrecer un entorno robusto para la evaluación de modelos de aprendizaje automático, tal y como muestra la figura~\ref{fig:app-arch}.

El punto de entrada de la aplicación es el módulo \texttt{cli}, que se encarga de gestionar la interfaz de línea de comandos (\emph{command-line interface}, CLI) que facilita la interacción del usuario con las funcionalidades principales de la aplicación. Este módulo permite a los usuarios ejecutar la aplicación desde la terminal, proporcionando una manera flexible y accesible de realizar diversas operaciones, como la selección de modelos de aprendizaje automático, la configuración de parámetros de ejecución y la especificación de archivos de datos de entrada. El módulo \texttt{cli} utiliza la librería \texttt{click} para facilitar el procesamiento de los argumentos de la línea de comandos. Este enfoque permite definir una variedad de opciones y argumentos que los usuarios pueden especificar al ejecutar la aplicación. Por ejemplo, los usuarios pueden seleccionar que opciones de limpieza de datos serán utilizadas, establecer el número iteraciones para la validación cruzada, y decidir si serán ejecutadas en paralelo en el procesador. El conjunto completo de opciones de línea de comandos que se pueden utilizar está recogido en el Anexo~\ref{app:cli}. Una vez que se capturan los argumentos, \texttt{cli} invoca las funciones correspondientes definidas en el modulo \texttt{mlcost}.

Este modulo se encarga de la integración del seguimiento de emisiones de carbono y consumo energético durante el entrenamiento y la evaluación de los modelos de aprendizaje automático. Este módulo utiliza la biblioteca codecarbon para realizar el seguimiento del impacto ambiental de estos procesos computacionales. El módulo es responsable de gestionar el proceso de entrenamiento, creando un objeto de la clase \texttt{Trainer} por cada modelo a evaluar y asegurándose de que los datos son preprocesados para mejorar la precisión del modelo. Una vez que los datos están preparados, el módulo comienza la medición de emisiones, encarga el entrenamiento del modelo y, al finalizar, detiene el rastreador de emisiones y recupera los datos finales de consumo. Estos datos incluyen la duración del seguimiento, el consumo energético y las emisiones equivalentes de carbono. Los resultados, junto con los datos de rendimiento obtenidos durante la validación del modelo, se registran en un archivo para posterior referencia.

El núcleo de la aplicación reside en el modulo \texttt{learn}, que expone la clase \texttt{Trainer} ya mencionada, donde se definen las tareas principales de carga de datos, entrenamiento y evaluación de modelos, así como la recopilación de métricas de desempeño. Estos procesos serán descritos en mayor detalle en las secciones posteriores.

Para finalizar, la aplicación contiene dos módulos auxiliares que facilitan el desarrollo y proporcionan utilidades para examinar los resultados obtenidos.

El modulo \texttt{graphs} maneja la visualización de los resultados mediante diversas funciones de creación de gráficas que utilizan la librería matplotlib para crear gráficos de dispersión, de líneas y de barras. Estas visualizaciones ayudan a interpretar las relaciones entre las emisiones, el consumo de energía y las métricas de rendimiento de los modelos evaluados. Las gráficas también permiten comparar el desempeño de diferentes modelos bajo diversas cargas de CPU, ofreciendo una visión clara de cómo los recursos del sistema afectan la eficiencia y la precisión del modelo.

El componente \texttt{utils} proporciona funciones auxiliares para la aplicación, tales como la impresión de resultados y la recopilación de información del sistema operativo, así como la gestión de archivos de salida en formato CSV. Este archivo incluye utilidades para manejar los datos de emisiones y consumo energético, formateando la información de manera que sea fácilmente interpretable. 


\section{Lectura y limpieza de los datos}
\label{sec:limpieza}

El componente más importante de todo proyecto de aprendizaje automático son sin duda los datos. Por este motivo se han desarrollado con el tiempo una gran cantidad de librerías para facilitar la tarea de los desarrolladores que quieren trabajar con información de forma estructurada. En Python una de las más importantes es la librería \texttt{pandas} (\ref{subsec:pandas}), que ofrece la clase \texttt{DataFrame} con la que se pueden procesar grandes cantidades de datos de forma similar a como se trabajaría con una tabla o hoja de cálculo, con filas y columnas con distintos tipos de información.

Los conjuntos de datos que están disponibles de forma libre en repositorios de universidades y otras organizaciones de ciencia de datos no siempre siguen un mismo formato. Por este motivo el primer paso para aplicar métodos de aprendizaje automático en datos específicos será siempre deshacerse del formato de presentación y guardarlos en estructuras de datos operables por un ordenador, como \texttt{DataFrames} de \texttt{pandas}. En este proyecto, la intención ha sido ser capaz de procesar datos presentados con varios formatos distintos. Para ello, se han creado distintas argumentos para la línea de comandos que pueden ser utilizados al lanzar la aplicación especificando las características concretas del conjunto de datos a tratar.

En general, el proceso de lectura y limpieza de los datos va seguir siempre las siguientes fases:
\begin{enumerate}
    \item Lectura
    \begin{enumerate}
        \item Leer el archivo que contiene los datos.
        \item Separar los datos de entrenamiento de los datos de testeo.
        \item Identificar el tipo de datos de cada columna.
    \end{enumerate}
    \item Limpieza
    \begin{enumerate}
        \item Eliminar columnas que no pueden ser utilizadas.
        \item Reemplazar valores numéricos que falten.
        \item Reemplazar columnas categóricas por columnas booleanas.
        \item Escalar las características numéricas.
    \end{enumerate}
\end{enumerate}

El primer paso es leer los datos de uno o varios archivos, que serán generalmente archivos de texto en formato \texttt{.txt} o \texttt{.csv}. Es en este paso en el que se dan mayores diferencias entre distintos conjuntos de datos y la razón de que se hayan añadido las distintas opciones de línea de comandos al programa para resolverlo. Los archivos de texto que contienen los datos pueden utilizar diferentes caracteres separadores entre columnas (como coma o espacio), representar valores que no han sido tomados con diferentes símbolos (como '?' o '-'), presentar o no una fila inicial con los nombres de las columnas, identificar la columna de las etiquetas de distintas maneras e incluso separar en distintos archivos los conjuntos de entrenamiento y de testeo de forma previa. Todas estas opciones son tenidas en cuenta durante la lectura para convertir los archivos de texto en \texttt{dataframes} sobre los que las librerías de aprendizaje pueden operar. En el Anexo~\ref{app:cli} se incluye un compendio de todas las opciones disponibles en la aplicación.

Una vez que los datos están recogidos, el siguiente paso es separar los datos de entrenamiento de los de testeo. Para ello, primero se descartarán todas las filas de datos que no estén etiquetadas, si las hubiera. Por defecto, la separación se hace al 80-20 y de forma aleatoria, excepto si los conjuntos están previamente separados en dos archivos. Para terminar el proceso de lectura, las columnas que contienen características numéricas se separan de las que contienen características categóricas, para poder tratarlas de forma específica durante la limpieza.

En la sección de limpieza, el objetivo es eliminar las características que puedan crear obstáculos en el entrenamiento de los modelos. Tres problemas básicos son tratados: falta de datos en columnas numéricas, datos presentados de forma categórica y diferentes escalas de los datos numéricos. Para lidiar con ello se utilizará un tipo de clases provistas por la librería \texttt{scikit-learn} denominadas transformadores. Estos transformadores encapsulan distintas herramientas de preprocesado y limpieza que son usadas a menudo. Para la falta de datos numéricos, se utilizara un introductor simple de medidas (\texttt{SimpleImputer}), que rellenará los datos que falten con la media de los datos disponibles. 

Respecto a las columnas categóricas, muchos algoritmos de aprendizaje no están diseñados para trabajar directamente con variables no numéricas (generalmente, porque limitaría la eficiencia de los algoritmos). Para resolverlo, se utilizará un transformador denominado \texttt{OneHotEncoder}. Este transformador reemplaza una característica categórica con varias características booleanas, de forma que para cada posible valor de la categoría se crea una nueva columna con un valor de sí o no dependiendo de a cual pertenece cada dato. Para simplificar, características categóricas con más de diez valores distintos posibles serán descartadas completamente. Lo mismo ocurrirá con cualquier otra columna que no pueda ser identificada como numérica o categórica y con las filas en las que falten datos de tipo categórico.

Un último transformador, \texttt{StandardScaler}, será aplicado a las características numéricas para alinear la escala de todas ellas mediante una técnica denominada normalización de características. Este proceso consiste en transformar las características numéricas para que se sitúen dentro de un rango común, generalmente entre 0 y 1 o para que tengan media cero y varianza uno, como se hace con el \texttt{StandardScaler}. Esto ayuda a obtener mejores resultados en los modelos de aprendizaje automático por dos razones. Primero, evita que características con valores grandes dominen a aquellas con valores más pequeños, asegurando que todas las características contribuyan de manera equitativa al modelo. Segundo, algunos algoritmos, como los basados en distancias (por ejemplo, vecinos más cercanos o máquinas de vector soporte), funcionan mejor y convergen más rápido cuando las características están en la misma escala.


\section{Entrenamiento}

Una vez que los datos están preparados para su uso comienza la fase de entrenamiento. En este proyecto, el objetivo es medir el gasto energético de distintos modelos y compararlo con la calidad de sus predicciones. Para ello, se han elegido una serie de modelos representativos de las familias de algoritmos más utilizadas. Estos modelos elegidos serán entrenados uno detrás de otro con el conjunto de datos preparado mientras se mide el consumo energético mediante las herramientas proporcionadas por CodeCarbon.

El procedimiento es sencillo. En primer lugar se comienzan las mediciones mediante la creación se un objeto de tipo \texttt{EmissionsTracker} que cuenta con simples métodos \texttt{start()} y \texttt{stop()}. A continuación, se entrena el modelo en la parte del conjunto de datos reservada para entrenamiento, y seguidamente se aplica el modelo entrenado a la parte del conjunto de datos reservada para testeo para intentar predecir correctamente la etiqueta de cada entrada que contiene. Para finalizar, se detiene la medición de emisiones y se almacenan los resultados obtenidos.


\subsection{Modelos escogidos}
\label{subsec:models-short}

Dentro de los modelos de aprendizaje automático existen numerosas clasificaciones de acuerdo al tipo de tareas a realizar y la naturaleza de los datos. Este proyecto se centrará en tareas de clasificación por aprendizaje supervisado. En este subgrupo, destacan una serie de familias de algoritmos que suelen obtener buenos resultados para una gran variedad de tipos de datos.
\begin{enumerate}
    \item Modelos lineales. Se trata de modelos sencillos, fáciles de interpretar y eficientes computacionalmente, que funcionan bien cuando las relaciones entre las características de entrada y la salida son aproximadamente lineales. Uno de sus modelos más representativos es el de regresión logística (ver \ref{subsec:model-linear}).
    \item Árboles decisores. Estos algoritmos pueden modelar relaciones complejas en los datos y lidiar con no linealidad. Dentro de esta familia destaca el modelo de Bosque Aleatorio, que agrega predicciones de varios árboles decisores para mejorar la robustez del modelo (ver \ref{subsec:model-random-forest}).
    \item Máquinas de vector soporte (\ref{subsec:model-svm}). Estos modelos son efectivos tanto en tareas de clasificación lineales como no lineales y destacan por su gran versatilidad. Funcionan de forma óptima en conjuntos de datos relativamente pequeños pero de gran complejidad.
    \item Vecinos más cercanos. Su máximo representante, k vecinos más cercanos (k-NN, ver \ref{subsec:model-neighbors}), es un algoritmo simple e intuitivo que se basa en buscar relaciones locales entre los datos y puede ser efectivo en tareas tanto de regresión como de clasificación.
    \item Naive Bayes (bayesiano ingenuo). Se trata de una familia de modelos que destaca por su gran eficiencia, especialmente frente a conjunto de datos de gran complejidad, y especialmente útil en tareas de clasificación de texto. El clasificador más representativo es el Naive Bayes gaussiano (\ref{subsec:model-naive-bayes}).
    \item Métodos de conjuntos (ensemble). Estos métodos combinan las características de múltiples modelos para intentar mejorar el desempeño total. Entre ellos destacan las máquinas de potenciación de gradiente, que construyen sólidos modelos predictivos de forma iterativa (ver \ref{subsec:model-gradient}).
    \item Redes neuronales. Las redes neuronales, especialmente los modelos de aprendizaje profundo como las redes neuronales profundas (Deep Neural Networks, DNN, ver \ref{subsec:model-neural}), pueden formar representaciones jerárquicas complejas a partir de los datos, y son especialmente efectivas en tareas que involucran grandes cantidades de datos con patrones complejos.
\end{enumerate}

En general, se espera que modelos de aprendizaje más complejos como los métodos de conjuntos, las redes neuronales y las máquinas de vector soporte produzcan mejores predicciones a cambio de un mayor gasto energético que otros modelos comparativamente más sencillos computacionalmente, como los modelos lineales, Naive Bayes y vecinos más cercanos. En el capitulo~\ref{chap:experimentos}, se analizará si los resultados obtenidos durante los experimentos se corresponden con esta aproximación teórica.


\section{Registro de resultados}

\subsection{Evaluación de las predicciones}
\label{sec:scoring}

Para poder obtener una medida de utilidad de los distintos modelos, es necesario evaluar la calidad de las predicciones que realizan. Existen dos métodos principales realizar esta evaluación. El primero y más sencillo consiste en aplicar la función de predicción del modelo a un subconjunto de muestras que hayan sido aisladas previamente para no formar parte del proceso de entrenamiento. Estas predicciones se comparan con las etiquetas correctas de las muestras para determinar si cada predicción ha sido acertada.
Cuatro resultados distintos son posibles por muestra y clase concreta: verdadero positivo (TP, identificada correctamente como perteneciente a la clase), falso positivo (FP, identificada incorrectamente como perteneciente a la clase), falso negativo (FN, identificada incorrectamente como no perteneciente a la clase), y verdadero negativo (TN, identificada correctamente como no perteneciente a la clase). Una forma común de visualizar estos resultados es mediante una matriz de confusión, como la mostrada en la figura~\ref{fig:confusion-matrix}.

\begin{figure}[H]
  \centerline{
     \includegraphics[width=0.8\textwidth, keepaspectratio]{img/confusion-matrix.jpg}
  }
  \caption{Matriz de confusión que muestra las relaciones entre posible resultados de la predicción.}
  \label{fig:confusion-matrix}
\end{figure}

A partir de las relaciones entre el número de muestras en cada uno de estos grupos se pueden extraer varias métricas del modelo, siendo las más comunes la exactitud, la precisión, la exhaustividad y el \emph{f-score}\cite{scikit-model-eval}. Estas métricas dan lugar a un valor entre 0 y 1, donde 0 es el peor resultado y 1 el mejor. También son comúnmente expresadas en porcentaje.

La \textbf{exactitud} (\emph{accuracy}) se define como la cercanía de la predicciones a su valor real. En tareas de clasificación se calcula como el número de predicciones correctas entre el número de muestras totales, como se puede ver en la ecuación \ref{eq:accuracy}. Una variante interesante de la exactitud que \texttt{scikit-learn} permite calcular es la exactitud balanceada, que evita medidas infladas de exactitud en conjuntos de datos no balanceados (con una o más clases sobrerrepresentadas en el conjunto) mediante la ponderación de cada muestra de acuerdo a la prevalencia inversa de su verdadera clase.

\begin{equation}
    a = \dfrac{TP+TN}{\text{muestras totales}}
\label{eq:accuracy}
\end{equation}

La \textbf{precisión} (\emph{precision}) y la \textbf{exhaustividad} (\emph{recall}) son métricas que analizan la relevancia de las muestras asignadas a cada clase y se calculan individualmente por clase. La precisión analiza el número de muestras correctamente clasificadas dentro de todas las muestras asignadas a una clase concreta, como muestra la ecuación \ref{eq:precision}, mientras que la exhaustividad analiza el número de muestras correctamente clasificadas en relación al número total de muestras reales existentes, como muestra la ecuación \ref{eq:recall}. Estas dos métricas pueden ser promediadas en función del peso relativo de cada clase un el conjunto para obtener una medida global de la precisión y exhaustividad del modelo.

\noindent
\begin{tabular}{@{}p{.4\linewidth}@{}p{.6\linewidth}@{}}
  \begin{equation}
     p = \dfrac{TP}{TP + FP}
  \label{eq:precision}
  \end{equation}
  &
  \begin{equation}
    r = \dfrac{TP}{TP + FN}
  \label{eq:recall}
  \end{equation}
\end{tabular}

En último lugar, es interesante mencionar el valor-F (\emph{F-score}), que se puede interpretar como una media harmónica de la precisión y la exhaustividad y tiene distintas variantes dependiendo de la importancia relativa de estas dos medidas. La denominada medida-$F_1$ da la misma importancia a la precisión y a la exhaustividad. Para cada clase, se calcula como muestra la ecuación \ref{eq:f-score}.

\begin{equation}
    F_1 = \dfrac{2\cdot TP}{2TP + FP + FN} = \dfrac{2pr}{p+r}
\label{eq:f-score}
\end{equation}

Estas cuatro medidas son calculadas por defecto en la aplicación desarrollada para todos los modelos entrenados. Para todos los casos, se utiliza la opción de scikit-learn \texttt{average='weighted'} para obtener las medidas, de forma que se puedan obtener valores más realistas en conjuntos con clases no balanceadas.

\subsubsection{Validación cruzada}

La aplicación desarrollada ofrece un segundo método para evaluar el rendimiento de un modelo mediante validación cruzada. Este método se basa en las mismas métricas ya mencionadas, pero con la peculiaridad de que éste se entrena varias veces de forma sucesiva con distintas distribuciones de los datos en un conjunto de entrenamiento y un conjunto de prueba. Posteriormente, se puede analizar la media y la desviación estándar de las métricas de la calidad de las predicciones obtenidas para cada distribución de las muestras y así obtener una idea más exacta del desempeño del modelo. En \texttt{scikit-learn}, la validación cruzada se puede implementar mediante el uso de la clase \texttt{KFold}, que divide el conjunto de datos en un número de pliegues especificados de forma aleatoria, y la función \texttt{cross\_validate}, que entrena el modelo y calcula las métricas deseadas para cada uno de los pliegues de forma sucesiva. Para determinar los plieges, la aplicación MLCost utiliza una clase derivada llamada \texttt{StratifiedKFold}, que ayuda a mantener el mismo ratio de muestras por clase en cada pliegue para conjuntos de datos no balanceados.

La validación cruzada ofrece varias ventajas significativas al evaluar el rendimiento de un modelo de aprendizaje automático. Una de las principales ventajas es que proporciona una estimación más robusta y fiable del desempeño del modelo al utilizar diferentes subconjuntos del conjunto de datos para entrenamiento y prueba en cada iteración. Esto ayuda a mitigar el riesgo de sobreajuste y ofrece una visión más generalizada de cómo se comportará el modelo con datos no vistos. Además, calcular la media y la desviación estándar de las métricas a través de los distintos pliegues permite una evaluación más precisa y detallada, identificando variaciones y asegurando que el modelo no solo es preciso sino también consistente.

Sin embargo, la validación cruzada también tiene desventajas. Uno de los principales inconvenientes es el aumento significativo en el tiempo de cómputo, ya que el modelo debe entrenarse y evaluarse múltiples veces, lo cual puede ser particularmente costoso en términos de tiempo y consumo energético, especialmente para conjuntos de datos grandes o modelos complejos. La utilización de un entorno de paralelización puede mitigar esta desventaja al permitir que los pliegues se procesen en paralelo, reduciendo así el tiempo total necesario para completar la validación cruzada. No obstante, la paralelización puede no ser igualmente efectiva en todos los tipos de hardware. Además, en entornos compartidos o con recursos limitados, la paralelización podría causar conflictos de recursos, afectando negativamente el rendimiento global del sistema. Por lo tanto, aunque la paralelización puede acelerar significativamente la validación cruzada, es crucial evaluar su uso en conjunción con las capacidades y limitaciones del hardware disponible. Este efecto será examinado durante las pruebas llevadas a cabo en la sección~\ref{sec:test-2-resources}.


\subsection{Medición de emisiones}

Durante la ejecución de su rastreador de emisiones, la librería CodeCarbon calcula varias medidas distintas para identificar el consumo energético. En primer lugar, las emisiones están geolocalizadas de una de las siguientes formas: mediante una conexión a internet automática que permita identificar la localización mediante rastreo de IP, o mediante la especificación de un país determinado en el código al crear el objeto rastreador de emisiones. Esta localización es necesaria para convertir los kilovatios consumidos durante el proceso de entrenamiento en emisiones de carbono equivalentes, que dependerán de la mezcla especifica de producción de energía que haya establecido cada país. De esta forma, si la energía estuviera producida en gran medida por energías renovables, las emisiones de carbono serían mucho menores que si la energía fuera producida en su totalidad en una planta de quema de carbón. En la sección de resultados se compararán las diferencias de emisiones producidas entrenando modelos en un mismo conjunto de datos en diferentes localizaciones.

Con estos datos, la librería CodeCarbon calcula las emisiones a partir de medidas de la capacidad del procesador y gráfica de la máquina, del porcentaje de su uso que corresponde al proceso de entrenamiento observado y de la duración total del proceso. En este proyecto, por cada modelo entrenado se guardan tres de estas medidas para su posterior análisis y comparativa: la energía consumida (en kilovatios hora, \unit{kWh}), las emisiones calculadas (en kilogramos equivalentes de carbono, $\unit{kg\;[CO_2eq]}$]) y la duración (en segundos). Durante el proceso de entrenamiento, estos valores son escritos en un archivo de texto de tipo CSV (\emph{comma-separated values}) junto con las medidas de calificación de las predicciones mencionadas en el apartado anterior. Este archivo de texto será utilizado posteriormente para dibujar gráficas de las que extraer conclusiones con una herramienta de gráficos desarrollada para este proyecto.


\subsection{Herramientas de visualización}

La visualización de los resultados obtenidos es crucial para interpretar y analizar las métricas de consumo energético y rendimiento de los modelos de aprendizaje automático evaluados. A través de gráficos, es posible identificar patrones, tendencias y relaciones entre distintas variables que de otra manera serían difíciles de detectar en tablas de datos. Esto facilita la toma de decisiones informadas y permite comunicar los hallazgos de manera más efectiva.

El motor de gráficos utilizado en este proyecto es Matplotlib (ver \ref{subsec:matplotlib}, una biblioteca de Python ampliamente utilizada para la creación de gráficos estáticos, animados e interactivos. Matplotlib proporciona una gran flexibilidad y control sobre la generación de gráficos, permitiendo a los desarrolladores personalizar todos los aspectos visuales de sus representaciones gráficas.

Para generar los gráficos, es necesario haber ejecutado previamente una medición de emisiones con la opción \texttt{--log} activada. Esto crea un archivo CSV que contiene los datos recopilados durante las pruebas, incluyendo métricas de rendimiento y consumo energético. Este archivo CSV es esencial para la visualización, ya que contiene toda la información requerida para producir los gráficos. Los gráficos se generan utilizando el comando \texttt{mlcost show -f <output-file.csv>}. Este comando lee el archivo CSV especificado y produce una variedad de gráficos que ayudan a visualizar los resultados de las pruebas.

El módulo \texttt{graph} incluye varias funciones auxiliares para la creación de diferentes tipos de gráficos. Entre ellas se encuentran funciones para generar gráficos de dispersión con tres o cuatro variables, así como gráficos de líneas y barras. Las funciones de dispersión permiten comparar variables como emisiones y precisión a través de diferentes modelos y conjuntos de datos, utilizando diferentes marcadores y colores para distinguir entre categorías. Por otro lado, los gráficos de líneas y barras se utilizan para visualizar tendencias y comparaciones categóricas, aprovechando las capacidades de la librería Pandas para agrupar los datos recogidos en \texttt{DataFrames} y producir gráficas con ellos de manera eficiente.

\section{Gestión de recursos}

Un aspecto importante para obtener mediciones precisas del consumo energético y del rendimiento de los modelos de aprendizaje automático es la gestión de los recursos disponibles en la máquina que está tomando las medidas. Los resultados obtenidos por la librería CodeCarbon para estimar el consumo eléctrico se basan en el modelo y el tiempo de utilización del procesador. De esta forma, CodeCarbon rastrea el uso de recursos durante el entrenamiento de los modelos y calcula las emisiones de $CO_2$ correspondientes.

La carga del procesador en el momento de tomar las medidas de consumo energético es un factor crucial. Una alta carga del procesador puede indicar que el sistema está ejecutando múltiples tareas simultáneamente, lo cual puede sesgar los resultados. Para mitigar este efecto, CodeCarbon intenta aislar el proceso de aprendizaje del resto de tareas del ordenador al utilizar la opción {\texttt{tracking\_mode="process"} en su clase medidora de emisiones \texttt{EmissionTracker}. Esta opción permite que la herramienta enfoque la toma de medidas en el proceso específico que se está evaluando, minimizando la interferencia de otras operaciones del sistema. 

Sin embargo, este enfoque añade una capa adicional de aproximación al cálculo de emisiones, por lo que para obtener medidas más consistentes será interesante mantener una carga baja del procesador al comenzar el proceso. Al inicio de la aplicación, ésta imprime al terminal tanto la información del sistema y del modelo del procesador como su carga de trabajo inicial en porcentaje de utilización. Además, el archivo de datos recopilados generado por el programa contiene una columna con la carga al finalizar el entrenamiento de cada modelo. Una carga baja al comienzo del experimento significará que la máquina no está realizando otras tareas pesadas que podrían influir en las mediciones de consumo energético y rendimiento. Esto ayudará a que los datos reflejen de manera más precisa el impacto del entrenamiento del modelo en el consumo energético.


\subsection{Procesamiento multinúcleo}

El uso de ordenadores con varios núcleos en el procesador o de varios procesadores asignados a la misma tarea puede afectar tanto al rendimiento como a las emisiones. En términos de rendimiento, la capacidad de realizar múltiples tareas simultáneamente (\emph{multitasking}) permite que los modelos se entrenen más rápido, ya que pueden aprovechar el paralelismo inherente a muchos de los algoritmos de aprendizaje automático. Sin embargo, esta mejora en el rendimiento puede venir acompañada de un aumento en el consumo energético, ya que más núcleos en uso implican un mayor consumo de energía.

La biblioteca scikit-learn gestiona el paralelismo a través del parámetro \texttt{n\_jobs}, el cual se puede especificar en diversas funciones y modelos para indicar el número de procesos paralelos que se deben utilizar. La aplicación desarrollada acepta este parámetro al ejecutarla por línea de comandos y lo comunica a scikit-learn, permitiendo que se configure la cantidad de CPUs disponibles para ejecutar tareas en paralelo. Esta configuración puede reducir significativamente el tiempo de cómputo en experimentos de gran escala.

Internamente, scikit-learn utiliza el contexto {joblib.parallel\_backend} para gestionar el paralelismo. {joblib.parallel\_backend} es una función que permite seleccionar el backend de paralelización que se utilizará durante la ejecución de las operaciones paralelas. El backend de joblib puede manejar diferentes tipos de paralelismo, como threading y multiprocessing, adaptándose a las características del hardware y a las necesidades del usuario. Cuando se especifica \texttt{n\_jobs}, {joblib.parallel\_backend} se encarga de dividir las tareas entre los procesos disponibles y de gestionar la sincronización y la recolección de resultados.

Dentro del proceso de aprendizaje, hay varias funciones en las que scikit-learn ofrece la posibilidad de ejecutar tareas en paralelo. Una de ellas es la validación cruzada, que es particularmente adecuada para la paralelización, ya que cada pliegue del conjunto de datos puede ser procesado de manera independiente. Por otra parte, varios modelos pueden ser entrenados directamente en paralelo dentro de una única iteración especificando el número de procesos a utilizar. Entre los modelos que admiten \texttt{n\_jobs} se encuentran el bosque aleatorio, las máquinas de potenciación de gradiente, y vecinos más cercanos. Sin embargo, otros modelos como los lineales, las máquinas de vector soporte, Naive Bayes, y las redes neuronales no siempre permiten especificar un número de procesos en paralelo de forma nativa.

Para lidiar con estas diferencias, la aplicación desarrollada se limita a aplicar el número de procesos únicamente a la validación cruzada y no al entrenamiento directo de los modelos. Esta decisión se toma para poder comparar todos los modelos en igualdad de condiciones. De esta forma, se asegura que cualquier mejora en el tiempo de ejecución sea atribuible únicamente a la paralelización del proceso de validación cruzada y no a diferencias intrínsecas en la implementación de cada modelo.

El efecto que la paralelización y la gestión de los recursos de la máquina donde se realiza el entrenamiento se podrá observar en el experimento de la sección \ref{sec:test-2-resources}. En esta prueba, 
se entrenarán los modelos en diferentes máquinas virtuales desplegadas en Microsoft Azure, cada una con configuraciones de procesador y memoria RAM distintas y alternando entre utilizar paralelización o no durante el entrenamiento. Este experimento permitirá estudiar los resultados de emplear varios procesadores en el entrenamiento por validación cruzada, analizando cómo el paralelismo y la configuración de hardware afectan tanto al rendimiento de los modelos como al consumo energético.

% \todoin{TO-DO EXPLICAR \\
% > Azure deployment
% > plantillas }

\clearpage

%%%%%%%%%%%%%%%%%%%%%%%%%%%%%%%%%%%%%%%%%%%%%%%%%%%%%%%%%%%%%%%%%%%%%%%%%%%%%%%%
%%%%%%%%%%%%%%%%%%%%%%%%%%%%%%%%%%%%%%%%%%%%%%%%%%%%%%%%%%%%%%%%%%%%%%%%%%%%%%%%
% EXPERIMENTOS Y VALIDACIÓN %
%%%%%%%%%%%%%%%%%%%%%%%%%%%%%%%%%%%%%%%%%%%%%%%%%%%%%%%%%%%%%%%%%%%%%%%%%%%%%%%%

\chapter{Experimentos y validación}
\label{chap:experimentos}

El objetivo de este capítulo es mostrar el funcionamiento de la aplicación en un caso de uso real en el que se tratará de extraer conclusiones generales acerca del consumo eléctrico de cada modelo y de si este consumo irá necesariamente acompañado de una mejora de los resultados de predicción.
Para ello se emplearán las herramientas descritas anteriormente para evaluar el consumo y el rendimiento de una serie de modelos formada por representantes de las principales familias de modelos de aprendizaje automático y recogidos en la sección~\ref{sec:models}. Estos modelos serán aplicados a los conjuntos de datos de distintas características definidos en la sección~\ref{sec:datasets}.

Durante la validación de la aplicación se llevarán acabo tres experimentos distintos.
El primero examinará el consumo energético en base al modelo seleccionado. En esta sección se tomarán varias medidas de consumo y rendimiento por modelo y conjunto de datos en una máquina con unos recursos de procesamiento concretos para analizar que modelos consumen más que otros y que características de los conjuntos de datos hacen incrementar este consumo.
El segundo consistirá en aislar un par de conjuntos de datos y tomar medidas de consumo con distintos recursos de procesado dedicados a la tarea de aprendizaje automático para observar el efecto de los recursos disponibles en el consumo energético de cada modelo.
Por último, se propondrán métodos de optimización de los modelos analizados y se examinará el efecto que pueda tener sobre su consumo. 

A través de este análisis, se pretende obtener una comprensión profunda de cómo diferentes modelos de aprendizaje automático consumen energía bajo diversas condiciones de trabajo. Este experimento también busca identificar patrones de consumo y eficiencia que puedan informar el diseño y la implementación de modelos más sostenibles y eficientes en el futuro.

\todo[inline]{Añadir esquema ???}
% What's the purpose of experiments?
% What are the expected results? More energy, more precision
% Outline / procedure / steps to follow

% 4.1 Análisis del consumo energético en base al modelo escogido
    % 4.1.1 Comparación en conjuntos de pequeño tamaño (100s - 1000s)
        % Tres conjuntos: iris, ionosphere, hepatitis
    % 4.1.2 Comparación en conjuntos de mediano tamaño (10000s)
        % Dos conjuntos: eeg-eye-state, electricity, letter, mnist_784
% 4.2 Análisis del consumo energético en base a los recursos disponibles
    % 4.2.1 Evolución del consumo con la carga del procesador
        % 1 dataset pequeño, 1 mediano
    % 4.2.2 Evolución del consumo con el aumento de recursos
        % 1 dataset grande (100000s) ?covertype?, 2-3 resource configs
% 4.3 Optimización

\section{Consumo energético basado en el modelo seleccionado}
 % - El consumo aumenta al aumentar el número de muestras
 %    1. Gráfico introductorio: número de muestras (x) vs emisiones (y), muchos modelos
 %        el consumo aumenta de forma exponencial con el número de muestras, unos modelos aumentan más que otros
 %    2. Introduce f-score: plot same lines with average f1-score instead of emissions. Los resultados son distintos, mayor consumo no implica mejor predicción
 %    3. Introduce scatter plot 4-way
    
 % - Algunos modelos son mejores que otros
 %    - Aumento de score implica aumento de consumo?
 %    - Compara average f1-score con consumo por modelo y dataset
 %    3. Introduce f-score con scatter plot 4-way, all models, 3 datasets (no average)
 %    4. Bar plot de dos datasets pequeños comparando score 

\subsubsection{Objetivos}

En esta sección se examinará el consumo energético una serie de modelos representativos aplicados a varios conjuntos de datos. El objetivo de este análisis será abordar las siguientes cuestiones clave:

\begin{itemize}
    \item Identificación de los modelos con mayor consumo energético.
    \item Determinación de los modelos cuyo consumo energético incrementa significativamente al aumentar el número de muestras.
    \item Evaluación de modelos que ofrecen mejores predicciones con menor consumo energético.
\end{itemize}

Dónde sea posible, se tratará de analizar estas cuestiones de forma general y obtener conclusiones que sean extrapolables más allá de los conjuntos de datos concretos que se hayan medido. Sin embargo, debido a la gran cantidad de variables involucradas en las variaciones de consumo entre unos casos y otros, es posible en otros conjuntos de datos se observen comportamientos distintos del consumo.

\subsubsection{Metodología}

Para analizar estas cuestiones todas las medidas de consumo serán tomadas con la aplicación desarrollada ejecutando en una misma máquina. Para cada modelo y conjunto de datos, se tomarán medidas de consumo y rendimiento utilizando validación cruzada con cinco iteraciones con un tamaño definido para los datos de testeo del 20\% del conjunto de datos. Esta técnica proporcionará una evaluación robusta y precisa tanto del comportamiento energético de los modelos como de su precisión y exactitud, ya que evitará en gran medida la presencia de valores atípicos y el riesgo de sobreajuste de los modelos.

\begin{table}[h]
    \centering
    \begin{tabular}{rl}
         Modelo & Dell XPS 15 9500\\
         Sistema Operativo & Ubuntu 20.04.6 LTS x86\_64\\
         Python & 3.12.2\\
         Procesador & Intel(R) Core(TM) i9-10885H CPU @ 2.40GHz\\
         Memoria & 7,63 GB\\
    \end{tabular}
    \caption{Características técnicas de la máquina utilizada para tomar las medidas}
    \label{tab:caracteristicas-tecnicas}
\end{table}

La aplicación será ejecutada con el siguiente comando para cada conjunto de datos distinto, en el cual \texttt{[dataset]} será sustituido por el archivo que contenga cada conjunto de datos. Adicionalmente, cualquiera de las opciones de lectura de datos descritas en la sección~\ref{sec:limpieza} podrá ser utilizada si el formato en el que se encuentren los datos lo requiere. Las características de la máquina utilizada están recogidas en la tabla~\ref{tab:caracteristicas-tecnicas}.
\begin{minted}{bash}
mlcost measure --log -cv 5 -d [dataset] [dataset-options]
\end{minted}

\subsubsection{Resultados}

La ejecución del comando anterior producirá un archivo tipo tabla de datos en formato \texttt{.csv}. La tabla~\ref{tab:medidas-1} recoge un extracto de los resultados obtenidos en el ordenador de referencia para seis conjuntos de datos distintos. El archivo completo está disponible en el repositorio de la aplicación.

\begin{table}[H]
\centerline{
\scalebox{0.78}{
\begin{tabular}{|llllllllllll|}
\hline
Dataset     & Modelo & CPU & Accuracy & Precision & F-score & Recall & Fit  & Total (s) & Emisiones & Energía  & Muestras \\
 &  & load (\%) &  & & & &  time (s) & &  (kg) &  (kWh) &  \\ \hline
Banknote    & Linear & 2.7           & 0.98      & 0.98      & 0.98    & 0.98          & 0.007             & 0.071            & 2.13E-07  & 1.10E-06 & 1372     \\
Banknote    & Linear & 2.7           & 0.97      & 0.97      & 0.97    & 0.97          & 0.006             & 0.071            & 2.13E-07  & 1.10E-06 & 1372     \\
Banknote    & Linear & 2.7           & 0.97      & 0.97      & 0.97    & 0.97          & 0.006             & 0.071            & 2.13E-07  & 1.10E-06 & 1372     \\
Banknote    & Linear & 2.7           & 0.99      & 0.99      & 0.99    & 0.99          & 0.005             & 0.071            & 2.13E-07  & 1.10E-06 & 1372     \\
Banknote    & Linear & 2.7           & 0.99      & 0.99      & 0.99    & 0.99          & 0.005             & 0.071            & 2.13E-07  & 1.10E-06 & 1372     \\
Banknote    & Forest & 2.7           & 0.99      & 0.99      & 0.99    & 0.99          & 0.184             & 1.429            & 3.27E-06  & 1.69E-05 & 1372     \\
Banknote    & Forest & 2.7           & 1.00      & 1.00      & 1.00    & 1.00          & 0.171             & 1.429            & 3.27E-06  & 1.69E-05 & 1372     \\
Banknote    & Forest & 2.7           & 0.99      & 0.99      & 0.99    & 0.99          & 0.154             & 1.429            & 3.27E-06  & 1.69E-05 & 1372     \\
Banknote    & Forest & 2.7           & 1.00      & 1.00      & 1.00    & 1.00          & 0.172             & 1.429            & 3.27E-06  & 1.69E-05 & 1372     \\
Banknote    & Forest & 2.7           & 1.00      & 1.00      & 1.00    & 1.00          & 0.158             & 1.429            & 3.27E-06  & 1.69E-05 & 1372     \\
\multicolumn{12}{|c|}{...} \\
Electricity & Neural & 102.4         & 0.82      & 0.83      & 0.83    & 0.83          & 132.768           & 518.905          & 1.19E-03  & 6.15E-03 & 45312 \\  \hline
\end{tabular}}}
\caption[Extracto de los resultados de entrenamiento]{Extracto de los resultados de entrenamiento\footnote{\url{https://github.com/l-gonz/tfg-gitt-mlcost/blob/main/model-comp-many.csv}}
\todo[inline]{Fix header format}}
\label{tab:medidas-1}
\end{table}

Cada fila en la tabla corresponde a las medidas tomadas durante una iteración de entrenamiento de cada modelo por validación cruzada. Al haber escogido utilizar validación cruzada de cinco iteraciones, la tabla de resultados cuenta con cinco filas por modelo y conjunto de datos. Sin embargo, algunas de las medidas, como la carga del procesador, el número de muestras del conjunto de datos, el tiempo total de entrenamiento, las emisiones del proceso y la energía consumida, son tomadas de forma global al finalizar todas las iteraciones de entrenamiento de cada modelo y conjunto.
Para cada iteración individual se recogen las medidas estadísticas de exactitud, precisión, exhaustividad y valor-F calculadas. Además, la implementación de validación cruzada de \texttt{scikit-learn} proporciona una medida del tiempo de entrenamiento empleado en cada iteración (fit time). Este valor puede ser utilizado junto con las emisiones y el tiempo totales de todas las iteraciones para calcular las emisiones de cada iteración de entrenamiento como muestra la ecuación~\ref{eq:fit-emissions}.

\begin{equation}
    E_1 = \frac{E_T}{t_T} \cdot t_1
    \label{eq:fit-emissions}
\end{equation}
\begin{conditions}
E_1   &   emisiones de la iteración \\
E_T   &   emisiones totales \\
t_T   &   tiempo total \\
t_1   &   tiempo de entrenamiento de la iteración
\end{conditions}

A partir de los resultados obtenidos se puede dibujar un diagrama de dispersión para visualizar cómo varían las emisiones. En la figura~\ref{fig:scatter-1} se ha utilizado el valor-F como medida de la calidad de las predicciones (eje Y), ya que es habitualmente más informativa en casos de distribuciones de clases no balanceadas. En el eje X, se han dibujado las emisiones con una escala logarítmica.

\begin{figure}[H]
  \centerline{
     \includegraphics[width=1.3\textwidth, keepaspectratio]{img/graph/4scatter-dataset-model.png}
  }
  \caption{Valor-F alcanzado por el modelo frente a las emisiones de carbono necesarias para entrenarlo, por modelo empleado y conjunto de datos utilizado}
  \label{fig:scatter-1}
\end{figure}
\todo[inline]{Fix plot titles}

% Scatter plot bla bla bla
- Outlier eeg-eye-state lower score

- Neural, higher emissions, average score, high variance, better score more complex dataset
- Support vector, starts well, but low score for eye and very high emissions for electricity

- Forest, high score, medium emissions, even eye
- Gradient, same but a little worse on both

- Neighbors, very low emissions

- Linear, low emissions, low score
- Naive bayes, very low score, low emissions


\begin{figure}[H]
  \centerline{
     \includegraphics[width=1\textwidth, keepaspectratio]{img/graph/line-nsamples-emission-log.png}
  }
  \caption{Evolución de las emisiones de carbono con el aumento de número de muestras del conjunto de datos}
  \label{fig:line-samples}
\end{figure}
\todo[inline]{Fix plot titles}


\clearpage

%%%%%%%%%%%%%%%%%%%%%%%%%%%%%%%%%%%%%%%%%%%%%%%%%%%%%%%%%%%%%%%%%%%%%%%%%%%%%%%%
%%%%%%%%%%%%%%%%%%%%%%%%%%%%%%%%%%%%%%%%%%%%%%%%%%%%%%%%%%%%%%%%%%%%%%%%%%%%%%%%
% CONCLUSIONES %
%%%%%%%%%%%%%%%%%%%%%%%%%%%%%%%%%%%%%%%%%%%%%%%%%%%%%%%%%%%%%%%%%%%%%%%%%%%%%%%%

\chapter{Conclusiones y trabajos futuros}
\label{chap:conclusiones}


\section{Consecución de objetivos}
\label{sec:consecucion-objetivos}


\section{Aplicación de lo aprendido}
\label{sec:aplicacion}


\section{Lecciones aprendidas}
\label{sec:lecciones_aprendidas}


\section{Trabajos futuros}
\label{sec:trabajos_futuros}



%%%%%%%%%%%%%%%%%%%%%%%%%%%%%%%%%%%%%%%%%%%%%%%%%%%%%%%%%%%%%%%%%%%%%%%%%%%%%%%%
%%%%%%%%%%%%%%%%%%%%%%%%%%%%%%%%%%%%%%%%%%%%%%%%%%%%%%%%%%%%%%%%%%%%%%%%%%%%%%%%
% GLOSARIO(S) %
%%%%%%%%%%%%%%%%%%%%%%%%%%%%%%%%%%%%%%%%%%%%%%%%%%%%%%%%%%%%%%%%%%%%%%%%%%%%%%%%

\printglossary[type=\acronymtype]

\printglossary

%%%%%%%%%%%%%%%%%%%%%%%%%%%%%%%%%%%%%%%%%%%%%%%%%%%%%%%%%%%%%%%%%%%%%%%%%%%%%%%%
%%%%%%%%%%%%%%%%%%%%%%%%%%%%%%%%%%%%%%%%%%%%%%%%%%%%%%%%%%%%%%%%%%%%%%%%%%%%%%%%
% APÉNDICE(S) %
%%%%%%%%%%%%%%%%%%%%%%%%%%%%%%%%%%%%%%%%%%%%%%%%%%%%%%%%%%%%%%%%%%%%%%%%%%%%%%%%

%\cleardoublepage
%\appendix
%\chapter{Manual de usuario}
%\label{app:manual}


%%%%%%%%%%%%%%%%%%%%%%%%%%%%%%%%%%%%%%%%%%%%%%%%%%%%%%%%%%%%%%%%%%%%%%%%%%%%%%%%
%%%%%%%%%%%%%%%%%%%%%%%%%%%%%%%%%%%%%%%%%%%%%%%%%%%%%%%%%%%%%%%%%%%%%%%%%%%%%%%%
% BIBLIOGRAFIA %
%%%%%%%%%%%%%%%%%%%%%%%%%%%%%%%%%%%%%%%%%%%%%%%%%%%%%%%%%%%%%%%%%%%%%%%%%%%%%%%%

\cleardoublepage

%% OLD BIBTEX CODE, TO BE DELETED
%\bibliographystyle{abbrv}
%\bibliographystyle{plain} 
%\bibliography{memoria}  % memoria.bib es el nombre del fichero que contiene las referencias bibliográficas.

% https://www.overleaf.com/learn/latex/Bibliography_management_with_biblatex
\raggedright\printbibliography[heading=bibintoc,title={Referencias}]

\end{document}
