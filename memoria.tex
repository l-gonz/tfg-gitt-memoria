%%%%%%%%%%%%%%%%%%%%%%%%%%%%%%%%%%%%%%%%%%%%%%%%%%%%%%%%%%%%%%%%%%%%%%%%%%%%%%%%
%% 
%%%%%%%%%%%%%%%%%%%%%%%%%%%%%%%%%%%%%%%%%%%%%%%%%%%%%%%%%%%%%%%%%%%%%%%%%%%%%%%%

\documentclass[a4paper, 12pt, oneside]{book}

%%-- Geometría principal (dejar activada la siguiente línea en la versión final)
\usepackage[a4paper, left=2.5cm, right=2.5cm, top=3cm, bottom=3cm]{geometry}
%%-- Activar esta línea y comentar la anterior en modo borrador, para comentarios al margen
%\usepackage[a4paper, left=2.5cm, right=2.5cm, top=3cm, bottom=3cm, marginparwidth=60pt]{geometry}

%%-- Hay que cargarlo antes que las traducciones
\usepackage{listing}                    % Listados de código

% Traducciones en XeLaTeX
\usepackage{polyglossia}
\setmainlanguage{spanish}    % Comenta esta línea si tu memoria es en inglés

% Traducciones particulares para español
% Caption tablas
\gappto\captionsspanish{
	\def\tablename{Tabla}
	\def\listingscaption{Código}
	\def\refname{Bibliografía}
	\def\appendixname{Apéndice}
	\def\listtablename{Índice de tablas}
	\def\listingname{Código}
	\def\listlistingname{Índice de fragmentos de código}
}

%% Tipografía y estilos
\usepackage[OT1]{fontenc}               % Keeps eulervm happy about accents encoding

% Símbolos y fuentes matemáticas elegantes: Euler virtual math fonts
% ¡Importante! Carga siempre las fuentes math AMS Euler ANTES QUE fontspec
\usepackage{amsmath}
\usepackage{amssymb}
\usepackage[OT1,euler-digits,euler-hat-accent,small]{eulervm}

% En XeLaTeX las fuentes se especifican con fontspec
\usepackage{fontspec}
\defaultfontfeatures{Scale=MatchLowercase, Ligatures=TeX}     % Default option in font config

% Fix para fuentes usadas con operadores y \mathrm
\DeclareSymbolFont{operators}{\encodingdefault}{\familydefault}{m}{n}

% Configura la fuente principal (serif): MinionPro
\setmainfont[Scale=0.96]{TeX Gyre Pagella}
% Configura la fuente sans-serif (\sffamily)
\setsansfont[Scale=MatchLowercase]{Lato}
% Configura la fuente para letra monoespaciada: Source Code Pro, escala 0.85
\setmonofont[Scale=0.85]{Source Code Pro}

%%-- Familias de fuentes específicas
%%-- Se pueden definir etiquetas para familias de fuentes personalizadas
%%-- que luego puedes emplear para cambiar el formato de una parte de texto
%%-- Ejemplo:
% \newfontfamily{\myriadprocond}{Myriad Pro Semibold Condensed.otf}

%%-- Opciones de interlineado y espacios
\linespread{1.07}                   % Aumentar interlineado para fuentes tipo Palatino
\setlength{\parskip}{\baselineskip} % Separar párrafos con línea en blanco

% Line and page breaking
\sloppy
\clubpenalty = 10000
\widowpenalty = 10000
\brokenpenalty = 10000
\usepackage{ragged2e}				% Enhanced ragged commands

%%-- Hipervínculos
\usepackage{url}

%%-- Gráficos y tablas
\PassOptionsToPackage{dvipdfmx,usenames,dvipsnames,x11names,table}{xcolor}             % Definiciones de colores
\PassOptionsToPackage{xetex}{graphicx}

\usepackage{subfig}                     % Subfiguras
\usepackage{pgf}
\usepackage{svg}                        % Integración de imágenes en formato SVG
\usepackage{float}                      % H para posicionar figuras
\usepackage{booktabs}                   % Already loads package xcolor
\usepackage{longtable}                  % Tables spanning several pages
\usepackage{multicol}                   % multiple column layout facilities
\usepackage{colortbl}                   % For coloured tables
\usepackage{array}                      % Column adjustment in tables

%%-- Bibliografía con Biblatex y Biber
% Más info:
% https://www.overleaf.com/learn/latex/Biblatex_bibliography_styles
% https://www.overleaf.com/learn/latex/biblatex_citation_styles
\usepackage[
    backend=biber,
    style=numeric,
    sorting=none
    ]{biblatex}
\addbibresource{memoria.bib}
% \DeclareFieldFormat{url}{\mkbibacro{URL}\addcolon\nobreakspace\url{#1}}
%\usepackage[nottoc, notlot, notlof, notindex]{tocbibind} %% Opciones de índice

%%-- Matemáticas e ingeniería
% El paquete units permite mostrar unidades correctamente
% Permite escribir unidades con espaciado y estilo de fuente correctos
\usepackage{units}         
% Ejemplo de uso: $\unit[100]{m}$ or $\unitfrac[100]{m}{s}$
% Entornos matemáticos
\newtheorem{theorem}{Theorem}

% Paquetes adicionales
\usepackage{url}                        %% Gestión correcta de enlaces
\usepackage{float}                      %% H para posicionar figuras
\usepackage[nottoc, notlot, notlof, notindex]{tocbibind}    %% Opciones de índice
\usepackage{metalogo}                   %% Múltiples logos para XeLaTeX

% Fuentes especiales y glifos
\usepackage{ccicons}                % Creative Commons icons
\usepackage{metalogo}               % XeTeX logo
\usepackage{fontawesome5}           % Fontawesome 5 icons
\usepackage{adforn} 

% Blindtext
% Opciones pangram, bible, random (defecto)
\usepackage[pangram]{blindtext}
% Lorem ipsum
\usepackage{lipsum}
% Kant lipsum
\usepackage{kantlipsum}

\usepackage{fancyvrb}               % Entornos verbatim extendidos
	\fvset{fontsize=\normalsize}    % Tamaño de fuente por defecto en fancy-verbatim
	
% Configura listas (itemize, enumerate) con iconos personalizados
% Fácil reinicio de numeración con enumerate
% Info: http://ctan.org/pkg/enumitem
\usepackage[shortlabels]{enumitem}
% Usar \usageitem para configurar iconos personalizados en listas
\newcommand{\usageitem}[1]{%
	\item[%
	{\makebox[2em]{\strut\color{GSyCblue} #1}}%
	]
}

%%-- Definición de colores personalizados
% \definecolor{LightGrey}{HTML}{EEEEEE}
% \definecolor{darkred}{rgb}{0.5,0,0}     %% Refs. cruzadas
% \definecolor{darkgreen}{rgb}{0,0.5,0}   %% Citas bibliográficas
% \definecolor{darkblue}{rgb}{0,0,0.5}    %% Hiperenlaces ordinarios (también ToC)

%%-- Configuración fragmentos de código
%%-- Minted necesita Python Pygments instalado en el sistema para funcionar
%%-- En Overleaf ya está instalada esta dependencia
\usepackage[labelfont=bf]{caption}
\newcommand{\source}[1]{\vspace{-3pt} \caption*{Fuente: {#1}} }
\usepackage{minted}
\usemintedstyle{vs}

%%-- Se debe cargar aquí para evitar warnings
\usepackage{csquotes}                   % Para traducciones con biblatex

%%-- Glosario de términos
\usepackage[acronym]{glossaries}
\makeglossaries
\loadglsentries{glossary}

% % Definición de cabeceras del documento, usando fancyhdr
% \usepackage{fancyhdr}
% %% Configuración de cabeceras para el cuerpo principal del documento
% \pagestyle{fancy}
% \fancyhead{}
% \fancyhead[RO,LE]{\myriadprocond{\thepage}}
% \renewcommand{\chaptermark}[1]{\markboth{\chaptername\ \thechapter.\ #1}{}}
% \renewcommand{\sectionmark}[1]{\markright{\thesection.\ #1}}
% \fancyhead[RE]{\myriadprocond{\leftmark}}
% \fancyhead[LO]{\myriadprocond{\rightmark}}
% \renewcommand{\headrulewidth}{0pt}
% \setlength{\headheight}{15pt} %% Al menos 15pt para evitar warning al compilar
% \fancyfoot{}
% %% Configuración para páginas con cabecera en blanco
% \fancypagestyle{plain}{%
% \fancyhf{}% clear all header and footer fields
% \fancyhead[RO,LE]{\myriadprocond{\thepage}}
% \renewcommand{\headrulewidth}{0pt}%
% \renewcommand{\footrulewidth}{0pt}%
% }
\AtBeginDocument{\addtocontents{toc}{\protect\thispagestyle{empty}}} 

%%%%%%%%%%%%%%%%%%%
%% Own changes
%%%%%%%%%%%%%%%%%%%
\newenvironment{conditions}[1][donde:]
  {#1 \begin{tabular}[t]{>{$}l<{$} @{${}={}$} l}}
  {\end{tabular}\\[\belowdisplayskip]}
\raggedbottom


%%%%%%%%%%%%%%%%%%%%%%%%%%%%%%%%%%%%%%%%

%%-- Metadatos del doc
\title{Comparativa del Consumo Energético de Algoritmos de Aprendizaje Automático}
\author{Laura Gonzalez Fernandez}

%%-- Hiperenlaces, siempre se carga al final del preámbulo
\usepackage[colorlinks]{hyperref}
\hypersetup{
    pdftoolbar=true,	% Muestra barra de herramientas en Adobe Acrobat
	pdfmenubar=true,	% Muestra menú en Adobe Acrobat
	pdftitle={TFG - Comparativa del Consumo Energético de Algoritmos de Aprendizaje Automático},
	pdfauthor={Laura Gonzalez Fernandez},
	pdfcreator={EIF, URJC},
	pdfproducer={XeLaTeX},
	pdfsubject={Machine-learning, eficiencia energética},
	pdfnewwindow=true,              %links open in new window
    colorlinks=true,                % false: boxed links; true: coloured links
    linkcolor=Firebrick4,           % enlaces internos 
    citecolor=Aquamarine4,          % enlaces a citas bibliográficas
    urlcolor=RoyalBlue3,            % hiperenlances ordinarios
    linktocpage=true                % Enlaces en núm. pág. en ToC
}

%%%---------------------------------------------------------------------------
% Comentarios en línea de revisión
% Este bloque se puede borrar cuando finalizamos el borrador

\usepackage[colorinlistoftodos]{todonotes}
\newcommand\todoin[2][]{\todo[inline, color=green!40, caption={2do}, #1]{
\begin{minipage}{\textwidth-4pt}#2\end{minipage}}}
\usepackage{verbatim}

% \hyphenpenalty 10
% \exhyphenpenalty 10
\pretolerance=5000
\tolerance=9000
\emergencystretch=2pt
\righthyphenmin=4
\lefthyphenmin=4

\interfootnotelinepenalty=10000

\usepackage[
    type={CC},
    modifier={by-sa},
    version={4.0},
]{doclicense}

%%%---------------------------------------------------------------------------

\begin{document}

%%-- Configuración común para todos los entornos listing
%%-- Descomentar para usar y personalizar valores
%\lstset{%
%breakatwhitespace=true,
% breaklines=true, 
% basicstyle=\footnotesize\ttfamily,
% keywordstyle=\color{blue},
% commentstyle=\color{green!40!black}, 
% language=Python} 
 

%%%%%%%%%%%%%%%%%%%%%%%%%%%%%%%%%%%%%%%%%%%%%%%%%%%%%%%%%%%%%%%%%%%%%%%%%%%%%%%%
% PORTADA

\begin{titlepage}
\begin{center}
\begin{tabular}[c]{c c}
%\includegraphics[bb=0 0 194 352, scale=0.25]{logo} &
\includegraphics[scale=1.5]{img/LogoURJC.png}
%&
%\begin{tabular}[b]{l}
%\Huge
%\textsf{UNIVERSIDAD} \\
%\Huge
%\textsf{REY JUAN CARLOS} \\
%\end{tabular}
\\
\end{tabular}

\vspace{1.5cm}

\Large 
ESCUELA DE INGENIERÍA DE FUENLABRADA

\vspace{1.2cm}

\Large 
GRADO EN INGENIERÍA EN TECNOLOGÍAS DE LA TELECOMUNICACIÓN

\vspace{0.8cm}
\LARGE 
\textbf{\uppercase{Trabajo Fin de Grado}}

\vspace{2cm}

\LARGE Comparativa del consumo energético de algoritmos de aprendizaje automático
\vspace{2cm}

\large
Autora : Laura González Fernández \\
Tutor : José Felipe Ortega Soto

\vspace{0.8cm}
\large
Curso Académico 2023/2024

\end{center}
\pagenumbering{gobble}
\end{titlepage}


\newpage
\mbox{}


%%%%%%%%%%%%%%%%%%%%%%%%%%%%%%%%%%%%%%%%%%%%%%%%%%%%%%%%%%%%%%%%%%%%%%%%%%%%%%%%
%%%% Para firmar
\clearpage
\frontmatter
\pagestyle{empty}
\chapter*{}
\thispagestyle{empty}

\vspace{-4cm}
\begin{center}
\LARGE
\textbf{Trabajo Fin de Grado}

\vspace{1cm}
\large
Comparativa del Consumo Energético de Algoritmos de Aprendizaje Automático

\vspace{1cm}
\large
\textbf{Autora :} Laura González Fernández  \\
\textbf{Tutor :} José Felipe Ortega Soto

\end{center}

\vspace{1cm}
La defensa del presente Proyecto Fin de Grado se realizó el día \qquad$\;\,$ de \qquad\qquad\qquad\qquad \newline de 2024, siendo calificada por el siguiente tribunal:


\vspace{0.5cm}
\textbf{Presidente:}

\vspace{0.8cm}
\textbf{Secretario:}

\vspace{0.8cm}
\textbf{Vocal:}


\vspace{0.8cm}
y habiendo obtenido la siguiente calificación:

\vspace{0.8cm}
\textbf{Calificación:}


\vspace{0.8cm}
\begin{flushright}
Fuenlabrada, a \qquad$\;\,$ de \qquad\qquad\qquad\qquad de 2024
\end{flushright}

%%%%%%%%%%%%%%%%%%%%%%%%%%%%%%%%%%%%%%%%%%%%%%%%%%%%%%%%%%%%%%%%%%%%%%%%%%%%%%%%
%%%% Dedicatoria

% \chapter*{}
% \thispagestyle{empty}
% %\pagenumbering{Roman} % para comenzar la numeración de paginas en numeros romanos
% \begin{flushright}
% \textit{To the future.\\}
% \end{flushright}

\chapter*{}
\thispagestyle{empty}
%\pagenumbering{Roman} % para comenzar la numeración de paginas en numeros romanos

\begin{flushright}
\begin{minipage}{8cm}
\raggedleft
©2024 Laura González Fernández  \\
Algunos derechos reservados. \\
\doclicenseLongText \\
\vspace{0.5cm}
\doclicenseImage
\end{minipage}

\end{flushright}

%%%%%%%%%%%%%%%%%%%%%%%%%%%%%%%%%%%%%%%%%%%%%%%%%%%%%%%%%%%%%%%%%%%%%%%%%%%%%%%%
%%%% Agradecimientos

\chapter*{Agradecimientos}
\thispagestyle{empty}
%\addcontentsline{toc}{chapter}{Agradecimientos} % si queremos que aparezca en el índice
\markboth{AGRADECIMIENTOS}{AGRADECIMIENTOS} % encabezado 

Quisiera aprovechar esta oportunidad para agradecer a las muchas personas que han contribuido de manera significativa a la realización de esta memoria de grado.

En primer lugar, agradezco a mi familia por su apoyo constante y paciencia a lo largo de mi trayectoria académica. Siempre confiaron en mí, incluso en los momentos de duda. Gracias también por su ayuda durante este proyecto, ya sea con la recopilación de datos o revisando mis resultados. A mi hermana, especialmente, gracias por soportarme estos últimos años mientras avanzaba en este proyecto.

Extiendo mi agradecimiento a mis supervisores por su orientación, experiencia y valiosos comentarios durante el desarrollo de la memoria. Su dedicación y disposición para compartir su conocimiento han sido esenciales.

A mis amigos y colegas, gracias por estar a mi lado durante este proceso. Su apoyo y las frecuentes preguntas sobre el estado de mi "interminable proyecto de IA sostenible" me motivaron a seguir adelante.

Esta memoria es el resultado de años de esfuerzo y dedicación, y agradezco a todos los que me apoyaron en este camino.

%%%%%%%%%%%%%%%%%%%%%%%%%%%%%%%%%%%%%%%%%%%%%%%%%%%%%%%%%%%%%%%%%%%%%%%%%%%%%%%%
%%%% Resumen

\chapter*{Resumen}
\thispagestyle{empty}
\addcontentsline{toc}{chapter}{Resumen} % si queremos que aparezca en el índice
\markboth{RESUMEN}{RESUMEN} % encabezado
%(máximo una página, se recomienda la estructura: antecedentes,
%objetivos, métodos, resultados y conclusiones)

El aprendizaje automático se ha desarrollado a pasos de gigante durante los últimos años, permitiendo a los sistemas informáticos abstraer relaciones complejas entre datos y hacer predicciones precisas. Sin embargo, estos avances también han incrementado el consumo energético debido al procesamiento de enormes cantidades de datos, lo que genera una huella ambiental significativa. En este contexto, se vuelve imperativo equilibrar la precisión y eficacia de los modelos de aprendizaje automático con su impacto energético.

Este proyecto tiene como objetivo proporcionar una herramienta que permita comparar modelos de aprendizaje automático en términos de precisión y consumo energético. Para ello, se ha desarrollado una aplicación utilizando scikit-learn, Python y CodeCarbon, junto con conjuntos de datos públicos. La arquitectura de la aplicación está diseñada para medir las emisiones de carbono producidas durante el entrenamiento de varios modelos, abarcando todo el proceso de aprendizaje desde la preparación de datos y la validación cruzada de los modelos hasta la recopilación de consumo energético usando CodeCarbon.

Para ofrecer una muestra del funcionamiento de la aplicación se llevaron a cabo dos experimentos principales. El primero comparó las emisiones y la precisión de varios modelos en diferentes conjuntos de datos de creciente complejidad. El modelo de vecinos más cercanos mostró un consumo energético bajo y precisión moderada, mientras que el modelo de bosque aleatorio ofreció alta precisión con consumo moderado. Las redes neuronales, aunque precisas, presentaron un consumo significativamente mayor. El segundo experimento evaluó las emisiones en máquinas con distintas configuraciones de recursos, usando máquinas virtuales en Microsoft Azure. Los resultados indicaron que el uso de paralelismo y mayor cantidad de procesadores aumenta el consumo energético por unidad de tiempo, pero reduce el tiempo total de entrenamiento, compensando el incremento en consumo.

En conclusión, este proyecto no solo proporciona una herramienta útil para la evaluación de modelos de aprendizaje automático desde una perspectiva de sostenibilidad, sino que también destaca la necesidad de tener en cuenta el impacto ambiental en el desarrollo de soluciones tecnológicas avanzadas. Futuros trabajos podrán expandir sobre esta base, explorando nuevos modelos y técnicas de optimización para reducir aún más el consumo energético sin comprometer la precisión.

%%%%%%%%%%%%%%%%%%%%%%%%%%%%%%%%%%%%%%%%%%%%%%%%%%%%%%%%%%%%%%%%%%%%%%%%%%%%%%%%
%%%% Resumen en inglés

\chapter*{Summary}
\addcontentsline{toc}{chapter}{Summary} % si queremos que aparezca en el índice
\markboth{SUMMARY}{SUMMARY} % encabezado

Machine learning has experienced great progress in recent years, allowing computer systems to abstract complex relationships between data and make accurate predictions. However, these advances have also increased energy consumption due to the cost of processing vast amounts of data, resulting in a significant environmental footprint. In this context, it is imperative to balance the precision and efficiency of machine learning models with their energy impact.

This project aims to provide a tool to compare machine learning models in terms of their precision and energy consumption. To achieve this, an application was developed using scikit-learn, Python, and CodeCarbon, along with publicly available datasets. The application's architecture is designed to measure carbon emissions produced during the training of various models, covering the entire learning process from data preparation and model cross-validation to energy consumption collection using CodeCarbon.

To demonstrate the application’s functionality, two main experiments were conducted. The first compared the emissions and precision of various models on different datasets of increasing complexity. The k-nearest neighbors model showed low energy consumption and moderate precision, while the random forest model offered high precision with moderate consumption. Neural networks, although precise, had significantly higher consumption. The second experiment evaluated emissions on machines with different resource configurations using virtual machines on Microsoft Azure. The results indicated that using parallelism and more processors increases energy consumption per unit of time but reduces total training time, compensating for the increased consumption.

In conclusion, this project provides a useful tool for evaluating machine learning models from a sustainability perspective, highlighting the importance of considering environmental impact in developing advanced technological solutions. Future work can build on this foundation by exploring new models and optimization techniques to further reduce energy consumption without compromising precision.

%%%%--------------------------------------------------------------------
% Lista de comentarios de revisión
% Se puede borrar este bloque al acabar el borrador

% \listoftodos
% \thispagestyle{empty}
% \markboth{TODO LIST}{TODO LIST} % encabezado
%%%%--------------------------------------------------------------------

%%%%%%%%%%%%%%%%%%%%%%%%%%%%%%%%%%%%%%%%%%%%%%%%%%%%%%%%%%%%%%%%%%%%%%%%%%%%%%%%
%%%%%%%%%%%%%%%%%%%%%%%%%%%%%%%%%%%%%%%%%%%%%%%%%%%%%%%%%%%%%%%%%%%%%%%%%%%%%%%%
% ÍNDICES %
%%%%%%%%%%%%%%%%%%%%%%%%%%%%%%%%%%%%%%%%%%%%%%%%%%%%%%%%%%%%%%%%%%%%%%%%%%%%%%%%

% Las buenas noticias es que los índices se generan automáticamente.
% Lo único que tienes que hacer es elegir cuáles quieren que se generen,
% y comentar/descomentar esa instrucción de LaTeX.

%%-- Índice de contenidos
\tableofcontents 
\thispagestyle{empty}
\cleardoublepage
%%-- Índice de figuras

\addcontentsline{toc}{chapter}{Lista de figuras} % para que aparezca en el indice de contenidos
\thispagestyle{empty}
\listoffigures % indice de figuras
\thispagestyle{empty}

\cleardoublepage
%%-- Índice de tablas
%\addcontentsline{toc}{chapter}{Lista de tablas} % para que aparezca en el indice de contenidos
%\listoftables % indice de tablas
% \cleardoublepage
%%-- Índice de fragmentos de código
% \addcontentsline{toc}{chapter}{List of listings} % para que aparezca en el indice de contenidos
% \listoflistings

%%%%%%%%%%%%%%%%%%%%%%%%%%%%%%%%%%%%%%%%%%%%%%%%%%%%%%%%%%%%%%%%%%%%%%%%%%%%%%%%
%%%%%%%%%%%%%%%%%%%%%%%%%%%%%%%%%%%%%%%%%%%%%%%%%%%%%%%%%%%%%%%%%%%%%%%%%%%%%%%%
%%%%%%%%%%%%%%%%%%%%%%%%%%%%%%%%%%%%%%%%%%%%%%%%%%%%%%%%%%%%%%%%%%%%%%%%%%%%%%%%
%%%%%%%%%%%%%%%%%%%%%%%%%%%%%%%   CONTENT   %%%%%%%%%%%%%%%%%%%%%%%%%%%%%%%%%%%%
%%%%%%%%%%%%%%%%%%%%%%%%%%%%%%%%%%%%%%%%%%%%%%%%%%%%%%%%%%%%%%%%%%%%%%%%%%%%%%%%
%%%%%%%%%%%%%%%%%%%%%%%%%%%%%%%%%%%%%%%%%%%%%%%%%%%%%%%%%%%%%%%%%%%%%%%%%%%%%%%%
%%%%%%%%%%%%%%%%%%%%%%%%%%%%%%%%%%%%%%%%%%%%%%%%%%%%%%%%%%%%%%%%%%%%%%%%%%%%%%%%
\mainmatter
\pagestyle{headings}

%\cleardoublepage
\chapter{Introducción}
\label{sec:intro}
\pagenumbering{arabic} 

El aprendizaje automático (\emph{Machine Learning} en inglés) es una rama de la inteligencia artificial y la ciencia computacional que se centra en el uso de datos y algoritmos para imitar la forma en la que los humanos aprenden con el objetivo de aumentar gradualmente su precisión.
Es un componente fundamental del campo de la ciencia de datos, cuya importancia ha experimentado un gran crecimiento recientemente. 
El aprendizaje automático hace uso de métodos estadísticos para entrenar algoritmos que hacen clasificaciones o predicciones y que permiten descubrir piezas clave de información dentro de proyectos de procesamiento de datos. 
Esta información afecta posteriormente en la toma de decisiones dentro de distintas aplicaciones y negocios, con una gran capacidad de impactar en el crecimiento de los mismos.

\section{Aprendizaje automático y consumo energético}

Gracias al desarrollo de nuevas tecnologías computacionales, el aprendizaje automático que se utiliza hoy en día es muy diferente de como era en el pasado.
El modelo actual surgió del reconocimiento de patrones y de la teoría de que los ordenadores pueden aprender sin necesidad de ser programados para resolver tareas específicas, cuando los investigadores interesados en la inteligencia artificial se empezaron a plantear si los ordenadores serían capaces de aprender a partir de datos.
De aquí surge la importancia del aspecto iterativo de este aprendizaje, que permite que el sistema se adapte de forma independiente cada vez que nuevos datos son incorporados y sea capaz de aprender de cada computación previa para producir decisiones y resultados que sean confiables y repetibles.

Es así que, aunque una gran parte de los algoritmos utilizados en el aprendizaje automático son conocidos desde hace relativamente bastante tiempo, la habilidad de aplicar automáticamente complejos cálculos matemáticos a grandes cantidades de datos una y otra vez, cada vez más rápidamente, es un desarrollo muy reciente conseguido gracias a los avances en componentes informáticos y la disminución de costes de grandes sistemas computacionales con enormes capacidades de memoria y procesamiento.
Este es especialmente el caso para el campo del aprendizaje profundo (\emph{deep learning}), donde los modelos han crecido en cálculos para llegar a alcanzar típicamente el orden de los GigaFlops y en requisitos de memoria que se encuentran típicamente en el orden de los millones de parámetros.

Sin embargo, este gran poder de procesamiento trae consigo un gran gasto energético.
El consumo de energía en la arquitectura de computadores ha sido foco de atención de investigadores interesados en obtener procesadores eficientes energéticamente de última generación durante décadas. 
Por otro lado, los investigadores interesados en el aprendizaje automático se han centrado principalmente en la producción de modelos cada vez más profundos y precisos, sin poner ningún límite en términos computacionales más allá de la disponibilidad de procesadores capaces.

%%%%%%%%%%%%%%%%%%

En el ámbito de los grandes modelos de aprendizaje, como los modelos de lenguaje (e.g., GPT-3), visión por computador y el aprendizaje en la nube, el impacto energético es particularmente elevado. Los grandes modelos de lenguaje, por ejemplo, requieren enormes cantidades de datos y recursos computacionales para su entrenamiento, lo que se traduce en un consumo energético considerable. Del mismo modo, las aplicaciones de visión por computador, especialmente aquellas que involucran redes convolucionales profundas (CNN), también son intensivas en energía. El aprendizaje en la nube agrava estos problemas al trasladar el consumo energético a los centros de datos, que pueden tener diferentes niveles de eficiencia y fuentes de energía.

Los grandes modelos de aprendizaje, como los modelos de lenguaje de gran escala (large language models, LLMs), han mostrado un impacto significativo en el consumo energético debido a su tamaño y complejidad. GPT-3, uno de los modelos más conocidos, cuenta con 175 mil millones de parámetros y requiere una cantidad masiva de energía para su entrenamiento. Los modelos de visión por computador, como las redes neuronales convolucionales (CNNs) y sus variantes más profundas, también consumen cantidades sustanciales de energía durante el entrenamiento y la inferencia.

El uso de aprendizaje en la nube, aunque ofrece flexibilidad y escalabilidad, puede tener implicaciones negativas para el consumo energético. Los centros de datos que alimentan estos servicios requieren grandes cantidades de electricidad, y su impacto depende en gran medida de la fuente de energía utilizada. Algunos centros de datos están migrando hacia fuentes de energía renovable para mitigar este impacto, pero la transición no es uniforme en todas las regiones.

% MORE Por qué clasificación, menciones a otros proyectos de análisis energético
% CITE https://www.sciencedirect.com/science/article/pii/S0743731518308773#b18

\section{Desarrollo de una aplicación comparativa: \emph{MLCost}}

\todoin{
Introducción a la aplicación desarrollada
}



\section{Objetivos del proyecto}
\label{sec:objetivos}

\subsection{Objetivo general} % título de subsección (se muestra)
\label{sec:objetivo-general} % identificador de subsección (no se muestra, es para poder referenciarla)


Este Trabajo de Fin de Grado tiene como objetivo crear una herramienta que permita la comparación sistemática del consumo energético y el impacto de la huella de carbono en los modelos más representativos de técnicas de clasificación de aprendizaje automático supervisado.


\subsection{Objetivos específicos}
\label{sec:objetivos-especificos}

Para esta finalidad se han tenido en cuenta los siguientes objetivos específicos:

    \begin{itemize}
        \item Estudiar las herramientas disponibles para aprendizaje automático.
        \item Estudiar el uso de herramientas de medida del consumo energético.
        \item Comparar el consumo energético en los algoritmos de clasificación más importantes.
        \item Analizar la relación entre la precisión de los modelos y su consumo energético en conjuntos de datos de distintos tamaños y características.
    \end{itemize}

% \section{Planificación temporal}
% \label{sec:planificacion-temporal}

% \todo[inline]{Completar con información temporal de Clockify}

\section{Estructura de la memoria}
\label{sec:estructura}

%% al final %%
\todo[inline]{TODO: estructura}

%\cleardoublepage
\chapter{Estado del arte}
\label{chap:sota}

\section{Revisión de la literatura}
\label{sec:lit-rev}

\todoin{
Literature review: \\
  > Codecarbon paper \\
  > Research projects into machine learning consumption \\
  > Métodos que han utilizado y resultados \\
  > ChatGPT, LLM, computer vision, cloud, etc. how much energy?
}

\section{Aprendizaje automático}
\label{sec:intro-ml}

\todoin{Mini intro to machine learning:  \\
  > supervised vs unsupervised \\
  > classification vs regression \\
  > modelos vs redes neuronales \\
  > features vs labels \\
  > train vs test \\
  > types of features}

\section{Librerías empleadas}

\subsection{Entorno de desarrollo: Visual Studio Code y WSL}
\label{sec:dev-env}

\todo[inline]{Get better info from other tfg}

VSCode es un editor de código fuente abierto altamente extensible que se ha convertido en una herramienta predilecta entre desarrolladores debido a su interfaz amigable, integración con Git y soporte para una amplia gama de lenguajes de programación y extensiones. WSL (Windows Subsystem for Linux) es una característica de Windows que permite ejecutar un entorno de Linux completo directamente sobre Windows sin la necesidad de una máquina virtual o sistemas de arranque dual.

Es posible combinar VSCode con WSL mediante una extensión específica ofrecida por VSCode para WSL que permite a los usuarios abrir carpetas y archivos directamente en el sistema de archivos de Linux, beneficiándose de las capacidades avanzadas de depuración y análisis de código que ofrece el editor. Además, la extensión proporciona a los desarrolladores acceso a un terminal de Linux real, donde pueden ejecutar y probar su código en el mismo entorno en el que se desplegará. 


\subsection{Codecarbon}

CodeCarbon \cite{codecarbon}\cite{codecarbon-software} es un paquete creado con la intención de permitir a desarrolladores monitorizar las emisiones de dióxido de carbono ($CO_{2}$) producidas por aplicaciones en Inteligencia Artificial y modelos de Aprendizaje Automático, que surge de la motivación de contar con una forma de registrar las enormes cantidades de energía que el auge de la IA ha provocado en la industria. El incremento del rendimiento y la precisión de los modelos de Aprendizaje Automático que se ha producido en años recientes se ha logrado a cambio de la utilización de enormes cantidades de información para conseguir el aprendizaje de los patrones y características subyacentes. Así, los modelos más avanzados emplean cantidades significativas de poder computacional, entrenando en procesadores avanzados durante semanas o meses y consumiendo en el proceso una gran cantidad de energía. Dependiendo de la red eléctrica utilizada, este desarrollo puede comportar la emisión de grandes cantidades de gases de efecto invernadero como el $CO_{2}$.

CodeCarbon estima la huella de carbono de una aplicación medida como kilogramos de $CO_{2}$ equivalentes, o $CO_{2}eq$, una medida estandarizada utilizada para expresar la capacidad de calentamiento global de varios gases de efecto invernadero como la cantidad de $CO_{2}$ que causaría un impacto ambiental equivalente. Para tareas de computación, que emiten $CO_{2}$ por medio de la electricidad que están consumiendo y que es generada como parte de la red eléctrica (por ejemplo, mediante la quema de combustibles fósiles como el carbón) las emisiones de carbono se miden en kilogramos de $CO_{2}$ equivalentes por kilovatio-hora. De esta forma, las emisiones de dióxido de carbono totales se calculan como el producto de la intensidad de carbono de la electricidad utilizada para la computación y la energía consumida por la infraestructura.

La intensidad de carbono de la electricidad se calcula como la media ponderada de las emisiones de las distintas fuentes de energía usadas para generar electricidad, incluyendo combustibles fósiles y renovables. En la herramienta se asigna un valor conocido de dióxido de carbono emitido por kilovatio-hora generado para cada uno de los combustibles (carbón, petróleo y gas natural). Otras fuentes renovables o consideradas como de bajo carbono incluyen la energía solar, hidroeléctrica, biomasa o geotérmica. La intensidad de carbono de cada combustible individual se calcula en base a medidas de generación de carbono y electricidad en los Estados Unidos, y aplicadas de forma generalizada en el resto del mundo. Cada red eléctrica local incluye una mezcla distinta de fuentes de energía y tiene por lo tanto asignada una intensidad de carbono total particular.

% IMAGE: Global distribution of carbon intensity (carbonboard)
% TABLE: Carbon intensity by energy source (Codecarbon/Methodology)
% CITE: CodeCarbon documentation

% _opcional_ : explanation on zero value for low-carbon fuels
% _opcional_ : explanation on power consuption calculation by CPU

\subsection{Scikit-Learn}

Scikit-learn \cite{scikit-learn} es un módulo desarrollado para Python que integra un amplio rango de algoritmos de aprendizaje automático de última generación para problemas tanto supervisados como no supervisados. Este paquete pretende llevar el aprendizaje automático a desarrolladores no especialistas mediante el uso de un lenguaje generalista de alto nivel. Se hace hincapié en la facilidad de uso, el rendimiento, la documentación y la consistencia de la API \cite{scikit-learn-api}. Tiene las mínimas dependencias necesarias y está distribuido bajo la licencia BSD, con el objetivo de incentivar su uso tanto en ambientes educativos como comerciales.

Scikit-learn expone una gran variedad de algoritmos de aprendizaje utilizando una interfaz consistente y orientada a la resolución de tareas, lo que permite una comparación sencilla entre distintos métodos de aprendizaje para una misma aplicación. Al depender del ecosistema científico de Python, puede ser integrado con facilidad en aplicaciones que se salgan del rango tradicional del análisis estadístico de datos. Además, los algoritmos, que han sido implementados en un lenguaje de alto nivel, pueden ser utilizados como bloques de construcción para desarrollar estrategias más complejas que se adecuen a cada caso particular.

\subsection{Microsoft Azure}
\label{subsec:azure}

Microsoft Azure es una plataforma de servicios en la nube que ofrece una amplia gama de herramientas y recursos para el despliegue y gestión de aplicaciones y servicios. Entre sus múltiples servicios, Azure permite a los usuarios crear y gestionar máquinas virtuales (VMs) con diversas configuraciones de recursos, adaptándose a las necesidades específicas de cada proyecto. Esta capacidad es especialmente valiosa en el campo del aprendizaje automático, donde las tareas de entrenamiento y prueba de modelos requieren recursos computacionales significativos y variados. Con Azure, los desarrolladores e investigadores pueden seleccionar configuraciones específicas de CPU, GPU, memoria y almacenamiento para optimizar el rendimiento y costo de sus experimentos de aprendizaje automático. Microsoft Azure ofrece un programa especial para estudiantes llamado "Azure for Students", que proporciona acceso gratuito a una variedad de servicios en la nube. Este programa incluye un crédito inicial de \$100 USD para usar en cualquier servicio de Azure durante 12 meses, sin necesidad de una tarjeta de crédito para registrarse. 

\subsection{Matplotlib}

Matplotlib es una de las librerías más destacadas y utilizadas en el ecosistema de Python para la creación de gráficos y visualizaciones de datos. Esta herramienta permite a los desarrolladores y científicos de datos generar una amplia variedad de gráficos estáticos, animados e interactivos con relativa facilidad. Matplotlib se caracteriza por su flexibilidad y capacidad para crear visualizaciones de alta calidad y personalizables, que van desde simples gráficos de líneas y barras hasta complejas visualizaciones tridimensionales y de mapas de calor. Además, su integración con otras librerías populares de Python, como NumPy y pandas, facilita el proceso de análisis y visualización de datos, convirtiéndola en una opción preferida en ámbitos académicos y profesionales.

El diseño de Matplotlib sigue una filosofía similar a la de MATLAB, lo que hace que sea especialmente accesible para usuarios familiarizados con ese entorno. Sin embargo, a diferencia de MATLAB, Matplotlib es de código abierto y gratuito. Las capacidades avanzadas de esta librería incluyen la personalización detallada de todos los elementos del gráfico, la posibilidad de exportar gráficos en múltiples formatos (como PNG, PDF y SVG), y la creación de gráficos interactivos utilizando bibliotecas adicionales como mpld3 y Plotly. 


%%%%
%% ALGORITHMS, CURRENTLY
%%%%

\section{Modelos utilizados}
\label{sec:models}

\subsection{Modelos lineales: regresión logística}

Los modelos lineales son una clase fundamental de técnicas estadísticas y de aprendizaje automático que se utilizan para predecir una variable objetivo a partir de una o más variables independientes. La característica principal de estos modelos es la relación lineal entre las variables independientes y la variable objetivo. En su forma más simple, el modelo lineal se representa mediante la ecuación $y = \beta_0 + \beta_1 x_1 + \beta_2 x_2 + \dots + \beta_n x_n$, donde $y$ es la variable dependiente, $x_1, x_2, \dots, x_n$ son las variables independientes, y $\beta_1, \beta_2, \dots, \beta_n$ son los coeficientes que representan el impacto de cada variable independiente sobre la variable dependiente.

Dentro de los modelos lineales, la regresión logística es especialmente relevante para problemas de clasificación. A diferencia de la regresión lineal, que se utiliza para problemas de predicción continua, la regresión logística se emplea cuando la variable objetivo es categórica. En su forma binaria más simple, la regresión logística predice la probabilidad de que una observación pertenezca a una de dos posibles categorías. La función logística o sigmoide, 
\begin{equation*}
    P(y=1\mid x) = \frac{1}{1+e^{-\beta_0 + \beta_1 x_1 + \beta_2 x_2 + \dots + \beta_n x_n}},
\end{equation*}
transforma cualquier valor real de la combinación lineal de las variables independientes en un valor entre 0 y 1, lo que se interpreta como una probabilidad. Los parámetros desconocidos $\beta_i$ son estimados habitualmente a través del método de máxima verosimilitud.

El uso de la regresión logística en problemas de clasificación es ventajoso por varias razones. Primero, es un modelo interpretativo, ya que los coeficientes obtenidos pueden proporcionar una idea clara de cómo cada variable independiente afecta la probabilidad de pertenecer a una categoría específica. Además, la regresión logística es relativamente fácil de implementar y eficiente computacionalmente, lo que la hace adecuada para grandes conjuntos de datos. Por último, aunque su capacidad para capturar relaciones complejas es limitada en comparación con modelos no lineales más avanzados, su simplicidad y efectividad la convierten en una opción sólida para establecer una línea base en proyectos de clasificación, permitiendo comparaciones posteriores con modelos más complejos.

La librería scikit-learn proporciona una interfaz para construir y evaluar modelos de regresión logística a través de su clase \texttt{LogisticRegression} \cite{sk-logistic-regression}. Esta clase permite a los usuarios configurar parámetros como la regularización, la penalización y el algoritmo de optimización. Esta implementación admite dos tipos de penalización: \texttt{L1} (Lasso) y \texttt{L2} (Ridge) que ayudan a prevenir el sobreajuste. Además, permite ajustar la fuerza de la regularización a través de un parámetro \texttt{C}. Esta flexibilidad facilita la adaptación del modelo a diferentes problemas y conjuntos de datos. La personalización aumenta al elegir un algoritmo de optimización mediante el parámetro \texttt{solver}, que admite varias opciones que pueden adaptarse al tamaño o el tipo de atributos del conjunto de datos a utilizar. Adicionalmente, \texttt{LogisticRegression} es capaz de manejar problemas de clasificación binaria y multiclase, con diferentes estrategias de acuerdo al algoritmo de resolución escogido.


\subsection{Árboles de decisión: bosque aleatorio}

Otra técnica popular de aprendizaje automático utilizada para problemas tanto de clasificación como de regresión son los árboles de decisión \cite{sk-decision-trees}. Su estructura jerárquica consiste en nodos de decisión que dividen iterativamente el conjunto de datos en subconjuntos más homogéneos, facilitando así la predicción de la variable objetivo. Cada nodo del árbol representa una prueba sobre un atributo específico, y cada rama denota el resultado de la prueba. Los árboles de decisión son fáciles de interpretar y visualizar, lo que los hace muy útiles para entender las decisiones del modelo y comunicar los resultados a personas no técnicas.

Sin embargo, los árboles de decisión tienen algunas limitaciones, como su tendencia a sobreajustarse a los datos de entrenamiento, lo que puede llevar a un desempeño deficiente en datos nuevos. Para abordar estas limitaciones, se emplean técnicas de ensamblado como los bosques aleatorios (\emph{Random Forests}). Un bosque aleatorio es un conjunto de árboles de decisión entrenados sobre diferentes subconjuntos del conjunto de datos y con diferentes características seleccionadas aleatoriamente. La predicción final se obtiene mediante el promedio de las predicciones individuales de los árboles (en el caso de regresión) o mediante un voto mayoritario (en el caso de clasificación). Esta metodología mejora significativamente la precisión del modelo y reduce el riesgo de sobreajuste.

El uso de bosques aleatorios en problemas de clasificación ofrece varias ventajas. Primero, al combinar múltiples árboles, el modelo se vuelve más robusto y generalizable, mitigando la influencia de outliers y ruido en los datos. Además, los bosques aleatorios proporcionan una medida de importancia de las características, lo que permite identificar las variables más relevantes para la predicción. Esta capacidad es especialmente útil en contextos académicos y profesionales donde la interpretación del modelo y la identificación de factores clave son cruciales para la toma de decisiones informadas.

La implementación de un modelo de bosque aleatorio en la librería scikit-learn se ofrece a través de la clase \texttt{RandomForestClassifier} \cite{sk-random-forest}, que permite ajustar diversos parámetros como el número de árboles (\texttt{n\_estimators}), la profundidad máxima de los árboles (\texttt{max\_depth}) y el número mínimo de muestras por hoja (\texttt{min\_samples\_leaf}). 

\subsection{Máquinas de vector soporte (SVM)}

En las máquinas de vector soporte (\emph{Support Vector Machines}, SVM) \cite{sk-svm-theory} el principal objetivo es encontrar el hiperplano óptimo que maximiza el margen entre las diferentes clases en un espacio de características. Este margen es la distancia más amplia posible entre el hiperplano y los puntos de datos más cercanos de cualquier clase, conocidos como vectores soporte. La SVM es particularmente eficaz en espacios de alta dimensionalidad y es robusta frente al sobreajuste, lo que la hace adecuada para conjuntos de datos complejos.

Una de las ventajas clave de las SVM es su capacidad para manejar problemas no lineales mediante el uso de funciones kernel. Estos núcleos transforman los datos en un espacio de mayor dimensión donde un hiperplano lineal puede separarlos. Los tipos de kernel más comunes incluyen el lineal, polinómico, radial (RBF), y sigmoide. Esta flexibilidad permite a las SVM abordar una amplia gama de problemas de clasificación, incluso cuando las relaciones entre las características y las etiquetas son altamente no lineales. Sin embargo, para que funcione de forma adecuada es necesario seleccionar el kernel apropiado y ajustar sus parámetros, por ejemplo, mediante validación cruzada.

En el contexto de problemas de clasificación, las SVM son especialmente útiles debido a su alta precisión y su capacidad para manejar tanto clases linealmente separables como no separables. En escenarios de clasificación binaria, las SVM son capaces de encontrar la frontera de decisión que maximiza la separación entre las dos clases. Para problemas de clasificación multiclase, se pueden aplicar estrategias para descomponer el problema en múltiples problemas de clasificación binaria. Aunque las SVM pueden ser computacionalmente intensivas, especialmente con grandes conjuntos de datos, tienen una gran eficacia en términos de rendimiento y de capacidad para generalizar bien a datos no vistos.

La clase \texttt{SVC} \cite{sk-svm} de scikit-learn permite ajustar parámetros clave como el tipo de kernel, el parámetro de regularización (\texttt{C}) y coeficientes específicos del kernel elegido.

\subsection{Vecinos más cercanos (k-NN)}

El método de vecinos más cercanos (k-Nearest Neighbors, k-NN) es una técnica de aprendizaje supervisado utilizada para resolver problemas tanto de clasificación como de regresión. Este método se basa en la premisa de que objetos similares generalmente se encuentran cerca en el espacio de características. En problemas de clasificación, el algoritmo k-NN asigna una etiqueta a una nueva instancia basándose en las etiquetas de sus k vecinos más cercanos en el conjunto de datos de entrenamiento. La cercanía o similitud entre las instancias se mide generalmente utilizando distancias métricas, como la distancia euclidiana, Manhattan, o Minkowski.

Una de las principales ventajas del k-NN es su simplicidad y facilidad de implementación. A diferencia de otros algoritmos que requieren un proceso de entrenamiento explícito, k-NN es un algoritmo perezoso que almacena todas las instancias del conjunto de entrenamiento y realiza el cálculo de las distancias en el momento de la predicción. Esta característica lo convierte en un método intuitivo y fácil de entender, aunque también implica que puede ser computacionalmente costoso, especialmente con grandes conjuntos de datos. Además, la elección del número de vecinos (k) y la métrica de distancia son cruciales para el rendimiento del modelo. Valores de k demasiado bajos pueden llevar a un sobreajuste, mientras que valores demasiado altos pueden conducir a un subajuste.

En el contexto de problemas de clasificación, el k-NN es especialmente útil en situaciones donde las fronteras de decisión entre clases no son lineales y pueden ser complejas. Debido a su naturaleza basada en instancias, el k-NN puede capturar patrones locales en los datos y adaptarse a cambios en la distribución de las clases. Sin embargo, el rendimiento de k-NN puede verse afectado por la presencia de ruido y características irrelevantes, lo que subraya la importancia de la preprocesamiento de datos, como la normalización y la selección de características. A pesar de estas limitaciones, el k-NN sigue siendo una herramienta valiosa para la clasificación debido a su simplicidad y flexibilidad.

La clase \texttt{KNeighborsClassifier} \cite{sk-knn} de la librería scikit-learn proporciona la interfaz para configurar y utilizar el algoritmo k-NN. Esta clase ofrece la posibilidad de especificar el número de vecinos a considerar mediante el parámetro \texttt{n\_neighbors} y de seleccionar la métrica de distancia adecuada utilizando el parámetro \texttt{metric}, que incluye opciones como la distancia euclidiana y Manhattan. Además, scikit-learn permite ajustar parámetros adicionales como el peso de los vecinos, que puede ser uniforme o basado en la distancia.

\subsection{Naive Bayes (Naive Bayes gaussiano)}

Los métodos Naive Bayes \cite{sk-naive-bayes} son un conjunto de algoritmos de aprendizaje supervisado basados en la aplicación del teorema de Bayes con la suposición "ingenua" (en inglés, \emph{naive}) de independencia condicional entre cada par de características dado el valor de la variable de clase. El teorema de Bayes postula la siguiente relación, dada la variable de clase $y$ y los vectores característica dependientes $x_{1}$ a $x_{n}$:
\begin{equation*}
    P(y \mid x_{1},\dots,x_{n}) = \dfrac{P(y)P(x_{1},\dots,x_{n}\mid y)}{P(x_{1},\dots,x_{n})}
\end{equation*}

Usando la suposición ingenua de independencia condicional de forma que
\begin{equation*}
    P(x_{i} \mid y,x_{1},\dots,x_{i-1},x_{i+1},\dots,x_{n})=P(x_{i}\mid y),
\end{equation*}

para todo $i$, esta relación se simplifica a 
\begin{equation*}
    P(y\mid x_{1},\dots,x_{n})=\dfrac{P(y) \prod^{n}_{i=1} P(x_{i}\mid y)}{P(x_{1},\dots,x_{n})}
\end{equation*}

Como $P(x_{1},\dots,x_{n})$ es constante dada la entrada, se puede utilizar la siguiente regla de clasificación:
\begin{equation*}
    P(y\mid x_{1},\dots,x_{n}) \propto P(y) \prod^{n}_{i=1}P(x_{i}\mid y) \Rightarrow 
    \hat{y} = \arg \max_{y} P(y) \prod^{n}_{i=1}P(x_{i}\mid y),
\end{equation*}

y se puede usar la estimación de máximo a posteriori (MAP) para estimar $P(y)$ y $P(x_{i}\mid y)$; y la primera expresión es entonces la frecuencia relativa de la clase $y$ en el conjunto de entrenamiento.

Los diferentes clasificadores Naive-Bayes difieren sobre todo en la suposición que realicen sobre la distribución de $P(x_{i} \mid y)$. El algoritmo Naive-Bayes gaussiano es una variante específica de Naive-Bayes que se utiliza cuando las características son continuas y se asumen distribuidas según una distribución normal (gaussiana), lo que es común en muchas aplicaciones reales. En este caso, el clasificador estima los parámetros de la distribución (media y varianza) para cada característica en cada clase utilizando los datos de entrenamiento. Durante la clasificación de una nueva instancia, el algoritmo calcula la probabilidad de que esta instancia pertenezca a cada clase basándose en las distribuciones gaussianas estimadas y asigna la clase con la probabilidad a posteriori más alta.

A pesar de sus suposiciones aparentemente simplificadas, los clasificadores Naive-Bayes obtienen buenos resultados en muchos problemas del mundo real, especialmente en aplicaciones como la clasificación de documentos y filtros de spam; y requieren pequeñas cantidades de datos de entrenamiento para estimar los parámetros necesarios. Estos clasificadores pueden llegar a ser extremadamente rápidos comparados a otros métodos más sofisticados. La separación de las características condicionales de clase significa que cada distribución puede ser estimada independientemente como una distribución unidimensional. Esto a su vez ayuda a aliviar problemas provocados por la dimensionalidad de las distribuciones.

La implementación de este algoritmo en la librería scikit-learn se encuentra en la clase \texttt{GaussianNB} \cite{sk-nb-gaussian}.

\subsection{Métodos de ensamblaje (ensemble): máquinas de potenciación de gradiente}

El método de ensamblaje por máquinas de potenciación de gradiente (Gradient Boosting Machines, GBM) es una técnica de aprendizaje automático que combina múltiples modelos débiles para crear un modelo más fuerte y preciso. La idea central de GBM es construir el modelo de forma secuencial, donde cada modelo sucesivo intenta corregir los errores del modelo anterior. En el contexto de problemas de clasificación, cada nuevo árbol de decisión se ajusta a los residuos de los árboles anteriores, utilizando el gradiente de la función de pérdida como guía. Esta estrategia permite que GBM capture relaciones complejas y no lineales en los datos, mejorando considerablemente la precisión del modelo.

Uno de los mayores beneficios de usar máquinas de potenciación de gradiente para problemas de clasificación es su capacidad para manejar datos heterogéneos y características con diferentes escalas y distribuciones. GBM puede ajustarse a los datos de manera muy detallada, lo que lo hace adecuado para problemas donde las relaciones entre las variables no son lineales ni simples. Sin embargo, este alto nivel de ajuste también puede llevar a un sobreajuste si no se controla adecuadamente. Por ello, es crucial utilizar técnicas como la validación cruzada y ajustar parámetros como el número de árboles, la tasa de aprendizaje, y la profundidad máxima de los árboles para encontrar el equilibrio adecuado entre sesgo y varianza.

En problemas de clasificación, GBM es particularmente eficaz debido a su capacidad para manejar clases no balanceadas y proporcionar probabilidades de clasificación en lugar de simplemente etiquetas de clase. Esto es especialmente útil en aplicaciones como la detección de fraudes, diagnóstico médico y otros escenarios donde la probabilidad de pertenecer a una clase particular puede ser más informativa que la clasificación binaria. Además, las implementaciones modernas de GBM, como XGBoost, LightGBM y CatBoost, han optimizado significativamente la velocidad y eficiencia del algoritmo, permitiendo su aplicación en grandes conjuntos de datos y en tiempo real.

La implementación de máquinas de potenciación de gradiente en la librería scikit-learn es accesible a través de la clase GradientBoostingClassifier \cite{sk-gradient-boost} permite a los usuarios ajustar una variedad de hiperparámetros para optimizar el rendimiento del modelo. Los usuarios pueden especificar el número de árboles, la tasa de aprendizaje, y otros parámetros clave para controlar la complejidad y la capacidad de generalización del modelo.

\subsection{Redes neuronales (Deep Neural Networks, DNN) Multi-layer Perceptron}

Las redes neuronales están compuestas por capas de neuronas artificiales que imitan el comportamiento de las neuronas en el cerebro humano. Cada neurona realiza una operación matemática simple sobre la entrada y pasa la salida a la siguiente capa. A través de múltiples capas, la red es capaz de capturar patrones y características complejas en los datos. Estás propiedades pueden aplicarse a situaciones de aprendizaje supervisado mediante el uso de algoritmos como el perceptrón multicapa (Multilayer Perceptron, MLP).

El algoritmo del perceptrón multicapa es una clase de red neuronal supervisada que consiste en una capa de entrada, una o más capas ocultas y una capa de salida. Cada capa está totalmente conectada a la siguiente, y cada conexión tiene un peso ajustable que se optimiza durante el proceso de entrenamiento. En problemas de clasificación, el MLP es especialmente útil debido a su capacidad para aprender representaciones no lineales complejas de los datos. El algoritmo utiliza funciones de activación no lineales, como la función sigmoide o la ReLU (Rectified Linear Unit), que permiten que la red capture y modele relaciones no lineales en los datos. El entrenamiento de un MLP se realiza mediante un proceso llamado retropropagación del error (backpropagation), que ajusta iterativamente los pesos para minimizar una función de pérdida, típicamente la entropía cruzada en problemas de clasificación.

El uso del MLP ofrece varias ventajas. Primero, debido a su estructura de múltiples capas, el MLP puede aprender y generalizar bien a partir de datos complejos y de alta dimensionalidad. Esto es particularmente útil en dominios como el reconocimiento de imágenes, la detección de voz y la clasificación de texto, donde las relaciones entre las características no son triviales. Además, las redes neuronales pueden ser adaptadas a problemas multiclase con facilidad, utilizando técnicas como la salida softmax en la capa final para obtener probabilidades de clase. Sin embargo, el MLP también requiere una cantidad considerable de datos y poder computacional para entrenar eficazmente, y la selección de la arquitectura y los hiperparámetros adecuados puede ser un proceso desafiante que a menudo implica ensayo y error y validación cruzada.

La librería scikit-learn implementa la clase MLPClassifier \cite{sk-multilayer-perceptron} para el algoritmo del perceptrón multicapa. Los usuarios pueden definir parámetros como el número de capas ocultas y el número de neuronas por capa (hidden\_layer\_sizes), la función de activación (activation), y el algoritmo de optimización (solver). Scikit-learn también permite ajustar la tasa de aprendizaje (learning\_rate) y utilizar técnicas de regularización para prevenir el sobreajuste.
\section{Datos utilizados}
\label{sec:datasets}

\todoin{Update with new datasets:  \\
  > Iris, Ionosphere, Banknote, Phoneme, Eeg-eye-state, Electricity \\
  > Covertype, Poker-hand \\
  > For-each: \\
  ---> content explain \\
  ---> size \\
  ---> features \\
  ---> origin, cites \\
  ---> what has been used for before}

Todos los conjuntos de datos utilizados en los análisis realizados están disponibles de forma libre en la web y proceden de dos fuentes: el repositorio de aprendizaje automático de la Universidad de California Irvine y la plataforma OpenML.

El repositorio de aprendizaje automático de la UCI \cite{ml-uci} es una colección de bases de datos, teorías de dominios y generadores de datos que son utilizados por la comunidad de aprendizaje automático para el análisis empírico de algoritmos de aprendizaje automático. El archivo fue creado en 1987 como un servidor FTP por David Aha y otros compañeros estudiantes en la UCI. Desde entonces ha sido ampliamente utilizado por estudiantes, educadores e investigadores de todo el mundo como una fuente primaria de conjuntos de datos para aprendizaje automático.

OpenML \cite{openml} es una plataforma abierta para compartir conjuntos de datos, algoritmos y experimentos, que tiene como objetivo lograr que las investigaciones sobre aprendizaje automático sean más fácilmente accesibles y reutilizables. Esta plataforma contiene un repositorio con más de cinco mil conjuntos de datos y resultados de experimentos que otros usuarios han realizado con ellos. Además, ofrece librerías para integrar la recuperación de sus datos directamente con el código de desarrollo de modelos de aprendizaje. Otras librerías como Pandas para procesado de datos, ofrecen también opciones para descargar los datos fácilmente de la plataforma.

A continuación se describen las características más importantes de los conjuntos de datos empleados.
% ANNEX All attributes

\subsection{Iris}

% This is perhaps the best known database to be found in the pattern recognition literature. Fisher's paper is a classic in the field and is referenced frequently to this day. (See Duda & Hart, for example.) The data set contains 3 classes of 50 instances each, where each class refers to a type of iris plant. One class is linearly separable from the other 2; the latter are NOT linearly separable from each other.

% Predicted attribute: class of iris plant.
% This is an exceedingly simple domain.

% Attribute Information:
% 1. sepal length in cm
% 2. sepal width in cm
% 3. petal length in cm
% 4. petal width in cm
% 5. class: 
%    -- Iris Setosa
%    -- Iris Versicolour
%    -- Iris Virginica

Iris esta entre las primeras bases de datos recogidas, y es una de las más conocidas y utilizadas en la literatura sobre reconocimiento de patrones. Los datos originales fueron publicados por R.A. Fisher en 1936, y la versión utilizada en este proyecto procede de la UCI (1988) \cite{iris-dataset}.
El conjunto contiene tres clases distintas con 50 ejemplos cada una, donde cada clase se refiere a una especie del género de plantas Iris. Cada ejemplo contiene cuatro atributos que describen la longitud y la anchura del sépalo y el pétalo de la planta en centímetros. Las tres opciones de clasificación son Iris Setosa, Iris Versicolour e Iris Virginica.

Se trata de un campo de clasificación extremadamente simple, donde una de las clases es linealmente separable de las otras dos, y estas últimas son separable entre sí de forma no lineal.



\subsection{Ionosfera}

Se trata de un conjunto de datos con 351 muestras procedente del repositorio de la UCI\footnote{\url{https://archive.ics.uci.edu/ml/datasets/Ionosphere}}. Contiene datos de radar obtenidos por el grupo de física espacial de la Universidad John Hopkins y donados por Vince Sigillito en 1989. El sistema radar está ubicado en Goose Bay, Labrador y consiste en un array de 16 antenas de alta frecuencia. El objetivo es la medición de electrones libres en la ionosfera y su clasificación binaria entre "buenas" respuestas del radar que indican evidencia de algún tipo de estructura en la ionosfera y "malas" respuestas en las que las señales simplemente pasan a través de la ionosfera. Las señales recibidas se procesaron utilizando una función de autocorrelación con el tiempo de pulso y el número de pulso como argumentos y cada una de las muestras del conjunto de datos está descrita por dos atributos continuos para cada uno de los 17 números de pulso, correspondientes al valor complejo obtenido de la señal electromagnética compleja. Hay por lo tanto un total de 34 características continuas por muestra.

\subsection{Billetes}

Este conjunto de datos contiene información extraída de 1372 imágenes tomadas para evaluar un procedimiento de autenticación de billetes. Fue donado en Agosto de 2012 por Volker Lohweg de la Universidad de Ciencias Aplicadas de Ostwestfalen-Lippe, Alemania, al repositorio de la UCI\footnote{\url{https://archive.ics.uci.edu/ml/datasets/banknote+authentication}}. Para la digitalización de las imágenes tomadas se empleó una cámara industrial normalmente utilizada para la inspección de impresiones. Las imágenes finales tienen un tamaño de 400x400 píxeles con una resolución de alrededor de 660 dpi en escala de grises. Posteriormente, se empleó una herramienta de transformada ondícula para extraer cuatro características continuas de las imágenes.

\subsection{Fonemas}
\subsection{Electroencefalograma}
\subsection{Electricidad}
% \subsection{Tipo de cubierta}
% \subsection{Mano de póquer}


%\cleardoublepage
\chapter{Diseño e implementación}
\label{chap:diseño}

Este capítulo proporciona una visión detallada sobre la metodología y las herramientas utilizadas para medir el consumo energético y evaluar el rendimiento de modelos de aprendizaje automático en el contexto de la aplicación desarrollada: \texttt{MLCost}. Se comenzará describiendo la arquitectura de la aplicación y los módulos que la componen, para posteriormente profundizar en la metodología del proceso de aprendizaje. Este proceso empezará con la preparación de los datos, donde destacan las técnicas de preprocesamiento y la selección de modelos representativos de diversas familias algorítmicas. 

Durante la fase de entrenamiento, se empleará la biblioteca CodeCarbon para medir las emisiones de carbono y el consumo de energía, utilizando técnicas como la validación cruzada para evaluar la precisión y el rendimiento de los modelos. La evaluación de las predicciones se realiza a través de métricas estándar como precisión, exhaustividad y F-score, para obtener una estimación robusta del desempeño del modelo. Para finalizar, se presentarán los resultados con distintas herramientas de visualización y se discutirá la gestión eficaz de recursos, enfocándose en la configuración del procesador y el uso de múltiples núcleos para optimizar el rendimiento y minimizar el sesgo en las mediciones de consumo energético.


\section{Arquitectura}

\begin{figure}[H]
  \centerline{
     \includegraphics[width=\textwidth, keepaspectratio]{img/general-arch.jpg}
  }
  \caption{Arquitectura de la aplicación desarrollada}
  \label{fig:app-arch}
\end{figure}

La arquitectura de la aplicación se organiza en torno a varios componentes clave que se integran en el paquete MLCost para ofrecer un entorno robusto para la evaluación de modelos de aprendizaje automático, tal y como muestra la figura~\ref{fig:app-arch}.

El punto de entrada de la aplicación es el módulo \texttt{cli}, que se encarga de gestionar la interfaz de línea de comandos (\emph{command-line interface}, CLI) que facilita la interacción del usuario con las funcionalidades principales de la aplicación. Este módulo permite a los usuarios ejecutar la aplicación desde la terminal, proporcionando una manera flexible y accesible de realizar diversas operaciones, como la selección de modelos de aprendizaje automático, la configuración de parámetros de ejecución y la especificación de archivos de datos de entrada. El módulo \texttt{cli} utiliza la librería \texttt{click} para facilitar el procesamiento de los argumentos de la línea de comandos. Este enfoque permite definir una variedad de opciones y argumentos que los usuarios pueden especificar al ejecutar la aplicación. Por ejemplo, los usuarios pueden seleccionar que opciones de limpieza de datos serán utilizadas, establecer el número iteraciones para la validación cruzada, y decidir si serán ejecutadas en paralelo en el procesador. El conjunto completo de opciones de línea de comandos que se pueden utilizar está recogido en el Anexo~\ref{app:cli}. Una vez que se capturan los argumentos, \texttt{cli} invoca las funciones correspondientes definidas en el modulo \texttt{mlcost}.

Este modulo se encarga de la integración del seguimiento de emisiones de carbono y consumo energético durante el entrenamiento y la evaluación de los modelos de aprendizaje automático. Este módulo utiliza la biblioteca codecarbon para realizar el seguimiento del impacto ambiental de estos procesos computacionales. El módulo es responsable de gestionar el proceso de entrenamiento, creando un objeto de la clase \texttt{Trainer} por cada modelo a evaluar y asegurándose de que los datos son preprocesados para mejorar la precisión del modelo. Una vez que los datos están preparados, el módulo comienza la medición de emisiones, encarga el entrenamiento del modelo y, al finalizar, detiene el rastreador de emisiones y recupera los datos finales de consumo. Estos datos incluyen la duración del seguimiento, el consumo energético y las emisiones equivalentes de carbono. Los resultados, junto con los datos de rendimiento obtenidos durante la validación del modelo, se registran en un archivo para posterior referencia.

El núcleo de la aplicación reside en el modulo \texttt{learn}, que expone la clase \texttt{Trainer} ya mencionada, donde se definen las tareas principales de carga de datos, entrenamiento y evaluación de modelos, así como la recopilación de métricas de desempeño. Estos procesos serán descritos en mayor detalle en las secciones posteriores.

Para finalizar, la aplicación contiene dos módulos auxiliares que facilitan el desarrollo y proporcionan utilidades para examinar los resultados obtenidos.

El modulo \texttt{graphs} maneja la visualización de los resultados mediante diversas funciones de creación de gráficas que utilizan la librería matplotlib para crear gráficos de dispersión, de líneas y de barras. Estas visualizaciones ayudan a interpretar las relaciones entre las emisiones, el consumo de energía y las métricas de rendimiento de los modelos evaluados. Las gráficas también permiten comparar el desempeño de diferentes modelos bajo diversas cargas de CPU, ofreciendo una visión clara de cómo los recursos del sistema afectan la eficiencia y la precisión del modelo.

El componente \texttt{utils} proporciona funciones auxiliares para la aplicación, tales como la impresión de resultados y la recopilación de información del sistema operativo, así como la gestión de archivos de salida en formato CSV. Este archivo incluye utilidades para manejar los datos de emisiones y consumo energético, formateando la información de manera que sea fácilmente interpretable. 


\section{Lectura y limpieza de los datos}
\label{sec:limpieza}

El componente más importante de todo proyecto de aprendizaje automático son sin duda los datos. Por este motivo se han desarrollado con el tiempo una gran cantidad de librerías para facilitar la tarea de los desarrolladores que quieren trabajar con información de forma estructurada. En Python una de las más importantes es la librería \texttt{pandas} (\ref{subsec:pandas}), que ofrece la clase \texttt{DataFrame} con la que se pueden procesar grandes cantidades de datos de forma similar a como se trabajaría con una tabla o hoja de cálculo, con filas y columnas con distintos tipos de información.

Los conjuntos de datos que están disponibles de forma libre en repositorios de universidades y otras organizaciones de ciencia de datos no siempre siguen un mismo formato. Por este motivo el primer paso para aplicar métodos de aprendizaje automático en datos específicos será siempre deshacerse del formato de presentación y guardarlos en estructuras de datos operables por un ordenador, como \texttt{DataFrames} de \texttt{pandas}. En este proyecto, la intención ha sido ser capaz de procesar datos presentados con varios formatos distintos. Para ello, se han creado distintas argumentos para la línea de comandos que pueden ser utilizados al lanzar la aplicación especificando las características concretas del conjunto de datos a tratar.

En general, el proceso de lectura y limpieza de los datos va seguir siempre las siguientes fases:
\begin{enumerate}
    \item Lectura
    \begin{enumerate}
        \item Leer el archivo que contiene los datos.
        \item Separar los datos de entrenamiento de los datos de testeo.
        \item Identificar el tipo de datos de cada columna.
    \end{enumerate}
    \item Limpieza
    \begin{enumerate}
        \item Eliminar columnas que no pueden ser utilizadas.
        \item Reemplazar valores numéricos que falten.
        \item Reemplazar columnas categóricas por columnas booleanas.
        \item Escalar las características numéricas.
    \end{enumerate}
\end{enumerate}

El primer paso es leer los datos de uno o varios archivos, que serán generalmente archivos de texto en formato \texttt{.txt} o \texttt{.csv}. Es en este paso en el que se dan mayores diferencias entre distintos conjuntos de datos y la razón de que se hayan añadido las distintas opciones de línea de comandos al programa para resolverlo. Los archivos de texto que contienen los datos pueden utilizar diferentes caracteres separadores entre columnas (como coma o espacio), representar valores que no han sido tomados con diferentes símbolos (como '?' o '-'), presentar o no una fila inicial con los nombres de las columnas, identificar la columna de las etiquetas de distintas maneras e incluso separar en distintos archivos los conjuntos de entrenamiento y de testeo de forma previa. Todas estas opciones son tenidas en cuenta durante la lectura para convertir los archivos de texto en \texttt{dataframes} sobre los que las librerías de aprendizaje pueden operar. En el Anexo~\ref{app:cli} se incluye un compendio de todas las opciones disponibles en la aplicación.

Una vez que los datos están recogidos, el siguiente paso es separar los datos de entrenamiento de los de testeo. Para ello, primero se descartarán todas las filas de datos que no estén etiquetadas, si las hubiera. Por defecto, la separación se hace al 80-20 y de forma aleatoria, excepto si los conjuntos están previamente separados en dos archivos. Para terminar el proceso de lectura, las columnas que contienen características numéricas se separan de las que contienen características categóricas, para poder tratarlas de forma específica durante la limpieza.

En la sección de limpieza, el objetivo es eliminar las características que puedan crear obstáculos en el entrenamiento de los modelos. Tres problemas básicos son tratados: falta de datos en columnas numéricas, datos presentados de forma categórica y diferentes escalas de los datos numéricos. Para lidiar con ello se utilizará un tipo de clases provistas por la librería \texttt{scikit-learn} denominadas transformadores. Estos transformadores encapsulan distintas herramientas de preprocesado y limpieza que son usadas a menudo. Para la falta de datos numéricos, se utilizara un introductor simple de medidas (\texttt{SimpleImputer}), que rellenará los datos que falten con la media de los datos disponibles. 

Respecto a las columnas categóricas, muchos algoritmos de aprendizaje no están diseñados para trabajar directamente con variables no numéricas (generalmente, porque limitaría la eficiencia de los algoritmos). Para resolverlo, se utilizará un transformador denominado \texttt{OneHotEncoder}. Este transformador reemplaza una característica categórica con varias características booleanas, de forma que para cada posible valor de la categoría se crea una nueva columna con un valor de sí o no dependiendo de a cual pertenece cada dato. Para simplificar, características categóricas con más de diez valores distintos posibles serán descartadas completamente. Lo mismo ocurrirá con cualquier otra columna que no pueda ser identificada como numérica o categórica y con las filas en las que falten datos de tipo categórico.

Un último transformador, \texttt{StandardScaler}, será aplicado a las características numéricas para alinear la escala de todas ellas mediante una técnica denominada normalización de características. Este proceso consiste en transformar las características numéricas para que se sitúen dentro de un rango común, generalmente entre 0 y 1 o para que tengan media cero y varianza uno, como se hace con el \texttt{StandardScaler}. Esto ayuda a obtener mejores resultados en los modelos de aprendizaje automático por dos razones. Primero, evita que características con valores grandes dominen a aquellas con valores más pequeños, asegurando que todas las características contribuyan de manera equitativa al modelo. Segundo, algunos algoritmos, como los basados en distancias (por ejemplo, vecinos más cercanos o máquinas de vector soporte), funcionan mejor y convergen más rápido cuando las características están en la misma escala.


\section{Entrenamiento}

Una vez que los datos están preparados para su uso comienza la fase de entrenamiento. En este proyecto, el objetivo es medir el gasto energético de distintos modelos y compararlo con la calidad de sus predicciones. Para ello, se han elegido una serie de modelos representativos de las familias de algoritmos más utilizadas. Estos modelos elegidos serán entrenados uno detrás de otro con el conjunto de datos preparado mientras se mide el consumo energético mediante las herramientas proporcionadas por CodeCarbon.

El procedimiento es sencillo. En primer lugar se comienzan las mediciones mediante la creación se un objeto de tipo \texttt{EmissionsTracker} que cuenta con simples métodos \texttt{start()} y \texttt{stop()}. A continuación, se entrena el modelo en la parte del conjunto de datos reservada para entrenamiento, y seguidamente se aplica el modelo entrenado a la parte del conjunto de datos reservada para testeo para intentar predecir correctamente la etiqueta de cada entrada que contiene. Para finalizar, se detiene la medición de emisiones y se almacenan los resultados obtenidos.


\subsection{Modelos escogidos}
\label{subsec:models-short}

Dentro de los modelos de aprendizaje automático existen numerosas clasificaciones de acuerdo al tipo de tareas a realizar y la naturaleza de los datos. Este proyecto se centrará en tareas de clasificación por aprendizaje supervisado. En este subgrupo, destacan una serie de familias de algoritmos que suelen obtener buenos resultados para una gran variedad de tipos de datos.
\begin{enumerate}
    \item Modelos lineales. Se trata de modelos sencillos, fáciles de interpretar y eficientes computacionalmente, que funcionan bien cuando las relaciones entre las características de entrada y la salida son aproximadamente lineales. Uno de sus modelos más representativos es el de regresión logística (ver \ref{subsec:model-linear}).
    \item Árboles decisores. Estos algoritmos pueden modelar relaciones complejas en los datos y lidiar con no linealidad. Dentro de esta familia destaca el modelo de Bosque Aleatorio, que agrega predicciones de varios árboles decisores para mejorar la robustez del modelo (ver \ref{subsec:model-random-forest}).
    \item Máquinas de vector soporte (\ref{subsec:model-svm}). Estos modelos son efectivos tanto en tareas de clasificación lineales como no lineales y destacan por su gran versatilidad. Funcionan de forma óptima en conjuntos de datos relativamente pequeños pero de gran complejidad.
    \item Vecinos más cercanos. Su máximo representante, k vecinos más cercanos (k-NN, ver \ref{subsec:model-neighbors}), es un algoritmo simple e intuitivo que se basa en buscar relaciones locales entre los datos y puede ser efectivo en tareas tanto de regresión como de clasificación.
    \item Naive Bayes (bayesiano ingenuo). Se trata de una familia de modelos que destaca por su gran eficiencia, especialmente frente a conjunto de datos de gran complejidad, y especialmente útil en tareas de clasificación de texto. El clasificador más representativo es el Naive Bayes gaussiano (\ref{subsec:model-naive-bayes}).
    \item Métodos de conjuntos (ensemble). Estos métodos combinan las características de múltiples modelos para intentar mejorar el desempeño total. Entre ellos destacan las máquinas de potenciación de gradiente, que construyen sólidos modelos predictivos de forma iterativa (ver \ref{subsec:model-gradient}).
    \item Redes neuronales. Las redes neuronales, especialmente los modelos de aprendizaje profundo como las redes neuronales profundas (Deep Neural Networks, DNN, ver \ref{subsec:model-neural}), pueden formar representaciones jerárquicas complejas a partir de los datos, y son especialmente efectivas en tareas que involucran grandes cantidades de datos con patrones complejos.
\end{enumerate}

En general, se espera que modelos de aprendizaje más complejos como los métodos de conjuntos, las redes neuronales y las máquinas de vector soporte produzcan mejores predicciones a cambio de un mayor gasto energético que otros modelos comparativamente más sencillos computacionalmente, como los modelos lineales, Naive Bayes y vecinos más cercanos. En el capitulo~\ref{chap:experimentos}, se analizará si los resultados obtenidos durante los experimentos se corresponden con esta aproximación teórica.


\section{Registro de resultados}

\subsection{Evaluación de las predicciones}
\label{sec:scoring}

Para poder obtener una medida de utilidad de los distintos modelos, es necesario evaluar la calidad de las predicciones que realizan. Existen dos métodos principales realizar esta evaluación. El primero y más sencillo consiste en aplicar la función de predicción del modelo a un subconjunto de muestras que hayan sido aisladas previamente para no formar parte del proceso de entrenamiento. Estas predicciones se comparan con las etiquetas correctas de las muestras para determinar si cada predicción ha sido acertada.
Cuatro resultados distintos son posibles por muestra y clase concreta: verdadero positivo (TP, identificada correctamente como perteneciente a la clase), falso positivo (FP, identificada incorrectamente como perteneciente a la clase), falso negativo (FN, identificada incorrectamente como no perteneciente a la clase), y verdadero negativo (TN, identificada correctamente como no perteneciente a la clase). Una forma común de visualizar estos resultados es mediante una matriz de confusión, como la mostrada en la figura~\ref{fig:confusion-matrix}.

\begin{figure}[H]
  \centerline{
     \includegraphics[width=0.8\textwidth, keepaspectratio]{img/confusion-matrix.jpg}
  }
  \caption{Matriz de confusión que muestra las relaciones entre posible resultados de la predicción.}
  \label{fig:confusion-matrix}
\end{figure}

A partir de las relaciones entre el número de muestras en cada uno de estos grupos se pueden extraer varias métricas del modelo, siendo las más comunes la exactitud, la precisión, la exhaustividad y el \emph{f-score}\cite{scikit-model-eval}. Estas métricas dan lugar a un valor entre 0 y 1, donde 0 es el peor resultado y 1 el mejor. También son comúnmente expresadas en porcentaje.

La \textbf{exactitud} (\emph{accuracy}) se define como la cercanía de la predicciones a su valor real. En tareas de clasificación se calcula como el número de predicciones correctas entre el número de muestras totales, como se puede ver en la ecuación \ref{eq:accuracy}. Una variante interesante de la exactitud que \texttt{scikit-learn} permite calcular es la exactitud balanceada, que evita medidas infladas de exactitud en conjuntos de datos no balanceados (con una o más clases sobrerrepresentadas en el conjunto) mediante la ponderación de cada muestra de acuerdo a la prevalencia inversa de su verdadera clase.

\begin{equation}
    a = \dfrac{TP+TN}{\text{muestras totales}}
\label{eq:accuracy}
\end{equation}

La \textbf{precisión} (\emph{precision}) y la \textbf{exhaustividad} (\emph{recall}) son métricas que analizan la relevancia de las muestras asignadas a cada clase y se calculan individualmente por clase. La precisión analiza el número de muestras correctamente clasificadas dentro de todas las muestras asignadas a una clase concreta, como muestra la ecuación \ref{eq:precision}, mientras que la exhaustividad analiza el número de muestras correctamente clasificadas en relación al número total de muestras reales existentes, como muestra la ecuación \ref{eq:recall}. Estas dos métricas pueden ser promediadas en función del peso relativo de cada clase un el conjunto para obtener una medida global de la precisión y exhaustividad del modelo.

\noindent
\begin{tabular}{@{}p{.4\linewidth}@{}p{.6\linewidth}@{}}
  \begin{equation}
     p = \dfrac{TP}{TP + FP}
  \label{eq:precision}
  \end{equation}
  &
  \begin{equation}
    r = \dfrac{TP}{TP + FN}
  \label{eq:recall}
  \end{equation}
\end{tabular}

En último lugar, es interesante mencionar el valor-F (\emph{F-score}), que se puede interpretar como una media harmónica de la precisión y la exhaustividad y tiene distintas variantes dependiendo de la importancia relativa de estas dos medidas. La denominada medida-$F_1$ da la misma importancia a la precisión y a la exhaustividad. Para cada clase, se calcula como muestra la ecuación \ref{eq:f-score}.

\begin{equation}
    F_1 = \dfrac{2\cdot TP}{2TP + FP + FN} = \dfrac{2pr}{p+r}
\label{eq:f-score}
\end{equation}

Estas cuatro medidas son calculadas por defecto en la aplicación desarrollada para todos los modelos entrenados. Para todos los casos, se utiliza la opción de scikit-learn \texttt{average='weighted'} para obtener las medidas, de forma que se puedan obtener valores más realistas en conjuntos con clases no balanceadas.

\subsubsection{Validación cruzada}

La aplicación desarrollada ofrece un segundo método para evaluar el rendimiento de un modelo mediante validación cruzada. Este método se basa en las mismas métricas ya mencionadas, pero con la peculiaridad de que éste se entrena varias veces de forma sucesiva con distintas distribuciones de los datos en un conjunto de entrenamiento y un conjunto de prueba. Posteriormente, se puede analizar la media y la desviación estándar de las métricas de la calidad de las predicciones obtenidas para cada distribución de las muestras y así obtener una idea más exacta del desempeño del modelo. En \texttt{scikit-learn}, la validación cruzada se puede implementar mediante el uso de la clase \texttt{KFold}, que divide el conjunto de datos en un número de pliegues especificados de forma aleatoria, y la función \texttt{cross\_validate}, que entrena el modelo y calcula las métricas deseadas para cada uno de los pliegues de forma sucesiva. Para determinar los plieges, la aplicación MLCost utiliza una clase derivada llamada \texttt{StratifiedKFold}, que ayuda a mantener el mismo ratio de muestras por clase en cada pliegue para conjuntos de datos no balanceados.

La validación cruzada ofrece varias ventajas significativas al evaluar el rendimiento de un modelo de aprendizaje automático. Una de las principales ventajas es que proporciona una estimación más robusta y fiable del desempeño del modelo al utilizar diferentes subconjuntos del conjunto de datos para entrenamiento y prueba en cada iteración. Esto ayuda a mitigar el riesgo de sobreajuste y ofrece una visión más generalizada de cómo se comportará el modelo con datos no vistos. Además, calcular la media y la desviación estándar de las métricas a través de los distintos pliegues permite una evaluación más precisa y detallada, identificando variaciones y asegurando que el modelo no solo es preciso sino también consistente.

Sin embargo, la validación cruzada también tiene desventajas. Uno de los principales inconvenientes es el aumento significativo en el tiempo de cómputo, ya que el modelo debe entrenarse y evaluarse múltiples veces, lo cual puede ser particularmente costoso en términos de tiempo y consumo energético, especialmente para conjuntos de datos grandes o modelos complejos. La utilización de un entorno de paralelización puede mitigar esta desventaja al permitir que los pliegues se procesen en paralelo, reduciendo así el tiempo total necesario para completar la validación cruzada. No obstante, la paralelización puede no ser igualmente efectiva en todos los tipos de hardware. Además, en entornos compartidos o con recursos limitados, la paralelización podría causar conflictos de recursos, afectando negativamente el rendimiento global del sistema. Por lo tanto, aunque la paralelización puede acelerar significativamente la validación cruzada, es crucial evaluar su uso en conjunción con las capacidades y limitaciones del hardware disponible. Este efecto será examinado durante las pruebas llevadas a cabo en la sección~\ref{sec:test-2-resources}.


\subsection{Medición de emisiones}

Durante la ejecución de su rastreador de emisiones, la librería CodeCarbon calcula varias medidas distintas para identificar el consumo energético. En primer lugar, las emisiones están geolocalizadas de una de las siguientes formas: mediante una conexión a internet automática que permita identificar la localización mediante rastreo de IP, o mediante la especificación de un país determinado en el código al crear el objeto rastreador de emisiones. Esta localización es necesaria para convertir los kilovatios consumidos durante el proceso de entrenamiento en emisiones de carbono equivalentes, que dependerán de la mezcla especifica de producción de energía que haya establecido cada país. De esta forma, si la energía estuviera producida en gran medida por energías renovables, las emisiones de carbono serían mucho menores que si la energía fuera producida en su totalidad en una planta de quema de carbón. En la sección de resultados se compararán las diferencias de emisiones producidas entrenando modelos en un mismo conjunto de datos en diferentes localizaciones.

Con estos datos, la librería CodeCarbon calcula las emisiones a partir de medidas de la capacidad del procesador y gráfica de la máquina, del porcentaje de su uso que corresponde al proceso de entrenamiento observado y de la duración total del proceso. En este proyecto, por cada modelo entrenado se guardan tres de estas medidas para su posterior análisis y comparativa: la energía consumida (en kilovatios hora, \unit{kWh}), las emisiones calculadas (en kilogramos equivalentes de carbono, $\unit{kg\;[CO_2eq]}$]) y la duración (en segundos). Durante el proceso de entrenamiento, estos valores son escritos en un archivo de texto de tipo CSV (\emph{comma-separated values}) junto con las medidas de calificación de las predicciones mencionadas en el apartado anterior. Este archivo de texto será utilizado posteriormente para dibujar gráficas de las que extraer conclusiones con una herramienta de gráficos desarrollada para este proyecto.


\subsection{Herramientas de visualización}

La visualización de los resultados obtenidos es crucial para interpretar y analizar las métricas de consumo energético y rendimiento de los modelos de aprendizaje automático evaluados. A través de gráficos, es posible identificar patrones, tendencias y relaciones entre distintas variables que de otra manera serían difíciles de detectar en tablas de datos. Esto facilita la toma de decisiones informadas y permite comunicar los hallazgos de manera más efectiva.

El motor de gráficos utilizado en este proyecto es Matplotlib (ver \ref{subsec:matplotlib}, una biblioteca de Python ampliamente utilizada para la creación de gráficos estáticos, animados e interactivos. Matplotlib proporciona una gran flexibilidad y control sobre la generación de gráficos, permitiendo a los desarrolladores personalizar todos los aspectos visuales de sus representaciones gráficas.

Para generar los gráficos, es necesario haber ejecutado previamente una medición de emisiones con la opción \texttt{--log} activada. Esto crea un archivo CSV que contiene los datos recopilados durante las pruebas, incluyendo métricas de rendimiento y consumo energético. Este archivo CSV es esencial para la visualización, ya que contiene toda la información requerida para producir los gráficos. Los gráficos se generan utilizando el comando \texttt{mlcost show -f <output-file.csv>}. Este comando lee el archivo CSV especificado y produce una variedad de gráficos que ayudan a visualizar los resultados de las pruebas.

El módulo \texttt{graph} incluye varias funciones auxiliares para la creación de diferentes tipos de gráficos. Entre ellas se encuentran funciones para generar gráficos de dispersión con tres o cuatro variables, así como gráficos de líneas y barras. Las funciones de dispersión permiten comparar variables como emisiones y precisión a través de diferentes modelos y conjuntos de datos, utilizando diferentes marcadores y colores para distinguir entre categorías. Por otro lado, los gráficos de líneas y barras se utilizan para visualizar tendencias y comparaciones categóricas, aprovechando las capacidades de la librería Pandas para agrupar los datos recogidos en \texttt{DataFrames} y producir gráficas con ellos de manera eficiente.

\section{Gestión de recursos}

Un aspecto importante para obtener mediciones precisas del consumo energético y del rendimiento de los modelos de aprendizaje automático es la gestión de los recursos disponibles en la máquina que está tomando las medidas. Los resultados obtenidos por la librería CodeCarbon para estimar el consumo eléctrico se basan en el modelo y el tiempo de utilización del procesador. De esta forma, CodeCarbon rastrea el uso de recursos durante el entrenamiento de los modelos y calcula las emisiones de $CO_2$ correspondientes.

La carga del procesador en el momento de tomar las medidas de consumo energético es un factor crucial. Una alta carga del procesador puede indicar que el sistema está ejecutando múltiples tareas simultáneamente, lo cual puede sesgar los resultados. Para mitigar este efecto, CodeCarbon intenta aislar el proceso de aprendizaje del resto de tareas del ordenador al utilizar la opción {\texttt{tracking\_mode="process"} en su clase medidora de emisiones \texttt{EmissionTracker}. Esta opción permite que la herramienta enfoque la toma de medidas en el proceso específico que se está evaluando, minimizando la interferencia de otras operaciones del sistema. 

Sin embargo, este enfoque añade una capa adicional de aproximación al cálculo de emisiones, por lo que para obtener medidas más consistentes será interesante mantener una carga baja del procesador al comenzar el proceso. Al inicio de la aplicación, ésta imprime al terminal tanto la información del sistema y del modelo del procesador como su carga de trabajo inicial en porcentaje de utilización. Además, el archivo de datos recopilados generado por el programa contiene una columna con la carga al finalizar el entrenamiento de cada modelo. Una carga baja al comienzo del experimento significará que la máquina no está realizando otras tareas pesadas que podrían influir en las mediciones de consumo energético y rendimiento. Esto ayudará a que los datos reflejen de manera más precisa el impacto del entrenamiento del modelo en el consumo energético.


\subsection{Procesamiento multinúcleo}

El uso de ordenadores con varios núcleos en el procesador o de varios procesadores asignados a la misma tarea puede afectar tanto al rendimiento como a las emisiones. En términos de rendimiento, la capacidad de realizar múltiples tareas simultáneamente (\emph{multitasking}) permite que los modelos se entrenen más rápido, ya que pueden aprovechar el paralelismo inherente a muchos de los algoritmos de aprendizaje automático. Sin embargo, esta mejora en el rendimiento puede venir acompañada de un aumento en el consumo energético, ya que más núcleos en uso implican un mayor consumo de energía.

La biblioteca scikit-learn gestiona el paralelismo a través del parámetro \texttt{n\_jobs}, el cual se puede especificar en diversas funciones y modelos para indicar el número de procesos paralelos que se deben utilizar. La aplicación desarrollada acepta este parámetro al ejecutarla por línea de comandos y lo comunica a scikit-learn, permitiendo que se configure la cantidad de CPUs disponibles para ejecutar tareas en paralelo. Esta configuración puede reducir significativamente el tiempo de cómputo en experimentos de gran escala.

Internamente, scikit-learn utiliza el contexto {joblib.parallel\_backend} para gestionar el paralelismo. {joblib.parallel\_backend} es una función que permite seleccionar el backend de paralelización que se utilizará durante la ejecución de las operaciones paralelas. El backend de joblib puede manejar diferentes tipos de paralelismo, como threading y multiprocessing, adaptándose a las características del hardware y a las necesidades del usuario. Cuando se especifica \texttt{n\_jobs}, {joblib.parallel\_backend} se encarga de dividir las tareas entre los procesos disponibles y de gestionar la sincronización y la recolección de resultados.

Dentro del proceso de aprendizaje, hay varias funciones en las que scikit-learn ofrece la posibilidad de ejecutar tareas en paralelo. Una de ellas es la validación cruzada, que es particularmente adecuada para la paralelización, ya que cada pliegue del conjunto de datos puede ser procesado de manera independiente. Por otra parte, varios modelos pueden ser entrenados directamente en paralelo dentro de una única iteración especificando el número de procesos a utilizar. Entre los modelos que admiten \texttt{n\_jobs} se encuentran el bosque aleatorio, las máquinas de potenciación de gradiente, y vecinos más cercanos. Sin embargo, otros modelos como los lineales, las máquinas de vector soporte, Naive Bayes, y las redes neuronales no siempre permiten especificar un número de procesos en paralelo de forma nativa.

Para lidiar con estas diferencias, la aplicación desarrollada se limita a aplicar el número de procesos únicamente a la validación cruzada y no al entrenamiento directo de los modelos. Esta decisión se toma para poder comparar todos los modelos en igualdad de condiciones. De esta forma, se asegura que cualquier mejora en el tiempo de ejecución sea atribuible únicamente a la paralelización del proceso de validación cruzada y no a diferencias intrínsecas en la implementación de cada modelo.

El efecto que la paralelización y la gestión de los recursos de la máquina donde se realiza el entrenamiento se podrá observar en el experimento de la sección \ref{sec:test-2-resources}. En esta prueba, 
se entrenarán los modelos en diferentes máquinas virtuales desplegadas en Microsoft Azure, cada una con configuraciones de procesador y memoria RAM distintas y alternando entre utilizar paralelización o no durante el entrenamiento. Este experimento permitirá estudiar los resultados de emplear varios procesadores en el entrenamiento por validación cruzada, analizando cómo el paralelismo y la configuración de hardware afectan tanto al rendimiento de los modelos como al consumo energético.

% \todoin{TO-DO EXPLICAR \\
% > Azure deployment
% > plantillas }

\clearpage
\chapter{Experimentos y validación}
\label{chap:experimentos}

El objetivo de este capítulo es mostrar el funcionamiento de la aplicación en un caso de uso real en el que se tratará de extraer conclusiones generales acerca del consumo eléctrico de cada modelo y de si este consumo irá necesariamente acompañado de una mejora de los resultados de predicción.
Para ello se emplearán las herramientas descritas anteriormente para evaluar el consumo y el rendimiento de una serie de modelos formada por representantes de las principales familias de modelos de aprendizaje automático y recogidos en la sección~\ref{sec:models}. Estos modelos serán aplicados a los conjuntos de datos de distintas características definidos en la sección~\ref{sec:datasets}.

Durante la validación de la aplicación se llevarán acabo tres experimentos distintos.
El primero examinará el consumo energético en base al modelo seleccionado. En esta sección se tomarán varias medidas de consumo y rendimiento por modelo y conjunto de datos en una máquina con unos recursos de procesamiento concretos para analizar que modelos consumen más que otros y que características de los conjuntos de datos hacen incrementar este consumo.
El segundo consistirá en aislar un par de conjuntos de datos y tomar medidas de consumo con distintos recursos de procesado dedicados a la tarea de aprendizaje automático para observar el efecto de los recursos disponibles en el consumo energético de cada modelo.
Por último, se propondrán métodos de optimización de los modelos analizados y se examinará el efecto que pueda tener sobre su consumo. 

A través de este análisis, se pretende obtener una comprensión profunda de cómo diferentes modelos de aprendizaje automático consumen energía bajo diversas condiciones de trabajo. Este experimento también busca identificar patrones de consumo y eficiencia que puedan informar el diseño y la implementación de modelos más sostenibles y eficientes en el futuro.

\todo[inline]{Añadir esquema ???}
% What's the purpose of experiments?
% What are the expected results? More energy, more precision
% Outline / procedure / steps to follow

% 4.1 Análisis del consumo energético en base al modelo escogido
    % 4.1.1 Comparación en conjuntos de pequeño tamaño (100s - 1000s)
        % Tres conjuntos: iris, ionosphere, hepatitis
    % 4.1.2 Comparación en conjuntos de mediano tamaño (10000s)
        % Dos conjuntos: eeg-eye-state, electricity, letter, mnist_784
% 4.2 Análisis del consumo energético en base a los recursos disponibles
    % 4.2.1 Evolución del consumo con la carga del procesador
        % 1 dataset pequeño, 1 mediano
    % 4.2.2 Evolución del consumo con el aumento de recursos
        % 1 dataset grande (100000s) ?covertype?, 2-3 resource configs
% 4.3 Optimización

\section{Consumo energético basado en el modelo seleccionado}
 % - El consumo aumenta al aumentar el número de muestras
 %    1. Gráfico introductorio: número de muestras (x) vs emisiones (y), muchos modelos
 %        el consumo aumenta de forma exponencial con el número de muestras, unos modelos aumentan más que otros
 %    2. Introduce f-score: plot same lines with average f1-score instead of emissions. Los resultados son distintos, mayor consumo no implica mejor predicción
 %    3. Introduce scatter plot 4-way
    
 % - Algunos modelos son mejores que otros
 %    - Aumento de score implica aumento de consumo?
 %    - Compara average f1-score con consumo por modelo y dataset
 %    3. Introduce f-score con scatter plot 4-way, all models, 3 datasets (no average)
 %    4. Bar plot de dos datasets pequeños comparando score 

\subsubsection{Objetivos}

En esta sección se examinará el consumo energético una serie de modelos representativos aplicados a varios conjuntos de datos. El objetivo de este análisis será abordar las siguientes cuestiones clave:

\begin{itemize}
    \item Identificación de los modelos con mayor consumo energético.
    \item Determinación de los modelos cuyo consumo energético incrementa significativamente al aumentar el número de muestras.
    \item Evaluación de modelos que ofrecen mejores predicciones con menor consumo energético.
\end{itemize}

Dónde sea posible, se tratará de analizar estas cuestiones de forma general y obtener conclusiones que sean extrapolables más allá de los conjuntos de datos concretos que se hayan medido. Sin embargo, debido a la gran cantidad de variables involucradas en las variaciones de consumo entre unos casos y otros, es posible en otros conjuntos de datos se observen comportamientos distintos del consumo.

\subsubsection{Metodología}

Para analizar estas cuestiones todas las medidas de consumo serán tomadas con la aplicación desarrollada ejecutando en una misma máquina. Para cada modelo y conjunto de datos, se tomarán medidas de consumo y rendimiento utilizando validación cruzada con cinco iteraciones con un tamaño definido para los datos de testeo del 20\% del conjunto de datos. Esta técnica proporcionará una evaluación robusta y precisa tanto del comportamiento energético de los modelos como de su precisión y exactitud, ya que evitará en gran medida la presencia de valores atípicos y el riesgo de sobreajuste de los modelos.

\begin{table}[h]
    \centering
    \begin{tabular}{rl}
         Modelo & Dell XPS 15 9500\\
         Sistema Operativo & Ubuntu 20.04.6 LTS x86\_64\\
         Python & 3.12.2\\
         Procesador & Intel(R) Core(TM) i9-10885H CPU @ 2.40GHz\\
         Memoria & 7,63 GB\\
    \end{tabular}
    \caption{Características técnicas de la máquina utilizada para tomar las medidas}
    \label{tab:caracteristicas-tecnicas}
\end{table}

La aplicación será ejecutada con el siguiente comando para cada conjunto de datos distinto, en el cual \texttt{[dataset]} será sustituido por el archivo que contenga cada conjunto de datos. Adicionalmente, cualquiera de las opciones de lectura de datos descritas en la sección~\ref{sec:limpieza} podrá ser utilizada si el formato en el que se encuentren los datos lo requiere. Las características de la máquina utilizada están recogidas en la tabla~\ref{tab:caracteristicas-tecnicas}.
\begin{minted}{bash}
mlcost measure --log -cv 5 -d [dataset] [dataset-options]
\end{minted}

\subsubsection{Resultados}

La ejecución del comando anterior producirá un archivo tipo tabla de datos en formato \texttt{.csv}. La tabla~\ref{tab:medidas-1} recoge un extracto de los resultados obtenidos en el ordenador de referencia para seis conjuntos de datos distintos. El archivo completo está disponible en el repositorio de la aplicación.

\begin{table}[H]
\centerline{
\scalebox{0.78}{
\begin{tabular}{|llllllllllll|}
\hline
Dataset     & Modelo & CPU & Accuracy & Precision & F-score & Recall & Fit  & Total (s) & Emisiones & Energía  & Muestras \\
 &  & load (\%) &  & & & &  time (s) & &  (kg) &  (kWh) &  \\ \hline
Banknote    & Linear & 2.7           & 0.98      & 0.98      & 0.98    & 0.98          & 0.007             & 0.071            & 2.13E-07  & 1.10E-06 & 1372     \\
Banknote    & Linear & 2.7           & 0.97      & 0.97      & 0.97    & 0.97          & 0.006             & 0.071            & 2.13E-07  & 1.10E-06 & 1372     \\
Banknote    & Linear & 2.7           & 0.97      & 0.97      & 0.97    & 0.97          & 0.006             & 0.071            & 2.13E-07  & 1.10E-06 & 1372     \\
Banknote    & Linear & 2.7           & 0.99      & 0.99      & 0.99    & 0.99          & 0.005             & 0.071            & 2.13E-07  & 1.10E-06 & 1372     \\
Banknote    & Linear & 2.7           & 0.99      & 0.99      & 0.99    & 0.99          & 0.005             & 0.071            & 2.13E-07  & 1.10E-06 & 1372     \\
Banknote    & Forest & 2.7           & 0.99      & 0.99      & 0.99    & 0.99          & 0.184             & 1.429            & 3.27E-06  & 1.69E-05 & 1372     \\
Banknote    & Forest & 2.7           & 1.00      & 1.00      & 1.00    & 1.00          & 0.171             & 1.429            & 3.27E-06  & 1.69E-05 & 1372     \\
Banknote    & Forest & 2.7           & 0.99      & 0.99      & 0.99    & 0.99          & 0.154             & 1.429            & 3.27E-06  & 1.69E-05 & 1372     \\
Banknote    & Forest & 2.7           & 1.00      & 1.00      & 1.00    & 1.00          & 0.172             & 1.429            & 3.27E-06  & 1.69E-05 & 1372     \\
Banknote    & Forest & 2.7           & 1.00      & 1.00      & 1.00    & 1.00          & 0.158             & 1.429            & 3.27E-06  & 1.69E-05 & 1372     \\
\multicolumn{12}{|c|}{...} \\
Electricity & Neural & 102.4         & 0.82      & 0.83      & 0.83    & 0.83          & 132.768           & 518.905          & 1.19E-03  & 6.15E-03 & 45312 \\  \hline
\end{tabular}}}
\caption[Extracto de los resultados de entrenamiento]{Extracto de los resultados de entrenamiento\footnote{\url{https://github.com/l-gonz/tfg-gitt-mlcost/blob/main/model-comp-many.csv}}
\todo[inline]{Fix header format}}
\label{tab:medidas-1}
\end{table}

Cada fila en la tabla corresponde a las medidas tomadas durante una iteración de entrenamiento de cada modelo por validación cruzada. Al haber escogido utilizar validación cruzada de cinco iteraciones, la tabla de resultados cuenta con cinco filas por modelo y conjunto de datos. Sin embargo, algunas de las medidas, como la carga del procesador, el número de muestras del conjunto de datos, el tiempo total de entrenamiento, las emisiones del proceso y la energía consumida, son tomadas de forma global al finalizar todas las iteraciones de entrenamiento de cada modelo y conjunto.
Para cada iteración individual se recogen las medidas estadísticas de exactitud, precisión, exhaustividad y valor-F calculadas. Además, la implementación de validación cruzada de \texttt{scikit-learn} proporciona una medida del tiempo de entrenamiento empleado en cada iteración (fit time). Este valor puede ser utilizado junto con las emisiones y el tiempo totales de todas las iteraciones para calcular las emisiones de cada iteración de entrenamiento como muestra la ecuación~\ref{eq:fit-emissions}.

\begin{equation}
    E_1 = \frac{E_T}{t_T} \cdot t_1
    \label{eq:fit-emissions}
\end{equation}
\begin{conditions}
E_1   &   emisiones de la iteración \\
E_T   &   emisiones totales \\
t_T   &   tiempo total \\
t_1   &   tiempo de entrenamiento de la iteración
\end{conditions}

A partir de los resultados obtenidos se puede dibujar un diagrama de dispersión para visualizar cómo varían las emisiones. En la figura~\ref{fig:scatter-1} se ha utilizado el valor-F como medida de la calidad de las predicciones (eje Y), ya que es habitualmente más informativa en casos de distribuciones de clases no balanceadas. En el eje X, se han dibujado las emisiones con una escala logarítmica.

\begin{figure}[H]
  \centerline{
     \includegraphics[width=1.3\textwidth, keepaspectratio]{img/graph/4scatter-dataset-model.png}
  }
  \caption{Valor-F alcanzado por el modelo frente a las emisiones de carbono necesarias para entrenarlo, por modelo empleado y conjunto de datos utilizado}
  \label{fig:scatter-1}
\end{figure}
\todo[inline]{Fix plot titles}

% Scatter plot bla bla bla
- Outlier eeg-eye-state lower score

- Neural, higher emissions, average score, high variance, better score more complex dataset
- Support vector, starts well, but low score for eye and very high emissions for electricity

- Forest, high score, medium emissions, even eye
- Gradient, same but a little worse on both

- Neighbors, very low emissions

- Linear, low emissions, low score
- Naive bayes, very low score, low emissions


\begin{figure}[H]
  \centerline{
     \includegraphics[width=1\textwidth, keepaspectratio]{img/graph/line-nsamples-emission-log.png}
  }
  \caption{Evolución de las emisiones de carbono con el aumento de número de muestras del conjunto de datos}
  \label{fig:line-samples}
\end{figure}
\todo[inline]{Fix plot titles}


\clearpage
\chapter{Conclusiones y trabajos futuros}
\label{chap:conclusiones}


\section{Consecución de objetivos}
\label{sec:consecucion-objetivos}


\section{Aplicación de lo aprendido}
\label{sec:aplicacion}


\section{Lecciones aprendidas}
\label{sec:lecciones_aprendidas}


\section{Trabajos futuros}
\label{sec:trabajos_futuros}



% GLOSARIO(S) %
\printglossary[type=\acronymtype]
\printglossary

% APÉNDICE(S) %
\cleardoublepage
\appendix
\chapter{Output del programa}

\section{Interfaz de comandos de la aplicación}
\label{app:cli}

\begin{minted}[breaklines, fontsize=\footnotesize, baselinestretch=1]{text}
Usage: mlcost measure [OPTIONS]

Options:
  -d, --dataset FILE             filepath to dataset, uses Iris dataset if
                                 none given. If using the --openml option, it
                                 is the dataset id in openml
  -l, --labels TEXT              labels column, defaults to last column
  -t, --test FILE                filepath to test set, if none will get split
                                 test set from dataset
  -s, --separator TEXT           separator, defaults to a comma (no spaces)
  -f, --codecarbon-file TEXT     filename for default output from codecarbon,
                                 can be used with codecarbon's own
                                 visualization tool
  -m, --model TEXT               specify one model to run, default is to run
                                 everything
  -cv, --cross-validate INTEGER  cross validate the model, specify the number
                                 of folds (default is 1, no cross validation)
  --online                       use Codecarbon Emission Tracker in online
                                 mode
  --log                          output to additional csv file with whole
                                 experiment data
  --no-header                    do not consider the first row in the data to
                                 be the header
  --openml                       fetch dataset from openml
  --parallel                     parellelize cross validation between all
                                 cores, needs -cv option
  -h, --help                     Show this message and exit.


Usage: mlcost plot [OPTIONS]

Options:
  -f, --file FILE  filepath to csv file that contains the data  [required]
  -h, --help       Show this message and exit.
\end{minted}

\section{Resultados de la limpieza y preprocesado de los conjuntos de datos}
\label{app:preprocessing}

\subsection{Iris}
\begin{minted}[tabsize=2,breaklines]{text}
$ mlcost measure --log -cv 5

---------------------------
DATA PREPROCESSING SUMMARY
Original data: 4.932e+03 bytes

Discarded features: 0
Discarded rows for missing labels: 0
Trained numerical features: ['sepal length (cm)', 'sepal width (cm)', 'petal length (cm)', 'petal width (cm)']
Trained categorical features: []

Removed rows from missing categorical values - Train: 0 , Test: 0
Final train set rows: 120, test set rows: 30

Target distribution:
        train      test
target                 
0       0.350  0.266667
1       0.325  0.366667
2       0.325  0.366667
---------------------------
\end{minted}

\subsection{Ionosfera}
\begin{minted}[tabsize=2,breaklines]{text}
$ mlcost measure --log -cv 5 -d data/ionosphere/ionosphere.data --no-header

---------------------------
DATA PREPROCESSING SUMMARY
Original data: 9.560e+04 bytes

Discarded features: 0
Discarded rows for missing labels: 0
Trained numerical features: [0, 1, 2, 3, 4, 5, 6, 7, 8, 9, 10, 11, 12, 13, 14, 15, 16, 17, 18, 19, 20, 21, 22, 23, 24, 25, 26, 27, 28, 29, 30, 31, 32, 33]
Trained categorical features: []

Removed rows from missing categorical values - Train: 0 , Test: 0
Final train set rows: 280, test set rows: 71

Target distribution:
       train      test
34                    
g   0.628571  0.690141
b   0.371429  0.309859
---------------------------
\end{minted}

\subsection{Autenticación de billetes}
\begin{minted}[tabsize=2,breaklines]{text}
$ mlcost measure --log -cv 5 -d data/banknote/banknote.txt --no-header

---------------------------
DATA PREPROCESSING SUMMARY
Original data: 4.404e+04 bytes

Discarded features: 0
Discarded rows for missing labels: 0
Trained numerical features: [0, 1, 2, 3]
Trained categorical features: []

Removed rows from missing categorical values - Train: 0 , Test: 0
Final train set rows: 1097, test set rows: 275

Target distribution:
      train      test
4                    
0  0.559708  0.538182
1  0.440292  0.461818
---------------------------
\end{minted}

\subsection{Fonemas}
\begin{minted}[tabsize=2,breaklines]{text}
$ mlcost measure --log -cv 5 -openml -d phoneme

---------------------------
DATA PREPROCESSING SUMMARY
Original data: 2.163e+05 bytes

Discarded features: 0
Discarded rows for missing labels: 0
Trained numerical features: ['V1', 'V2', 'V3', 'V4', 'V5']
Trained categorical features: []

Removed rows from missing categorical values - Train: 0 , Test: 0
Final train set rows: 4323, test set rows: 1081

Target distribution:
          train      test
Class                    
1      0.702984  0.720629
2      0.297016  0.279371
---------------------------
\end{minted}

\subsection{Electroencefalograma}
\begin{minted}[tabsize=2,breaklines]{text}
$ mlcost measure --log -cv 5 -openml -d eeg-eye-state

---------------------------
DATA PREPROCESSING SUMMARY
Original data: 1.678e+06 bytes

Discarded features: 0
Discarded rows for missing labels: 0
Trained numerical features: ['V1', 'V2', 'V3', 'V4', 'V5', 'V6', 'V7', 'V8', 'V9', 'V10', 'V11', 'V12', 'V13', 'V14']
Trained categorical features: []

Removed rows from missing categorical values - Train: 0 , Test: 0
Final train set rows: 11984, test set rows: 2996

Target distribution:
          train      test
Class                    
1      0.550067  0.555741
2      0.449933  0.444259
---------------------------
\end{minted}

\subsection{Electricidad}
\begin{minted}[tabsize=2,breaklines]{text}
$ mlcost measure --log -cv 5 -openml -d electricity

---------------------------
DATA PREPROCESSING SUMMARY
Original data: 2.584e+06 bytes

Discarded features: 0
Discarded rows for missing labels: 0
Trained numerical features: ['date', 'period', 'nswprice', 'nswdemand', 'vicprice', 'vicdemand', 'transfer']
Trained categorical features: ['day']

Removed rows from missing categorical values - Train: 0 , Test: 0
Final train set rows: 36249, test set rows: 9063

Target distribution:
         train      test
class                   
DOWN   0.57403  0.581154
UP     0.42597  0.418846
---------------------------
\end{minted}

% BIBLIOGRAFIA %
\cleardoublepage
% https://www.overleaf.com/learn/latex/Bibliography_management_with_biblatex
\raggedright\printbibliography[heading=bibintoc,title={Referencias}]

\end{document}
