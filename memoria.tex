%%%%%%%%%%%%%%%%%%%%%%%%%%%%%%%%%%%%%%%%%%%%%%%%%%%%%%%%%%%%%%%%%%%%%%%%%%%%%%%%
%% Plantilla de memoria en LaTeX para TFG/TFM - Universidad Rey Juan Carlos
%%
%% Por Gregorio Robles <grex arroba gsyc.urjc.es>
%%     Felipe Ortega   <felipe.ortega@urjc.es>
%%     Grupo de Sistemas y Comunicaciones (GSyC)
%%     Escuela Técnica Superior de Ingenieros de Telecomunicación
%%     Universidad Rey Juan Carlos
%%
%% (Muchas ideas tomadas de Internet, colegas del GSyC, antiguos alumnos...
%%  etc. Muchas gracias a todos)
%%
%% La última versión de esta plantilla está siempre disponible en:
%%     https://github.com/glimmerphoenix/plantilla-memoria
%%
%% - Ejecución en sistema local:
%% Para obtener el documento en PDF, ejecuta en la shell:
%%   make
%%
%% A diferencia de la anterior versión, que usaba la herramienta pdfLaTeX 
%% para compilar el documento, esta nueva versión de la plantilla usa
%% XeLaTeX. Es un compilador más moderno que, entre otras mejoras, incluye
%% soporte nativo para caracteres con codificación UTF-8, traducción políglota
%% de referencias (usando Biblatex) y soporte para fuentes OTF. Esta última
%% característic permite, por ejemplo, insertar iconos de la colección 
%% Fontawesome en el texto.
%%
%% XeLaTeX viene ya incluido en todas las distribuciones modernas de LaTeX.
%%
%% - Edición y ejecución en línea: 
%% Puedes descargar y subir la plantilla a
%% Overleaf, un editor de LaTeX colaborativo en línea. Overleaf ya tiene
%% instalados todos los paquetes LaTeX y otras dependencias software para
%% que esta plantilla compile correctamente.
%%
%% IMPORTANTE: Si compilas este documento en Overleaf, recuerda cambiar
%% la configuración (botón "Menu" en la esquina superior izquierda de la interfaz)
%% y elegir la opción Compiler --> XeLaTeX. En caso contrario no funcionará.
%%
%% - Nota: las imágenes deben ir en PNG, JPG, EPS o PDF. También se pueden usar
%% imágenes en otros formatos con algunos cambios en el preámbulo del documento.

%%%%%%%%%%%%%%%%%%%%%%%%%%%%%%%%%%%%%%%%%%%%%%%%%%%%%%%%%%%%%%%%%%%%%%%%%%%%%%%%

\documentclass[a4paper, 12pt]{book}

%%-- Geometría principal (dejar activada la siguiente línea en la versión final)
\usepackage[a4paper, left=2.5cm, right=2.5cm, top=3cm, bottom=3cm]{geometry}
%%-- Activar esta línea y comentar la anterior en modo borrador, para comentarios al margen
%\usepackage[a4paper, left=2.5cm, right=2.5cm, top=3cm, bottom=3cm, marginparwidth=60pt]{geometry}

%%-- Hay que cargarlo antes que las traducciones
\usepackage{listing}                    % Listados de código

% Traducciones en XeLaTeX
\usepackage{polyglossia}
\setmainlanguage{spanish}    % Comenta esta línea si tu memoria es en inglés

% Traducciones particulares para español
% Caption tablas
\gappto\captionsspanish{
	\def\tablename{Tabla}
	\def\listingscaption{Código}
	\def\refname{Bibliografía}
	\def\appendixname{Apéndice}
	\def\listtablename{Índice de tablas}
	\def\listingname{Código}
	\def\listlistingname{Índice de fragmentos de código}
}

%% Tipografía y estilos
\usepackage[OT1]{fontenc}               % Keeps eulervm happy about accents encoding

% Símbolos y fuentes matemáticas elegantes: Euler virtual math fonts
% ¡Importante! Carga siempre las fuentes math AMS Euler ANTES QUE fontspec
\usepackage{amsmath}
\usepackage{amssymb}
\usepackage[OT1,euler-digits,euler-hat-accent,small]{eulervm}

% En XeLaTeX las fuentes se especifican con fontspec
\usepackage{fontspec}
\defaultfontfeatures{Scale=MatchLowercase, Ligatures=TeX}     % Default option in font config

% Fix para fuentes usadas con operadores y \mathrm
\DeclareSymbolFont{operators}{\encodingdefault}{\familydefault}{m}{n}

% Configura la fuente principal (serif): MinionPro
\setmainfont[Scale=0.96]{TeX Gyre Pagella}
% Configura la fuente sans-serif (\sffamily)
\setsansfont[Scale=MatchLowercase]{Lato}
% Configura la fuente para letra monoespaciada: Source Code Pro, escala 0.85
\setmonofont[Scale=0.85]{Source Code Pro}

%%-- Familias de fuentes específicas
%%-- Se pueden definir etiquetas para familias de fuentes personalizadas
%%-- que luego puedes emplear para cambiar el formato de una parte de texto
%%-- Ejemplo:
% \newfontfamily{\myriadprocond}{Myriad Pro Semibold Condensed.otf}

%%-- Opciones de interlineado y espacios
\linespread{1.07}                   % Aumentar interlineado para fuentes tipo Palatino
\setlength{\parskip}{\baselineskip} % Separar párrafos con línea en blanco

%%-- Hipervínculos
\usepackage{url}

%%-- Gráficos y tablas
\PassOptionsToPackage{
    dvipdfmx,usenames,dvipsnames,
    x11names,table}{xcolor}             % Definiciones de colores
\PassOptionsToPackage{xetex}{graphicx}

\usepackage{subfig}                     % Subfiguras
\usepackage{pgf}
\usepackage{svg}                        % Integración de imágenes en formato SVG
\usepackage{float}                      % H para posicionar figuras
\usepackage{booktabs}                   % Already loads package xcolor
\usepackage{multicol}                   % multiple column layout facilities
\usepackage{colortbl}                   % For coloured tables

%%-- Bibliografía con Biblatex y Biber
% Más info:
% https://www.overleaf.com/learn/latex/Biblatex_bibliography_styles
% https://www.overleaf.com/learn/latex/biblatex_citation_styles
\usepackage[
    backend=biber,
    style=numeric,
    sorting=nty
    ]{biblatex}
\addbibresource{memoria.bib}
\DeclareFieldFormat{url}{\mkbibacro{URL}\addcolon\nobreakspace\url{#1}}
%\usepackage[nottoc, notlot, notlof, notindex]{tocbibind} %% Opciones de índice

%%-- Matemáticas e ingeniería
% El paquete units permite mostrar unidades correctamente
% Permite escribir unidades con espaciado y estilo de fuente correctos
\usepackage[ugly]{units}         
% Ejemplo de uso: $\unit[100]{m}$ or $\unitfrac[100]{m}{s}$
% Entornos matemáticos
\newtheorem{theorem}{Theorem}

% Paquetes adicionales
\usepackage{url}                        %% Gestión correcta de enlaces
\usepackage{float}                      %% H para posicionar figuras
\usepackage[nottoc, notlot, notlof, notindex]{tocbibind}    %% Opciones de índice
\usepackage{metalogo}                   %% Múltiples logos para XeLaTeX

% Fuentes especiales y glifos
\usepackage{ccicons}                % Creative Commons icons
\usepackage{metalogo}               % XeTeX logo
\usepackage{fontawesome5}           % Fontawesome 5 icons
\usepackage{adforn} 

% Blindtext
% Opciones pangram, bible, random (defecto)
\usepackage[pangram]{blindtext}
% Lorem ipsum
\usepackage{lipsum}
% Kant lipsum
\usepackage{kantlipsum}

\usepackage{fancyvrb}               % Entornos verbatim extendidos
	\fvset{fontsize=\normalsize}    % Tamaño de fuente por defecto en fancy-verbatim
	
% Configura listas (itemize, enumerate) con iconos personalizados
% Fácil reinicio de numeración con enumerate
% Info: http://ctan.org/pkg/enumitem
\usepackage[shortlabels]{enumitem}
% Usar \usageitem para configurar iconos personalizados en listas
\newcommand{\usageitem}[1]{%
	\item[%
	{\makebox[2em]{\strut\color{GSyCblue} #1}}%
	]
}

%%-- Definición de colores personalizados
% \definecolor{LightGrey}{HTML}{EEEEEE}
% \definecolor{darkred}{rgb}{0.5,0,0}     %% Refs. cruzadas
% \definecolor{darkgreen}{rgb}{0,0.5,0}   %% Citas bibliográficas
% \definecolor{darkblue}{rgb}{0,0,0.5}    %% Hiperenlaces ordinarios (también ToC)

%%-- Configuración fragmentos de código
%%-- Minted necesita Python Pygments instalado en el sistema para funcionar
%%-- En Overleaf ya está instalada esta dependencia
% \usepackage[center, labelfont=bf]{caption}
\usepackage{minted}

%%-- Se debe cargar aquí para evitar warnings
\usepackage{csquotes}                   % Para traducciones con biblatex

%%-- Glosario de términos
\usepackage[acronym]{glossaries}
\makeglossaries
\loadglsentries{glossary}

% % Definición de cabeceras del documento, usando fancyhdr
% \usepackage{fancyhdr}
% %% Configuración de cabeceras para el cuerpo principal del documento
% \pagestyle{fancy}
% \fancyhead{}
% \fancyhead[RO,LE]{\myriadprocond{\thepage}}
% \renewcommand{\chaptermark}[1]{\markboth{\chaptername\ \thechapter.\ #1}{}}
% \renewcommand{\sectionmark}[1]{\markright{\thesection.\ #1}}
% \fancyhead[RE]{\myriadprocond{\leftmark}}
% \fancyhead[LO]{\myriadprocond{\rightmark}}
% \renewcommand{\headrulewidth}{0pt}
% \setlength{\headheight}{15pt} %% Al menos 15pt para evitar warning al compilar
% \fancyfoot{}
% %% Configuración para páginas con cabecera en blanco
% \fancypagestyle{plain}{%
% \fancyhf{}% clear all header and footer fields
% \fancyhead[RO,LE]{\myriadprocond{\thepage}}
% \renewcommand{\headrulewidth}{0pt}%
% \renewcommand{\footrulewidth}{0pt}%
% }

%%-- Metadatos del doc
\title{Memoria del Proyecto}
\author{Nombre del autor}

%%-- Hiperenlaces, siempre se carga al final del preámbulo
\usepackage[colorlinks]{hyperref}
\hypersetup{
    pdftoolbar=true,	% Muestra barra de herramientas en Adobe Acrobat
	pdfmenubar=true,	% Muestra menú en Adobe Acrobat
	pdftitle={Título doc en ventana del visor o navegador},
	pdfauthor={Nombre del alumno/a},
	pdfcreator={ETSII/ETSIT, URJC},
	pdfproducer={XeLaTeX},
	pdfsubject={Topic1, Topic2, Topic3},
	pdfnewwindow=true,              %links open in new window
    colorlinks=true,                % false: boxed links; true: coloured links
    linkcolor=Firebrick4,           % enlaces internos 
    citecolor=Aquamarine4,          % enlaces a citas bibliográficas
    urlcolor=RoyalBlue3,            % hiperenlances ordinarios
    linktocpage=true                % Enlaces en núm. pág. en ToC
}

%%%---------------------------------------------------------------------------
% Comentarios en línea de revisión
% Este bloque se puede borrar cuando finalizamos el borrador

\usepackage[colorinlistoftodos]{todonotes}
\usepackage{verbatim}
%%%---------------------------------------------------------------------------

\begin{document}

%%-- Configuración común para todos los entornos listing
%%-- Descomentar para usar y personalizar valores
%\lstset{%
%breakatwhitespace=true,
% breaklines=true, 
% basicstyle=\footnotesize\ttfamily,
% keywordstyle=\color{blue},
% commentstyle=\color{green!40!black}, 
% language=Python} 
 

%%%%%%%%%%%%%%%%%%%%%%%%%%%%%%%%%%%%%%%%%%%%%%%%%%%%%%%%%%%%%%%%%%%%%%%%%%%%%%%%
% PORTADA

\begin{titlepage}
\begin{center}
\begin{tabular}[c]{c c}
%\includegraphics[bb=0 0 194 352, scale=0.25]{logo} &
\includegraphics[scale=1.5]{img/LogoURJC.png}
%&
%\begin{tabular}[b]{l}
%\Huge
%\textsf{UNIVERSIDAD} \\
%\Huge
%\textsf{REY JUAN CARLOS} \\
%\end{tabular}
\\
\end{tabular}

\vspace{3cm}

\Large 
GRADO EN INGENIERÍA EN TECNOLOGÍAS DE LA TELECOMUNICACIÓN

\vspace{0.4cm}

\large
Curso Académico 2021/2022

\vspace{0.8cm}

Trabajo Fin de Grado

\vspace{2cm}

\LARGE Comparativa de modelos de aprendizaje automático con respecto a su consumo energético
\vspace{3cm}

\large
Autor/a : Laura González Fernández \\
Tutor/a : Dr. Nombre del Profesor/a
\end{center}
\end{titlepage}

\newpage
\mbox{}
\thispagestyle{empty} % para que no se numere esta pagina


%%%%%%%%%%%%%%%%%%%%%%%%%%%%%%%%%%%%%%%%%%%%%%%%%%%%%%%%%%%%%%%%%%%%%%%%%%%%%%%%
%%%% Para firmar
\clearpage
\pagenumbering{gobble}
\chapter*{}

\vspace{-4cm}
\begin{center}
\LARGE
\textbf{Trabajo Fin de Grado}

\vspace{1cm}
\large
Título del Trabajo con Letras Capitales para Sustantivos y Adjetivos

\vspace{1cm}
\large
\textbf{Autora :} Laura González Fernández  \\
\textbf{Tutor/a :} Dr. Nombre del profesor/a

\end{center}

\vspace{1cm}
La defensa del presente Proyecto Fin de Grado se realizó el día 3\qquad$\;\,$ de
\qquad\qquad\qquad\qquad \newline de 20XX, siendo calificada por el siguiente tribunal:


\vspace{0.5cm}
\textbf{Presidente:}

\vspace{0.8cm}
\textbf{Secretario:}

\vspace{0.8cm}
\textbf{Vocal:}


\vspace{0.8cm}
y habiendo obtenido la siguiente calificación:

\vspace{0.8cm}
\textbf{Calificación:}


\vspace{0.8cm}
\begin{flushright}
Móstoles/Fuenlabrada, a \qquad$\;\,$ de \qquad\qquad\qquad\qquad de 20XX
\end{flushright}

%%%%%%%%%%%%%%%%%%%%%%%%%%%%%%%%%%%%%%%%%%%%%%%%%%%%%%%%%%%%%%%%%%%%%%%%%%%%%%%%
%%%% Dedicatoria

\chapter*{}
%\pagenumbering{Roman} % para comenzar la numeración de paginas en numeros romanos
\begin{flushright}
\textit{Aquí normalmente \\
se inserta una dedicatoria corta \\}
\end{flushright}

%%%%%%%%%%%%%%%%%%%%%%%%%%%%%%%%%%%%%%%%%%%%%%%%%%%%%%%%%%%%%%%%%%%%%%%%%%%%%%%%
%%%% Agradecimientos

\chapter*{Agradecimientos}
%\addcontentsline{toc}{chapter}{Agradecimientos} % si queremos que aparezca en el índice
\markboth{AGRADECIMIENTOS}{AGRADECIMIENTOS} % encabezado 

Aquí vienen los agradecimientos\ldots

Hay más espacio para explayarse y explicar a quién agradeces su apoyo o ayuda para
haber acabado el proyecto: familia, pareja, amigos, compañeros de clase\ldots

También hay quien, en algunos casos, hasta agradecer a su tutor o tutores del proyecto
la ayuda prestada\ldots

%%%%%%%%%%%%%%%%%%%%%%%%%%%%%%%%%%%%%%%%%%%%%%%%%%%%%%%%%%%%%%%%%%%%%%%%%%%%%%%%
%%%% Resumen

\chapter*{Resumen}
%\addcontentsline{toc}{chapter}{Resumen} % si queremos que aparezca en el índice
\markboth{RESUMEN}{RESUMEN} % encabezado

Aquí viene un resumen del proyecto.
Ha de constar de tres o cuatro párrafos, donde se presente de manera clara y concisa de qué va el proyecto. 
Han de quedar respondidas las siguientes preguntas:

\begin{itemize}
  \item ¿De qué va este proyecto? ¿Cuál es su objetivo principal?
  \item ¿Cómo se ha realizado? ¿Qué tecnologías están involucradas?
  \item ¿En qué contexto se ha realizado el proyecto? ¿Es un proyecto dentro de un marco general?
\end{itemize}

Lo mejor es escribir el resumen al final.
\\\noindent\rule{\textwidth}{0.4pt}
Resumen provisional:

El aprendizaje automático se ha desarrollado a pasos de gigante durante los últimos años y sus avances han permitido que los sistemas informáticos sean capaces de abstraer relaciones entre objetos y hacer predicciones sobre el futuro. Pero los grandes proyectos de aprendizaje requieren de la utilización de grandes cantidades de datos y los correspondientes potentes procesadores que sean capaces de trabajar con ellos. En el panorama actual cada vez más se vuelve una responsabilidad ser consciente de la huella causada en el medio ambiente por el consumo energético de los ordenadores.

Con este proyecto se pretende proporcionar una herramienta capaz de comparar entre los modelos más comunes empleados en aprendizaje automático, que pueda dar respuesta a la preocupación de equilibrar la precisión y eficacia de los modelos de aprendizaje con su impacto desde el punto de vista energético. Para ello se utilizarán las herramientas proporcionadas por Scikit-learn, Python y Codecarbon y conjuntos de datos representativos disponibles públicamente en la web con licencia abierta.


%%%%%%%%%%%%%%%%%%%%%%%%%%%%%%%%%%%%%%%%%%%%%%%%%%%%%%%%%%%%%%%%%%%%%%%%%%%%%%%%
%%%% Resumen en inglés

\chapter*{Summary}
%\addcontentsline{toc}{chapter}{Summary} % si queremos que aparezca en el índice
\markboth{SUMMARY}{SUMMARY} % encabezado

Here comes a translation of the ``Resumen'' into English. 
Please, double check it for correct grammar and spelling.
As it is the translation of the ``Resumen'', which is supposed to be written at the end, this as well should be filled out just before submitting.

%%%%--------------------------------------------------------------------
% Lista de comentarios de revisión
% Se puede borrar este bloque al acabar el borrador

%\listoftodos
%\markboth{TODO LIST}{TODO LIST} % encabezado
%%%%--------------------------------------------------------------------

%%%%%%%%%%%%%%%%%%%%%%%%%%%%%%%%%%%%%%%%%%%%%%%%%%%%%%%%%%%%%%%%%%%%%%%%%%%%%%%%
%%%%%%%%%%%%%%%%%%%%%%%%%%%%%%%%%%%%%%%%%%%%%%%%%%%%%%%%%%%%%%%%%%%%%%%%%%%%%%%%
% ÍNDICES %
%%%%%%%%%%%%%%%%%%%%%%%%%%%%%%%%%%%%%%%%%%%%%%%%%%%%%%%%%%%%%%%%%%%%%%%%%%%%%%%%

% Las buenas noticias es que los índices se generan automáticamente.
% Lo único que tienes que hacer es elegir cuáles quieren que se generen,
% y comentar/descomentar esa instrucción de LaTeX.

%%-- Índice de contenidos
\tableofcontents 
\cleardoublepage
%%-- Índice de figuras
%\addcontentsline{toc}{chapter}{Lista de figuras} % para que aparezca en el indice de contenidos
\listoffigures % indice de figuras
%\cleardoublepage
%%-- Índice de tablas
%\addcontentsline{toc}{chapter}{Lista de tablas} % para que aparezca en el indice de contenidos
%\listoftables % indice de tablas
\cleardoublepage
%%-- Índice de fragmentos de código
\listoflistings

%%%%%%%%%%%%%%%%%%%%%%%%%%%%%%%%%%%%%%%%%%%%%%%%%%%%%%%%%%%%%%%%%%%%%%%%%%%%%%%%
%%%%%%%%%%%%%%%%%%%%%%%%%%%%%%%%%%%%%%%%%%%%%%%%%%%%%%%%%%%%%%%%%%%%%%%%%%%%%%%%
% INTRODUCCIÓN %
%%%%%%%%%%%%%%%%%%%%%%%%%%%%%%%%%%%%%%%%%%%%%%%%%%%%%%%%%%%%%%%%%%%%%%%%%%%%%%%%

%\cleardoublepage
\chapter{Introducción}
\label{sec:intro}
\pagenumbering{arabic} % para empezar la numeración de página con números

%En este capítulo se introduce el proyecto.
%Debería tener información general sobre el mismo, dando la información sobre el contexto en el que se ha desarrollado.

%No te olvides de echarle un ojo a la página con los cinco errores de escritura más frecuentes\footnote{\url{http://www.tallerdeescritores.com/errores-de-escritura-frecuentes}}.

%Aconsejo a todo el mundo que mire y se inspire en memorias pasadas.
%Las memorias de los proyectos que he llevado yo están (casi) todas almacenadas en mi web del GSyC\footnote{\url{https://gsyc.urjc.es/~grex/pfcs/}}.

%%%%%%%%%%%%%%%%%%%

El aprendizaje automático (\emph{Machine Learning} en inglés) es una rama de la inteligencia artificial y la ciencia computacional que se centra en el uso de datos y algoritmos para imitar la forma en la que los humanos aprenden con el objetivo de aumentar gradualmente su precisión.
Es un componente fundamental del campo de la ciencia de datos, cuya importancia ha experimentado un gran crecimiento recientemente. 
El aprendizaje automático hace uso de métodos estadísticos para entrenar algoritmos que hacen clasificaciones o predicciones y que permiten descubrir piezas clave de información dentro de proyectos de procesamiento de datos. 
Esta información afecta posteriormente en la toma de decisiones dentro de distintas aplicaciones y negocios, con una gran capacidad de impactar en el crecimiento de los mismos.

\section{Aprendizaje automático y consumo energético}

Gracias al desarrollo de nuevas tecnologías computacionales, el aprendizaje automático que se utiliza hoy en día es muy diferente de como era en el pasado.
Este surgió del reconocimiento de patrones y de la teoría de que los ordenadores pueden aprender sin necesidad de ser específicamente programados para resolver tareas específicas, cuando los investigadores interesados en la inteligencia artificial se empezaron a plantear si los ordenadores serían capaces de aprender a partir de datos.
De aquí surge la importancia del aspecto iterativo de este aprendizaje, que permite que el sistema se adapte de forma independiente cada vez que nuevos datos son incorporados y sea capaz de aprender de cada computación previa para producir decisiones y resultados que sean confiables y repetibles.

Es así que, aunque una gran parte de los algoritmos utilizados en el aprendizaje automático son conocidos desde hace relativamente bastante tiempo, la habilidad de automáticamente aplicar complejos cálculos matemáticos a grandes cantidades de datos una y otra vez, cada vez más rápidamente, es un desarrollo muy reciente conseguido gracias a los avances en componentes informáticos y la disminución de costes de grandes sistemas computacionales con enormes capacidades de memoria y procesamiento.
Este es especialmente el caso para el campo del aprendizaje profundo (\emph{deep learning}), donde los modelos han crecido en cálculos para llegar a alcanzar típicamente el orden de los GigaFlops y en requisitos de memoria que se encuentran típicamente en el orden de los millones de parámetros.

Sin embargo, este gran poder de procesamiento trae consigo un gran gasto energético.
El consumo de energía en la arquitectura de computadores ha sido foco de atención de investigadores interesados en obtener procesadores eficientes energéticamente de última generación durante décadas. 
Por otro lado, los investigadores interesados en el aprendizaje automático se han centrado principalmente en la producción de modelos cada vez más profundos y precisos, sin poner ningún límite en términos computacionales más allá de la disponibilidad de procesadores capaces.

% Some awareness in energy consumption is starting to arise, originating from a few machine learning research groups [12], [14], [47], [61] and challenges such as The Low Power Image Recognition Challenge (LPIRC) [26]. Thus, we believe that efforts towards estimating energy consumption and developing tools for researchers to advance their research in energy consumption are necessary for a more scalable and sustainable future. %

% MORE Por qué clasificación, menciones a otros proyectos de análisis energético

%%-- Objetivos del  proyecto
%%-- Si la sección anterior ha quedado muy extensa, se puede considerar convertir
%%-- Las siguientes tres secciones en un capítulo independiente de la memoria

\section{Objetivos del proyecto}
\label{sec:objetivos}

\subsection{Objetivo general} % título de subsección (se muestra)
\label{sec:objetivo-general} % identificador de subsección (no se muestra, es para poder referenciarla)


Este Trabajo de Fin de Grado tiene como objetivo crear una herramienta que permita la comparación sistemática del consumo energético y el impacto de la huella de carbono en los modelos más representativos de técnicas de clasificación de aprendizaje automático supervisado.


\subsection{Objetivos específicos}
\label{sec:objetivos-especificos}

Para lograr esta meta se han tenido en cuenta los siguientes objetivos específicos:

    \begin{itemize}
        \item Estudiar los algoritmos de clasificación más importantes.
        \item Analizar la relación entre la precisión de los modelos y su consumo energético en conjuntos de datos de distintos tamaños y características.
        \item 
    \end{itemize}

\section{Planificación temporal}
\label{sec:planificacion-temporal}

Es conveniente que incluyas una descripción de lo que te ha llevado realizar el trabajo.
Hay gente que añade un diagrama de GANTT.
Lo importante es que quede claro cuánto tiempo has consumido en realizar el TFG/TFM 
(tiempo natural, p.ej., 6 meses) y a qué nivel de esfuerzo (p.ej., principalmente los 
fines de semana).

\section{Estructura de la memoria}
\label{sec:estructura}

Por último, en esta sección se introduce a alto nivel la organización del resto del documento
y qué contenidos se van a encontrar en cada capítulo.

    \begin{itemize}
      \item En el primer capítulo se hace una breve introducción al proyecto, se describen los objetivos del mismo y se refleja la planificación temporal.
      \item En el siguiente capítulo se describen las tecnologías utilizadas en el desarrollo de este TFM/TFG (Capítulo~\ref{chap:tecnologias}).
      \item En el capítulo~\ref{chap:diseño} Se describe el proceso de desarrollo
      de la herramienta \ldots
      \item En el capítulo~\ref{chap:experimentos} Se presentan las principales pruebas realizadas
      para validación de la plataforma/herramienta\ldots (o resultados de los experimentos
      efectuados).
      \item Por último, se presentan las conclusiones del proyecto así como los trabajos futuros que podrían derivarse de éste (Capítulo~\ref{chap:conclusiones}).
    \end{itemize}

%%%%%%%%%%%%%%%%%%%%%%%%%%%%%%%%%%%%%%%%%%%%%%%%%%%%%%%%%%%%%%%%%%%%%%%%%%%%%%%
%%%%%%%%%%%%%%%%%%%%%%%%%%%%%%%%%%%%%%%%%%%%%%%%%%%%%%%%%%%%%%%%%%%%%%%%%%%%%%%
%%%%%%%%%%%%%%%%%%%          NOTES             %%%%%%%%%%%%%%%%%%%%%%%%%%%%%%%%
%%%%%%%%%%%%%%%%%%%%%%%%%%%%%%%%%%%%%%%%%%%%%%%%%%%%%%%%%%%%%%%%%%%%%%%%%%%%%%%
%%%%%%%%%%%%%%%%%%%%%%%%%%%%%%%%%%%%%%%%%%%%%%%%%%%%%%%%%%%%%%%%%%%%%%%%%%%%%%%
\begin{comment}
\section{Sección}
\label{sec:seccion}

Esto es una sección, que es una estructura menor que un capítulo. 

Por cierto, a veces me comentáis que no os compila por las tildes.
Eso es un problema de codificación.
Al guardar el archivo, guardad la codificación de ``ISO-Latin-1'' a ``UTF-8'' (o viceversa) y funcionará.

\subsection{Estilo}
\label{subsec:estilo}

Recomiendo leer los consejos prácticos sobre escribir documentos científicos en \LaTeX \ de Diomidis Spinellis\footnote{\url{https://github.com/dspinellis/latex-advice}}.

Lee sobre el uso de las comas\footnote{\url{http://narrativabreve.com/2015/02/opiniones-de-un-corrector-de-estilo-11-recetas-para-escribir-correctamente-la-coma.html}}. 
Las comas en español no se ponen al tuntún.
Y nunca, nunca entre el sujeto y el predicado (p.ej. en ``Yo, hago el TFG'' sobre la coma).
La coma no debe separar el sujeto del predicado en una oración, pues se cortaría la secuencia natural del discurso.
No se considera apropiado el uso de la llamada coma respiratoria o \emph{coma criminal}.
Solamente se suele escribir una coma para marcar el lugar que queda cuando omitimos el verbo de una oración, pero es un caso que se da de manera muy infrecuente al escribir un texto científico (p.ej. ``El Real Madrid, campeón de Europa'').

A continuación, viene una figura, la Figura~\ref{figura:foro_hilos}. 
Observarás que el texto dentro de la referencia es el identificador de la figura (que se corresponden con el ``label'' dentro de la misma). 
También habrás tomado nota de cómo se ponen las ``comillas dobles'' para que se muestren correctamente. 
Nota que hay unas comillas de inicio (``) y otras de cierre (''), y que son diferentes.
Volviendo a las referencias, nota que al compilar, la primera vez se crea un diccionario con las referencias, y en la segunda compilación se ``rellenan'' estas referencias. 
Por eso hay que compilar dos veces tu memoria.
Si no, no se crearán las referencias.

 \begin{figure}
    \centering
    \includegraphics[bb=0 0 800 600, width=12cm, keepaspectratio]{img/foro1}
    \caption{Página con enlaces a hilos}
    \label{figura:foro_hilos}
 \end{figure}

A continuación un bloque ``verbatim'', que se utiliza para mostrar texto tal cual.
Se puede utilizar para ofrecer el contenido de correos electrónicos, código, entre otras cosas.

{\footnotesize
\begin{verbatim}
    From gaurav at gold-solutions.co.uk  Fri Jan 14 14:51:11 2005
    From: gaurav at gold-solutions.co.uk (gaurav_gold)
    Date: Fri Jan 14 19:25:51 2005
    Subject: [Mailman-Users] mailman issues
    Message-ID: <003c01c4fa40$1d99b4c0$94592252@gaurav7klgnyif>
    Dear Sir/Madam,
    How can people reply to the mailing list?  How do i turn off
    this feature? How can i also enable a feature where if someone
    replies the newsletter the email gets deleted?
    Thanks
    From msapiro at value.net  Fri Jan 14 19:48:51 2005
    From: msapiro at value.net (Mark Sapiro)
    Date: Fri Jan 14 19:49:04 2005
    Subject: [Mailman-Users] mailman issues
    In-Reply-To: <003c01c4fa40$1d99b4c0$94592252@gaurav7klgnyif>
    Message-ID: <PC173020050114104851057801b04d55@msapiro>
    gaurav_gold wrote:
    >How can people reply to the mailing list?  How do i turn off
    this feature? How can i also enable a feature where if someone
    replies the newsletter the email gets deleted?
    See the FAQ
    >Mailman FAQ: http://www.python.org/cgi-bin/faqw-mm.py
    article 3.11
\end{verbatim}
}

\end{comment}

%\cleardoublepage

%%%%%%%%%%%%%%%%%%%%%%%%%%%%%%%%%%%%%%%%%%%%%%%%%%%%%%%%%%%%%%%%%%%%%%%%%%%%%%%%
%%%%%%%%%%%%%%%%%%%%%%%%%%%%%%%%%%%%%%%%%%%%%%%%%%%%%%%%%%%%%%%%%%%%%%%%%%%%%%%%
% ESTADO DEL ARTE %
%%%%%%%%%%%%%%%%%%%%%%%%%%%%%%%%%%%%%%%%%%%%%%%%%%%%%%%%%%%%%%%%%%%%%%%%%%%%%%%%

\chapter{Estado del arte}               %% a.k.a "Tecnologías utilizadas"
\label{chap:tecnologias}

Descripción de las tecnologías que utilizas en tu trabajo. 
Con dos o tres párrafos por cada tecnología, vale. 
Se supone que aquí viene todo lo que no has hecho tú.

Puedes citar libros, como el de Bonabeau et al., sobre procesos estigmérgicos~\cite{bonabeau:_swarm}. 
Me encantan los procesos estigmérgicos.
Deberías leer más sobre ellos.
Pero quizás no ahora, que tenemos que terminar la memoria para sacarnos por fin el título.
Nota que el \~ \ añade un espacio en blanco, pero no deja que exista un salto de línea. 
Imprescindible ponerlo para las citas.

Citar es importantísimo en textos científico-técnicos. 
Porque no partimos de cero.
Es más, partir de cero es de tontos; lo suyo es aprovecharse de lo ya existente para construir encima y hacer cosas más sofisticadas.
¿Dónde puedo encontrar textos científicos que referenciar?
Un buen sitio es Google Scholar\footnote{\url{http://scholar.google.com}}.
Por ejemplo, si buscas por ``stigmergy libre software'' para encontrar trabajo sobre software libre y el concepto de \emph{estigmergia} (¿te he comentado que me gusta el concepto de estigmergia ya?), encontrarás un artículo que escribí hace tiempo cuyo título es ``Self-organized development in libre software: a model based on the stigmergy concept''.
Si pulsas sobre las comillas dobles (entre la estrella y el ``citado por ...'', justo debajo del extracto del resumen del artículo, te saldrá una ventana emergente con cómo citar.
Abajo a la derecha, aparece un enlace BibTeX.
Púlsalo y encontrarás la referencia en formato BibTeX, tal que así:

\clearpage
{\footnotesize
\begin{minted}{bibtex}
@inproceedings{robles2005self,
  title={Self-organized development in libre software:
         a model based on the stigmergy concept},
  author={Robles, Gregorio and Merelo, Juan Juli\'an 
          and Gonz\'alez-Barahona, Jes\'us M.},
  booktitle={ProSim'05},
  year={2005}
}
\end{minted}
}

Copia el texto en BibTeX y pégalo en el fichero \texttt{memoria.bib}, que es donde están las referencias bibliográficas.
Para incluir la referencia en el texto de la memoria, deberás citarlo, como hemos hecho antes con~\cite{bonabeau:_swarm}, lo que pasa es que en vez de el identificador de la cita anterior (bonabeau:\_swarm), tendrás que poner el nuevo (robles2005self).
Compila el fichero \texttt{memoria.tex} (\texttt{pdflatex memoria.tex}), añade la bibliografía (\texttt{bibtex memoria.aux}) y vuelve a compilar \texttt{memoria.tex} (\texttt{pdflatex memoria.tex})\ldots y \emph{voilà} ¡tenemos una nueva cita~\cite{robles2005self}!

También existe la posibilidad de poner notas al pie de página, por ejemplo, una para indicarte que visite la página del GSyC\footnote{\url{http://gsyc.es}}.

\section{Sección 1} 
\label{sec:seccion1}

Hemos hablado de cómo incluir figuras, pero no se ha descrito cómo incluir tablas.
A continuación se presenta un ejemplo de tabla, la Tabla \ref{tabla:ejemplo} (fíjate 
en cómo se introduce una referencia a la tabla).

\begin{table}
 \begin{center}
  \begin{tabular}{ | l | c | r |} % tenemos tres colummnas, la primera alineada a la izquierda (l), la segunda al centro (c) y la tercera a la derecha (r). El | indica que entre las columnas habrá una línea separadora.
    \hline
    Uno & 2 & 3 \\ \hline % el hline nos da una línea vertical
    Cuatro & 5 & 6 \\ \hline
    Siete & 8 & 9 \\
    \hline
  \end{tabular}
  \caption{Ejemplo de tabla. Aquí viene una pequeña descripción (el \emph{caption}) del contenido de la tabla. Si la tabla no es autoexplicativa, siempre viene bien aclararla aquí.}
  \label{tabla:ejemplo}
 \end{center}
\end{table}

\section{Entorno de desarrollo: PyCharm}
\label{sec:entorno_de_desarrollo}

%%-- El comando \gls{} permite incluir términos en el glosario, para luego reunirlos todos
%%-- en una tabla al comienzo o al final del documento, junto con sus definiciones.

PyCharm es un \gls{ide} dedicado concretamente a la programación en Python y desarrollado por la compañía checa JetBrains.

Proporciona análisis de código, un depurador gráfico, una consola de Python integrada, control de versiones y, además, soporta desarrollo web con Django. Todas estas características lo convierten en un entorno completo e intuitivo, idóneo para el desarrollo de proyectos académicos como el que nos ocupa.


\section{Redacción de la memoria: LaTeX/Overleaf}
\label{sec:redaccion_de_la_memoria}

LaTeX es un sistema de composición tipográfica de alta calidad que incluye características especialmente diseñadas para la producción de documentación técnica y científica. Estas características, entre las que se encuentran la posibilidad de incluir expresiones matemáticas, fragmentos de código, tablas y referencias, junto con el hecho de que se distribuya como software libre, han hecho que LaTeX se convierta en el estándar de facto para la redacción y publicación de artículos académicos, tesis y todo tipo de documentos científico-técnicos. 

Por su parte, Overleaf es un editor LaTeX colaborativo basado en la nube. Lanzado originalmente en 2012, fue creado por dos matemáticos que se inspiraron en su propia experiencia en el ámbito académico para crear una solución satisfactoria para la escritura científica colaborativa.

Además de por su perfil colaborativo, Overleaf destaca porque, pese a que en LaTeX el escritor utiliza texto plano en lugar de texto formateado (como ocurre en otros procesadores de texto como Microsoft Word, LibreOffice Writer y Apple Pages), éste puede visualizar en todo momento y paralelamente el texto formateado que resulta de la escritura del código fuente.

%\cleardoublepage

%%%%%%%%%%%%%%%%%%%%%%%%%%%%%%%%%%%%%%%%%%%%%%%%%%%%%%%%%%%%%%%%%%%%%%%%%%%%%%%%
%%%%%%%%%%%%%%%%%%%%%%%%%%%%%%%%%%%%%%%%%%%%%%%%%%%%%%%%%%%%%%%%%%%%%%%%%%%%%%%%
% DISEÑO E IMPLEMENTACIÓN %
%%%%%%%%%%%%%%%%%%%%%%%%%%%%%%%%%%%%%%%%%%%%%%%%%%%%%%%%%%%%%%%%%%%%%%%%%%%%%%%%

\include{files/3-diseño.tex}

%%%%%%%%%%%%%%%%%%%%%%%%%%%%%%%%%%%%%%%%%%%%%%%%%%%%%%%%%%%%%%%%%%%%%%%%%%%%%%%%
%%%%%%%%%%%%%%%%%%%%%%%%%%%%%%%%%%%%%%%%%%%%%%%%%%%%%%%%%%%%%%%%%%%%%%%%%%%%%%%%
% EXPERIMENTOS Y VALIDACIÓN %
%%%%%%%%%%%%%%%%%%%%%%%%%%%%%%%%%%%%%%%%%%%%%%%%%%%%%%%%%%%%%%%%%%%%%%%%%%%%%%%%

\chapter{Experimentos y validación}
\label{chap:experimentos}

El objetivo de este capítulo es mostrar el funcionamiento de la aplicación en un caso de uso real en el que se tratará de extraer conclusiones generales acerca del consumo eléctrico de cada modelo y de si este consumo irá necesariamente acompañado de una mejora de los resultados de predicción.
Para ello se emplearán las herramientas descritas anteriormente para evaluar el consumo y el rendimiento de una serie de modelos formada por representantes de las principales familias de modelos de aprendizaje automático y recogidos en la sección~\ref{sec:models}. Estos modelos serán aplicados a los conjuntos de datos de distintas características definidos en la sección~\ref{sec:datasets}.

Durante la validación de la aplicación se llevarán acabo tres experimentos distintos.
El primero examinará el consumo energético en base al modelo seleccionado. En esta sección se tomarán varias medidas de consumo y rendimiento por modelo y conjunto de datos en una máquina con unos recursos de procesamiento concretos para analizar que modelos consumen más que otros y que características de los conjuntos de datos hacen incrementar este consumo.
El segundo consistirá en aislar un par de conjuntos de datos y tomar medidas de consumo con distintos recursos de procesado dedicados a la tarea de aprendizaje automático para observar el efecto de los recursos disponibles en el consumo energético de cada modelo.
Por último, se propondrán métodos de optimización de los modelos analizados y se examinará el efecto que pueda tener sobre su consumo. 

A través de este análisis, se pretende obtener una comprensión profunda de cómo diferentes modelos de aprendizaje automático consumen energía bajo diversas condiciones de trabajo. Este experimento también busca identificar patrones de consumo y eficiencia que puedan informar el diseño y la implementación de modelos más sostenibles y eficientes en el futuro.

\todo[inline]{Añadir esquema ???}
% What's the purpose of experiments?
% What are the expected results? More energy, more precision
% Outline / procedure / steps to follow

% 4.1 Análisis del consumo energético en base al modelo escogido
    % 4.1.1 Comparación en conjuntos de pequeño tamaño (100s - 1000s)
        % Tres conjuntos: iris, ionosphere, hepatitis
    % 4.1.2 Comparación en conjuntos de mediano tamaño (10000s)
        % Dos conjuntos: eeg-eye-state, electricity, letter, mnist_784
% 4.2 Análisis del consumo energético en base a los recursos disponibles
    % 4.2.1 Evolución del consumo con la carga del procesador
        % 1 dataset pequeño, 1 mediano
    % 4.2.2 Evolución del consumo con el aumento de recursos
        % 1 dataset grande (100000s) ?covertype?, 2-3 resource configs
% 4.3 Optimización

\section{Consumo energético basado en el modelo seleccionado}
 % - El consumo aumenta al aumentar el número de muestras
 %    1. Gráfico introductorio: número de muestras (x) vs emisiones (y), muchos modelos
 %        el consumo aumenta de forma exponencial con el número de muestras, unos modelos aumentan más que otros
 %    2. Introduce f-score: plot same lines with average f1-score instead of emissions. Los resultados son distintos, mayor consumo no implica mejor predicción
 %    3. Introduce scatter plot 4-way
    
 % - Algunos modelos son mejores que otros
 %    - Aumento de score implica aumento de consumo?
 %    - Compara average f1-score con consumo por modelo y dataset
 %    3. Introduce f-score con scatter plot 4-way, all models, 3 datasets (no average)
 %    4. Bar plot de dos datasets pequeños comparando score 

En esta sección se examinará el consumo energético una serie de modelos representativos aplicados a varios conjuntos de datos. El objetivo de este análisis será abordar las siguientes cuestiones clave:

\begin{itemize}
    \item Identificación de los modelos con mayor consumo energético.
    \item Determinación de los modelos cuyo consumo energético incrementa significativamente al aumentar el número de muestras.
    \item Evaluación de modelos que ofrecen mejores predicciones con menor consumo energético.
\end{itemize}

Dónde sea posible, se tratará de analizar estas cuestiones de forma general y obtener conclusiones que sean extrapolables más allá de los conjuntos de datos concretos que se hayan medido. Sin embargo, debido a la gran cantidad de variables involucradas en las variaciones de consumo entre unos casos y otros, es posible en otros conjuntos de datos se observen comportamientos distintos del consumo.

Para analizar estas cuestiones todas las medidas de consumo serán tomadas con la aplicación desarrollada ejecutando en una misma máquina. Para cada modelo y conjunto de datos, se tomarán medidas de consumo y rendimiento utilizando validación cruzada con cinco iteraciones con un tamaño definido para los datos de testeo del 20\% del conjunto de datos. Esta técnica proporcionará una evaluación robusta y precisa tanto del comportamiento energético de los modelos como de su precisión y exactitud, ya que evitará en gran medida la presencia de valores atípicos y el riesgo de sobreajuste de los modelos.

\begin{table}[h]
    \centering
    \begin{tabular}{rl}
         Modelo & Dell XPS 15 9500\\
         Sistema Operativo & Ubuntu 20.04.6 LTS x86\_64\\
         Python & 3.12.2\\
         Procesador & Intel(R) Core(TM) i9-10885H CPU @ 2.40GHz\\
         Memoria & 7,63 GB\\
    \end{tabular}
    \caption{Características técnicas de la máquina utilizada para tomar las medidas}
    \label{tab:caracteristicas-tecnicas}
\end{table}

La aplicación será ejecutada con el siguiente comando para cada conjunto de datos distinto, en el cual \texttt{[dataset]} será sustituido por el archivo que contenga cada conjunto de datos. Adicionalmente, cualquiera de las opciones de lectura de datos descritas en la sección~\ref{sec:limpieza} podrá ser utilizada si el formato en el que se encuentren los datos lo requiere. Las características de la máquina utilizada están recogidas en la tabla~\ref{tab:caracteristicas-tecnicas}.
\begin{minted}{bash}
mlcost measure --log -cv 5 -d [dataset] [dataset-options]
\end{minted}

La ejecución de este comando producirá un archivo tipo tabla de datos en formato \texttt{.csv} con filas de medidas por modelo para el conjunto de datos especificado. Cada una de estas filas corresponderá a las medidas tomadas durante una iteración de la validación cruzada. Estas medidas incluirán el consumo energético y tiempo empleado en entrenar el modelo en el conjunto de datos de entrenamiento separado para esa iteración concreta, además de la exactitud, la precisión, la exhaustividad y el valor-F calculados en el conjunto de datos de prueba restante de acuerdo a la definición mostrada en la sección~\ref{sec:scoring}.

La tabla~\ref{tab:medidas-1} recoge muestra un extracto de las medidas tomadas. El archivo completo está disponible en el repositorio de la aplicación.

\begin{table}[h]
\centerline{
\scalebox{0.78}{
\begin{tabular}{|llllllllllll|}
\hline
Dataset     & Modelo & CPU & Accuracy & Precision & F-score & Recall & Fit  & Total (s) & Emisiones & Energía  & Muestras \\
 &  & load (\%) &  & & & &  time (s) & &  (kg) &  (kWh) &  \\ \hline
Banknote    & Linear & 2.7           & 0.98      & 0.98      & 0.98    & 0.98          & 0.007             & 0.071            & 2.13E-07  & 1.10E-06 & 1372     \\
Banknote    & Linear & 2.7           & 0.97      & 0.97      & 0.97    & 0.97          & 0.006             & 0.071            & 2.13E-07  & 1.10E-06 & 1372     \\
Banknote    & Linear & 2.7           & 0.97      & 0.97      & 0.97    & 0.97          & 0.006             & 0.071            & 2.13E-07  & 1.10E-06 & 1372     \\
Banknote    & Linear & 2.7           & 0.99      & 0.99      & 0.99    & 0.99          & 0.005             & 0.071            & 2.13E-07  & 1.10E-06 & 1372     \\
Banknote    & Linear & 2.7           & 0.99      & 0.99      & 0.99    & 0.99          & 0.005             & 0.071            & 2.13E-07  & 1.10E-06 & 1372     \\
Banknote    & Forest & 2.7           & 0.99      & 0.99      & 0.99    & 0.99          & 0.184             & 1.429            & 3.27E-06  & 1.69E-05 & 1372     \\
Banknote    & Forest & 2.7           & 1.00      & 1.00      & 1.00    & 1.00          & 0.171             & 1.429            & 3.27E-06  & 1.69E-05 & 1372     \\
Banknote    & Forest & 2.7           & 0.99      & 0.99      & 0.99    & 0.99          & 0.154             & 1.429            & 3.27E-06  & 1.69E-05 & 1372     \\
Banknote    & Forest & 2.7           & 1.00      & 1.00      & 1.00    & 1.00          & 0.172             & 1.429            & 3.27E-06  & 1.69E-05 & 1372     \\
Banknote    & Forest & 2.7           & 1.00      & 1.00      & 1.00    & 1.00          & 0.158             & 1.429            & 3.27E-06  & 1.69E-05 & 1372     \\
\multicolumn{12}{|c|}{...} \\
Electricity & Neural & 102.4         & 0.82      & 0.83      & 0.83    & 0.83          & 132.768           & 518.905          & 1.19E-03  & 6.15E-03 & 45312 \\  \hline
\end{tabular}}}
\caption[Extracto de los resultados de entrenamiento]{Extracto de los resultados de entrenamiento\footnote{\url{https://github.com/l-gonz/tfg-gitt-mlcost/blob/main/model-comp-many.csv}}
\todo[inline]{Fix format}}
\label{tab:medidas-1}
\end{table}



\clearpage

%%%%%%%%%%%%%%%%%%%%%%%%%%%%%%%%%%%%%%%%%%%%%%%%%%%%%%%%%%%%%%%%%%%%%%%%%%%%%%%%
%%%%%%%%%%%%%%%%%%%%%%%%%%%%%%%%%%%%%%%%%%%%%%%%%%%%%%%%%%%%%%%%%%%%%%%%%%%%%%%%
% CONCLUSIONES %
%%%%%%%%%%%%%%%%%%%%%%%%%%%%%%%%%%%%%%%%%%%%%%%%%%%%%%%%%%%%%%%%%%%%%%%%%%%%%%%%

\chapter{Conclusiones y trabajos futuros}
\label{chap:conclusiones}


\section{Consecución de objetivos}
\label{sec:consecucion-objetivos}

Esta sección es la sección espejo de las dos primeras del capítulo de objetivos, donde se planteaba el objetivo general y se elaboraban los específicos.

Es aquí donde hay que debatir qué se ha conseguido y qué no. 
Cuando algo no se ha conseguido, se ha de justificar, en términos de qué problemas se han encontrado y qué medidas se han tomado para mitigar esos problemas.

Y si has llegado hasta aquí, siempre es bueno pasarle el corrector ortográfico, que las erratas quedan fatal en la memoria final.
Para eso, en Linux tenemos aspell, que se ejecuta de la siguiente manera desde la línea de \emph{shell}:

\begin{minted}{bash}
  aspell --lang=es_ES -c memoria.tex
\end{minted}

\section{Aplicación de lo aprendido}
\label{sec:aplicacion}

Aquí viene lo que has aprendido durante el Grado/Máster y que has aplicado en el TFG/TFM. Una buena idea es poner las asignaturas más relacionadas y comentar en un párrafo los conocimientos y habilidades puestos en práctica.

\begin{enumerate}
  \item a
  \item b
\end{enumerate}


\section{Lecciones aprendidas}
\label{sec:lecciones_aprendidas}

Aquí viene lo que has aprendido en el Trabajo Fin de Grado/Máster.

\begin{enumerate}
  \item Aquí viene uno.
  \item Aquí viene otro.
\end{enumerate}


\section{Trabajos futuros}
\label{sec:trabajos_futuros}

Ningún proyecto ni software se termina, así que aquí vienen ideas y funcionalidades que estaría bien tener implementadas en el futuro.

Es un apartado que sirve para dar ideas de cara a futuros TFGs/TFMs.

%%%%%%%%%%%%%%%%%%%%%%%%%%%%%%%%%%%%%%%%%%%%%%%%%%%%%%%%%%%%%%%%%%%%%%%%%%%%%%%%
%%%%%%%%%%%%%%%%%%%%%%%%%%%%%%%%%%%%%%%%%%%%%%%%%%%%%%%%%%%%%%%%%%%%%%%%%%%%%%%%
% GLOSARIO(S) %
%%%%%%%%%%%%%%%%%%%%%%%%%%%%%%%%%%%%%%%%%%%%%%%%%%%%%%%%%%%%%%%%%%%%%%%%%%%%%%%%

\printglossary[type=\acronymtype]

\printglossary

%%%%%%%%%%%%%%%%%%%%%%%%%%%%%%%%%%%%%%%%%%%%%%%%%%%%%%%%%%%%%%%%%%%%%%%%%%%%%%%%
%%%%%%%%%%%%%%%%%%%%%%%%%%%%%%%%%%%%%%%%%%%%%%%%%%%%%%%%%%%%%%%%%%%%%%%%%%%%%%%%
% APÉNDICE(S) %
%%%%%%%%%%%%%%%%%%%%%%%%%%%%%%%%%%%%%%%%%%%%%%%%%%%%%%%%%%%%%%%%%%%%%%%%%%%%%%%%

%\cleardoublepage
%\appendix
%\chapter{Manual de usuario}
%\label{app:manual}


%%%%%%%%%%%%%%%%%%%%%%%%%%%%%%%%%%%%%%%%%%%%%%%%%%%%%%%%%%%%%%%%%%%%%%%%%%%%%%%%
%%%%%%%%%%%%%%%%%%%%%%%%%%%%%%%%%%%%%%%%%%%%%%%%%%%%%%%%%%%%%%%%%%%%%%%%%%%%%%%%
% BIBLIOGRAFIA %
%%%%%%%%%%%%%%%%%%%%%%%%%%%%%%%%%%%%%%%%%%%%%%%%%%%%%%%%%%%%%%%%%%%%%%%%%%%%%%%%

\cleardoublepage

%% OLD BIBTEX CODE, TO BE DELETED
%\bibliographystyle{abbrv}
%\bibliographystyle{plain} 
%\bibliography{memoria}  % memoria.bib es el nombre del fichero que contiene las referencias bibliográficas.

% https://www.overleaf.com/learn/latex/Bibliography_management_with_biblatex
\raggedright\printbibliography[heading=bibintoc,title={Referencias}]

\end{document}
